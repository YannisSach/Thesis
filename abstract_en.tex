Thorough verification and testing of concurrent programs is an important, but also
challenging task. In order to verify a concurrent program one must examine all possible 
different interleavings the scheduler can produce. Stateless model checking with Dynamic Partial Order Reduction 
is a technique proposed to deal with state space explosion. Nevertheless, for larger programs the verification 
takes longer than the developers are willing to wait. In these cases, bounded search can be proved useful. Bounded search,
in contrast to the DPOR, alleviates state-space explosion by pruning the executions that exceed a bound. 

This thesis describes the implementation of the preemption bounded DPOR (BPOR) on Nidhugg, a bug finding tool which targets bugs caused by concurrency
and relaxed memory consistency in concurrent programs. Specifically three bounding techniques were implemented: the Vanilla-BPOR, the BPOR,
and the Source-BPOR. The three techniques were evaluated both in synthetic and in real world software. Specifically Read-Copy-Update mechanism of Linux Kernel
was verified again. Moreover it is examined whether optimizations that have been suggested for the 
unbounded DPOR can improve the efficiency of BPOR.