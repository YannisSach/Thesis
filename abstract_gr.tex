H διεξοδική επαλήθευση και έλεγχος προγραμμάτων γραμμένων στο πρότυπο του ταυτοχρονισμού είναι ένα μιά τόσο σημαντική
όσο και απαιτητική εργασία. Προκειμένουν να επαληθεύσουμε την ορθότητα ενός τέτοιου προγραμμάτος θα πρέπει να εξετάσουμε
όλες τις δυνατές δρομολογήσεις που αυτό μπορεί να παράξει. Το Stateless model checking με τη χρήση αναγωγής σε δυναμικές
σχέσεις μερική διάταξης (Dynamic Partial Order Reduction ή DPOR) είναι μια τεχνική η οποία αντιμετωπίζει το πρόβλημα του της
έκρηξης του μεγέθους του χώρου καταστάσεων. Παρολ' αυτά η επαλήθευση μεγάλων προγραμμάτων μπορεί να διαρκέσει
περισσότερο από όσο θα επιθυμούσαν οι developers. Σε αυτές τις περιπτώσεις ο περιορισμός (οριοθέτηση) της αναζήτησης
(bounded search) με τη χρήση κάποιου ορίου μπορεί να αποδειχθεί αρκετά χρήσιμη. Η οριοθετημένη αναζήτηση, σε αντίθεση με την DPOR 
απαλείφει το πρόβλημα της έκρηξης του μεγέθους το χώρου καταστάσεων αγνοώντας δρομολογήσεις που ξεπερνούν κάποιο όριο

Στην παρούσα διπλωματική περιγράφεται η υλοποίηση του preemption bounded DPOR (BPOR) στο Nidhugg ένα εργαλία για την εύρεση λαθών (bugs)
που στοχεύει στην ανεύρεση σφαλμάτων που προκείπτουν από το μοντέλο του ταυτοχρονισμού και τα χαλαρά μοντέλα μνήμης (relaxed memory models).
Συγκεκριμένα υλοποιήθηκαν τρεις τεχνικές περιορισμού: ο Naive-BPOR, o Nidhugg-BPOR και ο Source-BPOR. Οι τεχνικές αυτές αξιολογήθηκαν τόσο σε 
συνθετικά τεστ όσο και σε λογισμικό που χρησιμοποιείται στην πραγματικότητα. Συγκεκριμένα ο μηχανισμός Read-Copy-Update του πυρήνα του Linux
επαληθεύτηκε ξανά με τη χρήση αυτών των τεχνικών. Επιπλέον εξετάστηκε κατα πόσο βελτιστοποιήσεις που εφαρμόζονται στον μη περιορισμένο DPOR μπορούν
να βελτιώσουν την επίδοση του BPOR.