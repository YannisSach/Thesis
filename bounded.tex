\chapter{Bounding Techniques for DPOR}

\section{Challenges}

\subsection{Conservative Branches}
It has been shown in a previous chapter that conservative sets cannot utilize the sleep set optimization. This is due to the fact that these branches are not produced by conflicts
and as a result it is impossible to "wake up" another process whose next step may be a local operation. The problem is getting even more complex considering that when a conservative branch
is added the algorithm "forgets" what was previously in this place. This lack of memory leads to an explosion of the state space. This explosion is greater than
the explosion happening when exploring the unbounded state space. 
In order to explain this better an example with a writer and 3 readers is given in Figure \ref{writer-3readers explosion}.

\trace{w3rbpor.pdf}{writer-3readers explosion}

As we can infer from the example there are more states to be explored than the expected ones. This is caused by the addition of conservative branches. 
The extra trace suchs as the r2.r1.w trace which explores the already explored r1.r2.w trace. The algorithm is not aware when adding the 
conservative trace whether an equivalent trace will be explored. The situation is worse when more readers are to read the same global variable.
The total traces explored for a big bound approach the $N!$ where $N$ the total number of threads. We can easily notice that execution seems to be aware of the previous executions since sleep sets cannot
be used. This happens due to the redundant inversions of the reading operations. 

\subsection{Sleep Sets}
The results of the various bounding algorithms suggest that the number of sleep set blocked traces is subtle compared to the number of the explored traces. This is due to the conservative
branches. It was made clear that when both a conservative and a non conservative branch of the same thread is added then the conservative branch prevails. This trace would be 
redundant in an unbounded version of the algorithm but it is not in a bounded version since the non conservative branch may had been rejected. Moreover, even if both threads had been accepted 
by the algorithm there may be a later scheduling which may be rejected when the trace was caused by the non-conservative branch and be accepted by the equivalent trace caused by the conservative
branch. 

Another "problem" with the sleep sets is that are "in favor" of the branches that increase the bound count while they block traces with lower bound count.
In the example below it is shown that the branch which would not have caused a increase of the bound is rejected.

\trace{sleepsetproblem.pdf}{Sleep set contradiction}

This behavior is quite reasonable if we consider the depth-first nature of the algorithm. As it was demonstrated in the first chapter, the DPOR algorithm performs a DPOR search. As a result
it takes the branches which are located lower in the traces where the bound count is higher since more preemptive switches have taken place. The purpose of the sleep sets is to block redundant traces, i.e. traces that have already been
explored. As a result the equivalent traces with lower bound count will be rejected.

It is clear that the problem lies on the nature of the DPOR algorithm. A method that would explore the state space in breadth first way would be unfeasible since it would
entail a huge memory overhead which stems from the storage of the traces that have not be completely explored. 

\subsection{Source Sets - Optimal DPOR}
The source set technique which can lead many times to optimal coverage of the state space manages avoid the exploration of redundant sets. However as it was discussed
in the previous section it avoids the scheduling of the traces. Unfortunately it cannot be used in the when conservative branches are added since these branches are not
related to the sleep sets. As a result the conservative branches alone will never lead to sleep set blocked traces.

However, in many test cases there are some sleep set blocked traces which are caused by conditional reads and writes. When a technique which does not utilize 
the sleep sets is to be used these traces would have been easily eliminated. Unfortunately, it was experimentally shown that explored traces outnumber the sleepset blocked
traces and, consequently, the implementation of such an algorithm would have a minor impact.

Moreover, we have shown that even if we maintain the source-set optimization for the non-conservative branches the results will be equivalent with using
persistent sets. 
The idea of keeping the rejected traces from Source-DPOR that would have been added from DPOR (with persistent sets) was rejected since it harms
the soundness of the algorithm.

\subsection {The Performance - Soundness Tradeoff}

From the previous discussion it is clear that all the bounding algorithms have to deal with a tradeoff. Some algorithms may be faster but
explore a smaller amount of traces, without even covering the full state space within the bound (Algorithm \ref{Vanilla}), 
whereas others explore the whole state space (Algorithm \ref{Nidhugg BPOR}) within the bound but require more time.


\section{Naive-BPOR}

The first bounded technique to be presented is the Naive-BPOR (Algorithm \ref{Vanilla}). The purpose of the algorithm is to block threads that exceed the bound
limit. 

\begin{algorithm}
    \caption{Naive-BPOR}
    \label{Vanilla}
    \SetKwInOut{Input}{input}
    Explore($\langle \rangle$,$\emptyset$,$b$)\;
    \Fn{Explore($E$,$Sleep$,$b$)}{
        \If{$\exists p \in (enabled(s_{[E]}) \backslash Sleep)$ such that $B_v(E.p) \leq b)$ }{
            $backtrack(E) :={p}$ \;
            \While{$\exists p \in (backtrack(E) \backslash Sleep$ and $B_v(E.p) \leq b$}{
                \ForEach{$e \in dom(E)$ such that $e \lesssim_{E.p} next_{[E]}(p) $}{
                    \Let{$E' = pre(E,e)$}
                    \Let{$u = notdep(e,E).p$}
                    \If{$I_{E'}(u) \cap backtrack(E') = \emptyset$}{
                        add some $q' \in I_{[E']}(u) to backtrack(E') $ \;
                    }
                }
                \Let{$Sleep' := \{q \in Sleep \mid E \models p \diamondsuit q \} $}
                $Explore(E.p, Sleep, b)$ \;
                add $p$ to $Sleep$ \;

            }
        }
    }
\end{algorithm}

The Algorithm \ref{Vanilla} is almost the same with Source-DPOR(Algorithm \ref{Source}). The only additions made are related to the 
thread scheduling. When a a step of a process $p$ added to $E$ result the trace $E.p : B_v(E.p) > b$ then this process is not allowed to be scheduled.
This algorithm is not sound i.e., it does not examine every trace that compensates with the bound

Lets take for example the writer-2 readers example with $b=0$ show in Figure \ref{Naive-BPOR for bound=$0$}. 

\label{Vanilla0}    
\trace{w2rvbound.pdf}{Naive-BPOR for bound=$0$}

As we can see there are 4 traces that do not exceed the bound. These are:
$p.q.q.r.r$, $q.q.p.r.r$, $r.r.p.q.q$, $q.q.r.r.p$.
However the vanilla-BPOR is not able to explore them all; $r.r.p.q.q$ is not explored.
As it was shown in the comparison of persistent and source sets, r is never registered as the first event of the trace
since this will lead to a sleep set blocked trace. The branch that would lead to an equivalent trace to $r.r.p.q.q$ is rejected
since it would have higher bound count.

\section{BPOR}

\begin{algorithm}
    \caption{BPOR}
    \label{BPOR}
    \SetKwInOut{Input}{input}
    \SetKwInput{Initialization}{Explore($ \emptyset $)}
    \SetKwHangingKw{Let}{let}
    \Fn{Explore($E$)}{
        \Let{$s := last(E)$}
        \For{all process $p$}{
            \For{all process $q \neq p$}{
            \If{$\exists i = max(\{ i \in dom(E) \mid E_i$ is dependent and may be co-enabled with $next(s,p)$ and $ E_i.tid = q \} $}{
                \uIf{$p \in enabled(pre(E,i)))$}{
                    add $p$ to $backtrack(pre(E,i))$ \;
                }
                \Else{add $enabled(pre(E,i))$ to $backtrack(pre(E,i))$ \;}
                \uIf{$j = max(\{ j \in dom(E) \mid j = 0 $ or $ S_{j-1}.tid \neq S_j.tid $ and $ j<i \})$}{
                    \uIf{$p \in enabled(pre(E,i)))$}{
                        add $p$ to $backtrack(pre(E,i))$ \;
                    }
                    \Else{add $enabled(pre(E,i))$ to $backtrack(pre(E,i))$ \;}
                }
            }
            }
        }
        \If{$p \in enabled(s)$}{
            add $p$ to $backtrack(s)$ \;
        }
        \Else{
            add any $u \in enabled(s)$ to $backtrack(s)$ \;
        }
        \Let{$visited = \emptyset $}
        \While{$ \exists u \in (enabled(s) \cap backtrack(s) \backslash visited) $}{
            add $u$ to $visited$ \;
            \uIf{$(Bv(S.next(s, u)) \leq c)$}{
                Explore($S.next(s, u)$) \;
            }
        }
    }
\end{algorithm}


\section{Nidhugg-BPOR}
Having implemented persistent sets correctly the next task is the implementation of a BPOR algorithm. The novelty of the BPOR is the introduction of conservative branches. These
are branches that are introduced in order to guarantee the exploration of the whole state space. It is common for a trace to exceed the bound limit whereas there is
an equivalent trace which does not. The conservative branches are used for this purpose.

\begin{definition}{(Trace block)}
For a trace $T$ a sequence $B$ of consecutive events is a trace block iff all events happen in the same thread i.e. all the events have the same thread id.
\end{definition}

The idea behind conservative branches is quit simple. When a branch is added a conservative branch is added at the beginning of the corresponding block.
Usually concurrent events take place inside a block. As a result when a branch is taken then the preemption count will most probably increased. However had this
branch been added at the beginning of the block the preemption count would not have been increased. 

The algorithm implemented is presented here \cite{BPOR} in detail. A modification of this algorithm is used in order to take advantage of the Nidhugg's infrastructure.
The algorithm is presented in Algorithm \ref{Nidhugg BPOR}.

\begin{algorithm}
    \caption{Nidhugg-BPOR}
    \label{Nidhugg BPOR}
    \SetKwInOut{Input}{input}
    \SetKwHangingKw{Let}{let}
    Explore($\langle \rangle$,$\emptyset$,$b$)\;
    \Fn{Explore($E$,$Sleep$,$b$)}{
        \If{$\exists p \in ((enabled(s_{[E]}) \backslash Sleep)$ and $B_v(E.p) <=b$}{
            backtrack(E) $:={p}$ \;
            \While{$\exists p \in (backtrack(E) \backslash Sleep $ and $B_v(E.p) <=b$}{
                \ForEach{$e \in dom(E)$ such that $e \lesssim_{E.p} next_{[E]}(p)$}{
                    \Let{$E' = pre(E,e)$}
                    \Let{$u = notdep(e,E).p$}
                    \Let{$CI = \{ i \in I_{E'}(u) \mid i \rightarrow p \}$}
                    \If{$CI \cap backtrack(E') = \emptyset$}{
                        \If{$CI \neq \emptyset$}{
                            add some $q' \in CI$ to $backtrack(E') $ \;
                        }
                        \Else{
                             add some $q' \in I_{[E']}(u)$ to $backtrack(E') $ \;}
                        }
                    \Let{$E''= pre\_block(e,E)$}
                    \Let{$u = notdep(e,E).p$}
                    \Let{$CI = \{ i \in I_{E''}(u) \mid i \rightarrow p \}$}
                    \If{$CI \cap backtrack(E') = \emptyset$}{
                        \If{$CI \neq \emptyset$}{
                            add some $q' \in CI$ to $backtrack(E') $ \;
                        }
                        \Else{
                            add some $c(q') \in I_{[E'']}(u)$ to $backtrack(E'') $ \;
                        }
                    } 
                }
                \Let{ $Sleep' := \{q \in Sleep \mid E \models p \diamondsuit q \}$ } 
                $Explore(E.p, Sleep)$ \;
                \If{$p$ is not conservative}{
                    add $p$ to $Sleep$ \;
                }
            }
        }
    }
\end{algorithm}

A critical challenge arises when a DPOR algorithm is used in tandem with sleep sets. This stems from the fact that conservative branches are not added due to a
concurrent event. By observing the sleep set algorithm we notice that if we follow the same strategy as with non-conservative branches many traces will end up being blocked.

Let us take the writer-2readers example as shown in Figure \ref{Usage of non-conservative branches}.

\trace{w2rpersistent.pdf}{Usage of non-conservative branches}

We notice that the last trace is sleep set blocked while it should be examined. The algorithm is unaware that the thread r should be removed from the sleep set since there is no
or will ever be found any conflict with the first command of the thread which is related with a non shared variable. In order to deal with this problem when a conservative
branch is chosen then it should not be added to the sleep set. However there must be a set recording all the branches that where added at this certain point of the trace
so no thread is added twice. The solution is based on the notion of the conservative sets where every thread that was added to the branch is recorded. 

Intuitively the algorithm is the same with the Source-DPOR with the addition of the conservative branches. The solution is based on the notion of the conservative sets where every thread that was added to the branch is recorded.  However many challenges araise which are discussed 
in the implementation section.

\tracelong{w2rbpor.pdf}{Example of BPOR execution}

\section{Source-BPOR}

Having discussed BPOR algorithm the next step is to try combine source sets with the algorithm. The first observation we have to make is that
source sets and thus the algorithm for creating these sets is not suitable for adding conservative branches. A quick explanation is given in the next writer-2readers example
even though the problem will be further discussed later. Let us assume that we have followed the source set algorithm for adding conservative sets. 
The results are shown at Figure \ref{Following source sets for conservative branches}.

\trace{w2rsourceconservative.pdf}{Following source sets for conservative branches}

It is clear that some traces are not explored. Specifically, the trace which start with r has been rejected. The reason is that it shares the same initials with r1 even at the
beginning of that block. As a result the algorithm must create to persistent sets when conservative threads are added. 

Having made the preceding observations the algorithm used for Source-BPOR is shown in Algorithm \ref{SBPOR}.

\begin{algorithm}
    \caption{Source-BPOR}
    \label{SBPOR}
    \SetKwInOut{Input}{input}
    \SetKwHangingKw{Let}{let}
    Explore($\langle \rangle$,$\emptyset$,$b$)\;
    \Fn{Explore($E$,$Sleep$,$b$)}{
        \If{$\exists p \in ((enabled(s_{[E]}) \backslash Sleep)$ and $B_v(E.p) <=b$}{
            backtrack(E) $:={p}$ \;
            \While{$\exists p \in (backtrack(E) \backslash Sleep $ and $B_v(E.p) <=b$}{
                \ForEach{$e \in dom(E)$ such that $e \lesssim_{E.p} next_{[E]}(p)$}{
                    \Let{$E' = pre(E,e)$}
                    \Let{$u = notdep(e,E).p$}
                    \If{$I_{E'}(u) \cap backtrack(E') = \emptyset$}{
                        add some $q' \in I_{[E']}(u) \text{ to } backtrack(E') $ \;
                    }
                    \Let{$E''= pre\_block(e,E)$}
                    \Let{$u = notdep(e,E).p$}
                    \Let{$CI = \{ i \in I_{E''}(u) \mid i \rightarrow p \}$}
                    \If{$CI \cap backtrack(E') = \emptyset$}{
                        \If{$CI \neq \emptyset$}{
                            add some $q' \in CI$ to $backtrack(E') $ \;
                        }
                        \Else{
                            add some $c(q') \in I_{[E'']}(u)$ to $backtrack(E'') $ \;
                        }
                    } 
                }
                \Let{ $Sleep' := \{q \in Sleep \mid E \models p \diamondsuit q \}$ } 
                $Explore(E.p, Sleep)$ \;
                \If{$p$ is not conservative}{
                    add $p$ to $Sleep$ \;
                }
            }
        }
    }
\end{algorithm}

We can notice that the Algorithm \ref{SBPOR} works as Source-DPOR (Alg. \ref{Source}) for non-conservative traces and as BPOR (Alg. \ref{Nidhugg BPOR}) for conservative traces.
