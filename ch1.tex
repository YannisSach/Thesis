\chapter{Introduction}

Moore's Law, named after Intel's co-founder Gordon Moore, states that the number of transistors that can be placed on an integrated circuit doubles
roughly every two years. For decades, chipmakers have succeeded in shrinking chip geometries, allowing Moore's Law to remain on track and consumers to 
get their hands on ever more powerful laptops, tablets, and smartphones. Software developers could just lay back and wait for the Moore's Law to take effect.
However, constraints such as heat, clock speeds have largely stood still, and the incremental increase of the performance of each individual 
processor core impede the further acceleration of software execution. In order for developers to compensate with the demand of efficient software, programming paradigms such
as concurrent programming have become a necessity. However, new challenges arise from concurrent programming since it is harder and more
error-prone than its sequential counterpart. When programming with multiple execution threads many errors may occur due to the fact the many execution threads 
may access and edit the shared memory or require to execute lines of code excluding other threads.

More specifically, the typical problems with concurrency can be outlined as follows:
\begin{itemize}
\item Race condition: A strange interleaving of processes has an unintended effect.
\item Deadlock: Two or more processes stop and wait for each other.
\item Livelock: Two or more processes keep executing without making any progress.
\end{itemize}

These problems are usually Heisenbugs \cite{Musu08} – they can alter their behavior or completely
disappear when one tries to isolate them – since they go hand in hand with the order of
execution of the processes involved.

\section{Testing and Verification of Concurrent Programs}

Testing and verifying the correctness of a concurrent program can be proved a demanding task. 
A technique used for the systematic exploration of a program's state space is model checking \cite{WikipediaModelChecking}.
Model checking is a method for formally verifying concurrent systems through specifications about the system expressed as 
temporal logic formulas and efficient algorithms that can traverse the model defined by the system and check whether
the specifications hold. The major problem model checking tools have to face is the combinatorial explosion of the state space since
a vast number of global states have to be captured and stored. Many techniques have been proposed in order to tackle this problem.
Stateless model checking, for example, avoids storing global states. This technique has been implemented in tools such as Verisoft \cite{SMC,Gode05}, 
CHESS \cite{Musu08}, Concuerror \cite{Chri13}, Nidhugg \cite{AbdullaAronisAtigJonssonLeonardssonSagonasSMC2015} and \cite{RCMC}. The observation that two 
interleavings are equivalent if one can be obtained from the other by swapping adjacent, independent execution steps is the core of the partial
order reduction \cite{Valmari1991, Peled1993, Godefroid1996,POR,JACM} technique used by many of these tools. Dynamic Partial Order Reduction (DPOR) techniques
capture dependencies between operations of concurrent threads while the program is running \cite{FlanaganDPOR}. The exploration begins with an arbitrary interleaving whose steps are then
used to identify dependent operations and points where alternative interleavings need to
be explored in order to capture all program behaviours. Another approach is the bounded model checking \cite{BoundedModelChecking} where the finite state
machine is unrolled for a fixed number of steps and the specifications are checked within these steps. Bounded model checking can be combined with the partial
order reduction for modeling executions \cite{PORinBMC} and was effectively implemented in tools such as CBMC \cite{CBMC}.
Unfortunately all these techniques still have to deal with the problem of the state space explosion. In order to deal with this problem further
bounding of the exploration is required. Many different bounding techniques have been examined \cite{Thomson} such as preemption bounding, delay
bounding, a controlled random scheduler, and probabilistic concurrency testing (PCT).

\section{Aim of this Thesis}

The purpose of this thesis is the implementation of a preemption bounding technique \cite{BPOR} for Nidhugg and the combination
of this technique with the a novel technique \cite{AbdullaAronisJohnssonSagonasDPOR2014} suggested for better coverage of the state space.
The bounded-DPOR was used to verify the linux kernel \cite{LinuxKernel} and specifically RCU \cite{Spin}. RCU is a synchronization
mechanism used heavily in Linux kernel, and many of the kernel’s subsystems rely on RCU’s correct operation. By using BPOR the minimum preemptive
switches required to track failure injections were counted. As a result the empirical observation that errors occur in a small bound count was confirmed.
Moreover, the possible application of various optimizations used for unbounded DPOR on bounded DPOR are examined. 

\section{Overview}
In Chapter \ref{Chapter 2} the theoretical background utilized in Nidhugg is given for both unbounded and bounded DPOR is given.
In Chapter \ref{Chapter 3} the implementation of the methods for Nidhugg is discussed. In Chapter \ref{Chapter 4} the different techniques
implemented are evaluated using both synthetic tests and RCU. In Chapter \ref{Chapter 5} further discussion on possible optimatizations
is made. In Chapter \ref{Chapter 6} we summarize the previous chapters and the conclusions
we drew from this thesis, and present some possible extensions to our work.
