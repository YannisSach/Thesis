\chapter{Background}
\label{Chapter 2}

\section{Concurrent Programming}

Concurrent computing, which is implemented by concurrent programming paradigm, is a form of computing in which several 
computations are executed during overlapping time 
periods—concurrently—instead of sequentially (one completing before the next starts). 
This is a property of a system—this may be an individual program, a computer, or a network—and there is a separate execution point 
or ``thread of control'' for each computation (``process''). A concurrent system is one where a computation can advance without waiting for 
all other computations to complete.
The main challenge in designing concurrent programs is concurrency control: ensuring the correct sequencing of the 
interactions or communications between different computational executions, and coordinating access to resources that are shared among executions.
Potential problems include race conditions, deadlocks, livelocks and resource starvation. 
The scheduler is usually responsible for running a thread. Due to this scheduling non-determinism the programmer cannot always be aware of which thread
will be scheduled next.

An important aspect of a concurrent program is the notion of the set of interleavings which is the set of all the execution paths a program can follow.
Intuitively, if we imagine a process as a (possibly infinite) sequence/trace of statements (e.g. obtained by loop unfolding),
then the set of possible interleavings of several processes consists of all possible sequences of statements of any of these processes.

As it can be inferred, debugging this kind of programs can be proved extremely challenging. The challenge mainly emerges from the fact that it is 
not always clear which thread command will be executed. Moreover the error may not always occur during debugging since there may be only a limited
number of interleavings that produce an error. 

\section{Concurrency Errors}
At this point, it is important to introduce the notion of concurrency error and how explain how it differs from any other error.

\begin{definition}{(Concurrency Error)}
    Concurrency error is an error caused by the scheduler's non-determinism.
\end{definition}

An example of a program containing a concurrency error is shown in the Listing \ref{Example of concurrency error}. In this program variable $x$ equals 1 in the beginning of the program. However,
if thread $zero$ is scheduled before $divider$ a division by zero will take place. On the other hand, a division by $x$ cannot be considered a concurrency error in other cases
such as the one shown in Listing \ref{Example of non-concurrency error} where a division by zero occurs without the scheduler's intervention.

\Code{./code/zero.c}{Example of non-concurrency error}
\Code{./code/zeroconc.c}{Example of concurrency error}

\section{Testing, Model Checking, and Verification}

Dynamic software model checking is a form of systematic testing which is applicable to industrial-size software and consists of adapting model checking.
Many tools have been developed over the last decades that use this paradigm which aim to test concurrent and data-driven software. Model checking is more computationally
expensive than traditional software testing since it provides better coverage. However, it is cheaper than more general forms of verification such as interactive theorem
proving, which provides more extended verification guarantees, since it can be automated in a higher level.
Hence, dynamic software model checking offers an attractive practical trade-off between testing and formal verification. 

The graph shown in Figure \ref{Comparing Testing, Model Checking and Verification} \cite{TestingvsVerification} greatly demonstrates the differences between testing, model checking and verification.

\trace{testmodver.png}{Comparing Testing, Model Checking and Verification}

\section{Stateless Model Checking and Partial Order Reduction}

In order to find an error of a concurrent program, one must examine every possible interleaving, (i.e., all the
different ways that a program can be executed) this program can produce. Usually the error would occur only under some
interleaving the programmer did not take into consideration, making its detection extremely difficult. Stateless model
checking is based on the idea of driving the program along all these possible interleavings. However, this approach
suffers from state explosion, i.e., the number of all possible interleavings grows exponentially with the size of the
program and the number of threads. Several approaches to this problem have been proposed in order to deal with this
challenge such as partial order reduction \cite{Godefroid1996} and bounding techniques \cite{BPOR}. 

Partial order reduction aims to reduce the number of interleavings explored by eliminating the 
exploration of equivalent interleavings. Each interleaving can be represented by a trace.
These equivalent traces are produced by the inversion of independent events which do not affect the results of the program. For example, the scheduling of two threads that read a local
variable can be inverted since the result of the operation is not affected by the order under which these operations occurs. 
There are two ways that a partial order reduction algorithm can be implemented. The first is a
static partial order reduction algorithm \cite{Static1997} where the dependencies between two threads are tracked before the execution of the concurrent program. 
The second is the Dynamic partial order reduction (DPOR) \cite{FlanaganDPOR} which observes the program's dependencies of runtime. It is important to notice that the size of
the state space still grows exponentially. 

For larger programs DPOR often runs longer than developers are willing to wait. In these cases, bounding techniques can be proved useful. Bounded techniques,
in contrast to DPOR, alleviates state-space explosion by pruning the executions that exceed a bound \cite{Thomson}. There have been proposed many bounded techniques
such as preemption bounded exploration \cite{BPOR} or delay bounded exploration \cite{Delay11}. All bounded search techniques are based on the notion that many of the concurrency bugs can be
tracked even when the bound limit is set to be small, thus the time required for a bug to be found is significantly smaller.

When it comes to bounded search, new challenges arise \cite{BPOR} since more dependencies, related to the bound limit, are introduced. 
Many redundant traces must be explored in order to compensate with the bound since a bounded search algorithm cannot be aware if an equivalent trace with greater bound count
has already been explored.

It is important to notice that some bounded techniques can be regarded as testing, in the sense that explore only a subset of total state space, while
others can be considered as verification, in the sense that can verify that no error can occur in the given bound.

\section{Vector Clocks}

A vector clock is an algorithm for generating a partial ordering of events in a distributed or concurrent system and detecting causality violations. 
Just as in Lamport timestamps \cite{Lamp}, interprocess messages contain the state of the sending process's logical clock. 
A vector clock of a system of $N$ processes is an array/vector of $N$ logical clocks, one clock per process; 
a local ``smallest possible values'' copy of the global clock-array is kept in each process, with the following rules for clock updates:

\begin{enumerate}
    \item Each process experiencing an internal event, it increments its own logical clock in the vector by one.
    \item Each time a process receives a message or performs an action on a shared variable, it increments its own logical clock in the vector by one and updates each element in its vector 
    by taking the maximum of the value in its own vector clock and the value in the vector in the received message or the maximum value of all processes that share
    the same shared variable.
\end{enumerate}

An example execution of the algorithm is shown in Figure \ref{Clock example} where both the source code and an explored trace are given.
As we can easily notice for each command of thread <0> (main thread) the clock of the main thread increases. The thread <0.0> starts to run its clock for the thread <0> is 8 since 
that is the moment when the thread was spawned. When the value of y is read the clock for <0> increases again so it corresponds with the y=1 event. When the first thread is
scheduled again then its clock for the <0.0> is 7 since \verb|pthread_join()| command takes place.


\Side{./code/clocks.c}{Vector Clock example}{./code/clocks.out}{Vector Clock output}{Clock example}

Every algorithm that is presented in this thesis is based on vector clocks.


\section{Notation}

Before delving in the problem of dynamic partial order reduction it is crucial to explain the notation that will be used.
An execution sequence $E$ of a system is a finite sequence of
execution steps of its processes that is performed from the initial
state which we denote as $s_0$. Since each execution step is deterministic, an execution
sequence $E$ is uniquely characterized by the sequence of processes
that perform steps in $E$. For instance, $p.p.q$ denotes the execution
sequence where first $p$ performs two steps, followed by a step of $q$.
The sequence of processes that perform steps in $E$ also uniquely
determine the (global) state of the system after $E$, which is denoted
$s_{[E]}$. For a state $s$, let $enabled(s)$ denote the set of processes $p$ that
are enabled in $s$ (i.e., for which execute $p(s)$ is defined). We use $.$ to
denote concatenation of sequences of processes. Thus, if $p$ is not
blocked after $E$, then $E.p$ is an execution sequence.
An event of $E$ is a particular occurrence of a process in $E$.
We use $\langle p,i \rangle$ to denote the ith event of process $p$ in the execution
sequence $E$. In other words, the event $\langle p,i \rangle$ is the ith execution step
of process $p$ in the execution sequence $E$. We use $dom(E)$ to denote
the set of events $\langle p,i \rangle$ which are in $E$, i.e., $\langle p,i \rangle \in dom(E)$ iff $E$
contains at least $i$ steps of $p$. We will use $e,e',...$ , to range over
events. We use $proc(e)$ to denote the process $p$ of an event $e = \langle p, i \rangle$.
If $E.w$ is an execution sequence, obtained by concatenating $E$ and
$w$, then $dom_{[E]}(w)$ denotes $dom(E.w) \ dom(E)$, i.e. the events in
$E.w$ which are in $w$. As a special case, we use $next_{[E]}(p)$ to denote
$dom_{[E]}(p)$.
We use $<_E$ to denote the total order between events in $E$, i.e.
$e <_E e'$  denotes that $e$ occurs before $e'$  in $E$. We use $E'\leq E$ to
denote that the sequence $E'$ is a prefix of the sequence $E$.

\section{Event Dependencies}

One of the most important concepts when we have to deal with an algorithm that searches the whole state space of the different schedulings is the 
happens-before relation in an execution sequence. Usually this relation is denoted with $\rightarrow$ symbol. For example, if the relation $\rightarrow$ 
for two events $e,e'$ in $dom(E)$ holds true then the event $e$ happens-before $e'$. This relation usually appears in the message exchange, when $e$ is the message
transmission and $e'$ is the event when the message is received. For the context of Nidhugg $e \rightarrow e'$ would hold true when at least one of the two events
is a write operation on the same shared variable. It is fathomable that any DPOR algorithm should be able to assign this happens-before relations. 
In practice, the happens-before assignment is implemented with the use of vector clocks.

\begin{definition}{(happens-before assignment)}
    A happens-before assignment, which assigns a
    unique happens-before relation $\rightarrow E$ to any execution sequence
    $E$, is valid if it satisfies the following properties for all execution
    sequences $E$.
    \begin{enumerate}
        \item $\rightarrow_{E}$ is a partial order on $dom(E)$, which is included in $<_E$. In other words every scheduling is part of the set of all possible
        partial order of the program.
        \item The execution steps of each process are totally ordered, i.e. 
        $\langle p,i \rangle \rightarrow_E \langle p,i+1 \rangle$ whenever $\langle p, i+1 \rangle \in dom(E)$.
        \item If $E'$ is a prefix of $E$ then $\rightarrow_E$ and $\rightarrow_{E'}$ are the same on $dom(E')$.
        \item Any linearization $E'$ of $\rightarrow_E$ on $dom(E)$ is an execution sequence which has exactly the same “happens-before” relation
$\rightarrow_{E'}$ as $\rightarrow_E$. This means that the relation $\rightarrow_E$ induces a set
of equivalent execution sequences, all with the same “happens-before” relation. 
We use $E \simeq E'$ to denote that $E$ and $E'$ are
linearizations of the same “happens-before” relation, and $[E] \simeq$ 
to denote the equivalence class of E.
    \item If $E \simeq E'$ then $s_{[E]} = s_{[E']}$ (i.e. two equivalent traces will lead to the same state).
    \item For any sequences $E, E'$ and $w$, such that $E.w$ is an execution
sequence, we have $E \simeq E'$  if and only if $E.w \simeq' E'.w$.
    \end{enumerate}
\end{definition}

The first six properties should be obvious for any reasonable
happens-before relation. The only non-obvious is the
last. Intuitively, if the next step of p happens before the next step
of $r$ after the sequence $E$, then the step of $p$ still happens before
the step of $r$ even when some step of another process, which is not
dependent with $p$, is inserted between $p$ and $r$. This property holds
in any reasonable computation model that we can think of. As
examples, one situation is when $p$ and $q$ read a shared variable that
is written by $r$. Another situation is that $p$ sends a message that is
received by $r$. If an intervening process $q$ is independent with $p$, it
cannot affect this message, and so $r$ still receives the same message.
Properties 4 and 5 together imply, as a special case, that if $e$
and $e'$ are two consecutive events in E with $e \not \rightarrow_{E} e'$, then they can
be swapped and the (global) state after the two events remains the
same.

\section{Independence and Races}

We now define independence between events of a computation. If
$E.p$ and $E.w$ are both execution sequences, then $E \models p\diamondsuit w$ denotes
that $E.p.w$ is an execution sequence such that $next_{[E]}(p) \not \rightarrow_{E.p.w} e$
for any $e \in dom([E.p])(w)$. In other words, $E \models p \diamondsuit w$ states that
the next event of $p$ would not “happen before” any event in $w$
in the execution sequence $E.p.w$. Intuitively, it means that $p$ is
independent with $w$ after $E$. In the special case when $w$ contains
only one process $q$, then $E \models p \diamondsuit q$ denotes that the next steps of
$p$ and $q$ are independent after $E$. We use $E'\models p \diamondsuit w$ to denote that
$E \not \models p \diamondsuit w$ does not hold.

For a sequence $w$ and $p \in w$, let $w \backslash p$ denote the sequence
$w$ with its first occurrence of $p$ removed, and let $w \uparrow p$ denote the
prefix of w up to but not including the first occurrence of $p$. For
an execution sequence $E$ and an event $e \in  dom(E)$, let $pre(E,e)$
denote the prefix of $E$ up to, but not including, the event $e$. For an
execution sequence $E$ and an event $e \in E$, let $notdep(e, E)$ be the
sub-sequence of $E$ consisting of the events that occur after $e$ but do
not “happen after” $e$ (i.e., the events $e'$ that occur after $e$ such that
$e \not \rightarrow_E e'$).


A central concept in most DPOR algorithms is that of a race.
Intuitively, two events, $e$ and $e'$ in an execution sequence $E$, where
$e$ occurs before $e'$ in $E$, are in a race if
\begin{itemize}
\item $e$ happens-before $e'$ in $E$, and
\item $e$ and $e'$ are “concurrent”, i.e., there is an equivalent execution
sequence $E' \simeq E$ in which $e$ and $e'$ are adjacent.
\end{itemize}
Formally, let $e \lessdot_E e'$ denote that $proc(e) \not = proc(e')$, that $e \rightarrow_E e'$,
and that there is no event $e'' \in dom(E)$, different from $e'$ and $e$,
such that $e \rightarrow_E e'' \rightarrow_E e'$.

Whenever a DPOR algorithm detects a race, then it will check
whether the events in the race can be executed in the reverse order.
Since the events are related by the happens-before relation, this may
lead to a different global state: therefore the algorithm must try to
explore a corresponding execution sequence. Let $e \lesssim_E e'$ denote
that $e \lessdot_E e'$, and that the race can be reversed. Formally, if $E' \lesssim E$
and $e$ occurs immediately before $e'$ in $E'$, then $proc(e')$ was not
blocked before the occurrence of $e$.


\section{Dynamic Partial Order Reduction}

Before explaining the DPOR algorithm it is important to define sufficient sets.

\begin{definition}{(Sufficient Sets)}
A set of transitions is sufficient in a state $s$ if any relevant
state reachable via an enabled transition from $s$ is also reachable from $s$ via at least one of the transitions in the sufficient
set. A search can thus explore only the transitions in the
sufficient set from $s$ because all relevant states still remain
reachable. The set containing all enabled threads is trivially
sufficient in $s$, but smaller sufficient sets enable more state
space reduction.
\end{definition}

Many techniques have been proposed in order to implement a DPOR algorithm. What most of these techniques share in common is the structure shown in Algorithm.
\ref{GeneralDPOR}.
\SetKwProg{Fn}{Function}{}{}

\SetKwHangingKw{Let}{let}
\begin{algorithm}[H]
    \caption{General form of DPOR}
    \label{GeneralDPOR}
    Explore($\emptyset$)\;
    \Fn{Explore($E$)}{
     \Let{$T = Sufficient\_set(final(E)$)}
     \For{all $t \in T$}{
        Explore($E.t$) \;
    }
    }
\end{algorithm}

where $final(E)$ represents the state that will be reached when the execution sequence $E$ is executed.

The algorithm above describes a DFS search in the state space of all possible interleavings.
As it can be inferred from the algorithm the most important step is that of the calculation of the set $T$.

\begin{definition}{(Enabled sets, $enabled(s)$)}
    Given a state $s$, $enabled(s)$ represents the set of all the threads that can be scheduled immediately after $s$.
\end{definition}

An obvious property the sufficient sets must hold is that Sufficient\_set$(final(E)) \subseteq enabled(E)$.

Intuitively $enabled(s)$ represents the threads that are not blocked or have already finished their execution.

In bibliography many types of sufficient sets can be found \cite{Godefroid1996}. 
In this thesis we mainly focus on persistent sets and on source sets.


\section{Persistent Sets}

A persistent set in a state $s$ is a sufficient set of transitions to
explore from $s$ while maintaining local state reachability for acyclic state spaces \cite{God97}. A selective search using persistent
sets explores a persistent set of transitions from each state $s$ where $enabled(s) \neq \emptyset$ and prunes enabled transitions that
are not persistent in s.
In a more formal way:\\

\begin{definition}{(Persistent Sets)}
Let $s$ be a state, and let $W \subseteq E(s)$ be a set
of execution sequences from $s$. A set $T$ of transitions is a persistent set for $W$
after $s$ if for each prefix $w$ of some sequence in $W$, which contains no occurrence
of a transition in $T$,  we have $E \vdash t \diamondsuit w$ for each $t \in T$.
\end{definition}

The above definition can be described as follows: If $t \in T$ and there is another thread $t'$ that can be executed until a command which
is in a race with $t$, then $t'$ belongs in the persistent set.

Notice that the definition of persistent sets suggests a way to construct them.

In Figure \ref{Construction of persistent sets} two different examples of persistent set construction are given. We denote the persistent set of branches the execution will take with $BR{}$.
In the first, let a concurrent program contain 3 threads $p$, $q$, and $r$. Thread $p$ changes the value of the variable (writer) and the other ($q$ and $r$) just read this variable (readers).
Let $p.q.q.r.r$ be an interleaving. According to the definition of the persistent sets, $q$ and $r$ are in a race with $p$, thus, $q$ and $r$ must also be on the persistent set
of the first command of the interleaving. In Figure \ref{Construction of persistent sets} we notice that both $r$ and $q$ threads are added to the persistent set of the first
command of the trace since both conflict with the write operation. 
In the second example, let $p$ and $r$  be a readers and $q$ be a writer. We notice that both $r$ and $q$ are added. However, there is no conflict between $p$ and $r$ since both $p$ and $r$
just read the variable $x$. The reason why the thread $r$ is added is the conflict that will be produced by the $q$'s write operation.

\trace{persistent.pdf}{Construction of persistent sets}

\section{Source Sets}

Before defining source sets, we have to give some more useful definitions.

\begin{definition}{($dom(E)$)}
    The set of events-transitions happening during the scheduling of $E$.
\end{definition}

\begin{definition}{(Initials after an execution sequence $E.w$, $I_{[E]}(w)$)}
For an execution sequence $E.w$, let $I_{[E]}(w)$ denote the set of
processes that perform events $e$ in $dom_{[E]}(w)$ that have no
“happens-before” predecessors in $dom_{[E]}(w)$. More formally,
$p \in I_{[E]}(w)$ if $p \in w$ and there is no other event $e \in dom_{[E]}(w)$ with
$e \rightarrow_{E.w} next_{[E]}(p)$.
\end{definition}

By relaxing the definition of Initials we can get the definition of Weak Initials, $WI$.

\begin{definition}{(Weak Initials after an execution sequence $E.w$, $WI_{[E]}(w)$)}
For an execution sequence $E.w$, let $WI_{[E]}(w)$ denote the union of $I_{[E]}(w)$ and the set of
processes that perform events $p$ such that $p \in enabled(s_{[E]}) $.
\end{definition}

\begin{definition}{(Source Sets)}
Let $E$ be an execution sequence,
and let $W$ be a set of sequences, such that $E.w$ is an execution
sequence for each $w \in W$. A set $T$ of processes is a source set for
$W$ after $E$ if for each $w \in W$ we have $WI_{[E]}(w) \cap P  = \emptyset$.
\end{definition}

A source set is a set of threads that guarantee that the whole state space will be explored. 
Notice that their is no requirement related to the races of the events.
What the above definition implies is that source can be considered every set of threads that contains these threads 
that are able to cover the whole state-space.
In fact, it suggests a property for the sufficient sets to hold.

\section{Sleep Sets}

Another technique complementary to the persistent or source sets aiming to reduce the number of interleavings is the sleep set technique.
Sleep sets prohibit visited transitions from executing again
until the search explores a dependent transition. Assume that
the search explores transition $t$ from state $s$, backtracks $t$,
then explores $t_0$ from $s$ instead. Unless the search explores
a transition that is dependent with $t$, no states are reachable
via $t_0$ that were not already reachable via $t$ from s. Thus, $t$
“sleeps” unless a dependent transition is explored.

A short example on sleep sets is the following:
Let the concurrent program of one writer (w1) and two readers (r1,r2).
let w1 <0.0>: w(x) r1 <0.1>: (local operations), r(x) and r2 <0.2>: (local operations), r(x).

The resulted traces are demonstrated in the Listing \ref{Sleep set example}. Notice that whenever a process executes a command 
which is in race with some command of another process and the latter process is ``sleeping'', it wakes up. 
As we can infer from the execution of the DPOR algorithm the interleaving which started from r2 was blocked since it would lead to an interleaving which
has already been explored. Notice that this is due to the fact that r1 cannot ``wake up'' since its first transition (local operations) does not conflict with any other transition
in the program. 

It can be proved \cite{Godefroid1996} that sleeps will eventually block all the redundant interleavings and thus the only interleavings that will be explored till their end (where all threads that could be executed, have been executed).
As a result an optimal algorithm should be able to not consider these interleavings whatsoever.

\Output{./code/sleep_sets.out}{Sleep set example}

\section{Comparing Persistent Sets with Source Sets}

Note that the definition of source sets is much more relaxed than the definition of the persistent sets. 
This relaxation enables the source sets to be much more efficient than the persistent sets. In Figure \ref{Non-minimal persistent sets}
an example is given were source sets and persistent sets differ.

\begin{figure*}
    \begin{lstlisting}[frame=none,numbers=none]
        Initially: x = y = z = 0 
    \end{lstlisting}
    \begin{minipage}{0.3\textwidth}
      \begin{lstlisting}[frame=none, numbers=none]
        p:
        m := x; (p1)
        if (m = 0) then
            z := 1; (p2)
      \end{lstlisting}
    \end{minipage}
    \begin{minipage}{0.3\textwidth}
        \begin{lstlisting}[frame=none, numbers=none]
            q:
            n := y; (q1)
            if (n = 0) then
                x := 1; (q2)
        \end{lstlisting}
      \end{minipage}
      \begin{minipage}{0.3\textwidth}
        \begin{lstlisting}[frame=none, numbers=none]
            r:
            o := z; (r1)
            if (o = 0) then
                y := 1; (r2)
        \end{lstlisting}
      \end{minipage}
      \caption{Program with non-minimal persistent sets}
      \label{Non-minimal persistent sets}
  \end{figure*}

From the example, it is clear that the reason why source sets are an improvement over persistent sets is the fact that minimum source sets can eliminate
sleep set blocked traces i.e., traces that would eventually be blocked by the sleep sets. An algorithm that would only calculate minimal source sets would be optimal \cite{AbdullaAronisJohnssonSagonasDPOR2014}, hence
would never explore two equivalent interleavings.

It is obvious that a single transition cannot be a source set. For
instance, the set $\{ p_1 \}$ does not contain the initials of execution $q_1.q_2.p_1.r_1.r_2$,
since q2 and p1 perform conflicting accesses. On the other hand, any subset
containing two enabled transitions is a source set. To see this, let us choose
$\{p_1, q_1 \}$ as the source set. Obviously, $\{p1, q1 \}$ contains an initial of any execution
that starts with either $p_1$ or $q_1$. Any execution sequence which starts with $r_1$ is
equivalent to an execution obtained by moving the first step of either $p_1$ or $q_1$ to
the beginning:
\begin{itemize}
\item If $q_1$ occurs before $r_2$, then $q_1$ is an initial, since it does not conflict with
any other transition.
\item If $q_1$ occurs after $r_2$, then $p_1$ is independent of all steps, so $p_1$ is an initial.
We claim that $\{p_1, q_1 \}$ cannot be a persistent set. The reason is that the execution
sequence $\{r_1.r_2 \}$ does not contain any transition in the persistent set, but its second
step is dependent with $q_1$. By symmetry, it follows that no other two-transition
set can be a persistent set.
\end{itemize}

In other words, persistent sets have the unpleasant property that adding a process
may disturb the persistent set so that even more process may have to be added.
This property is relevant in the context of DPOR, where the first member of the
persistent set is often chosen rather arbitrarily (it is the next process in the first
exploration after $E$), and where the persistent set is expanded by need.

Continuing the comparison between source sets and persistent sets, we first
note some rather direct properties, including the following.

\begin{itemize}
\item Any persistent set is a source set.
\item Any one-process source set is a persistent set.
\end{itemize}


\section{Bounded Search and Preemption Bounding}

Bounded search explores only executions that do not exceed
a bound \cite{BPOR,Thomson}. The bound may be any property of a
sequence of transitions. A bound evaluation function $B_v(E)$
computes the bounded value for a sequence of transitions $E$.
A bound evaluation function $B_v$ and bound $c$ are inputs to
bounded search. Bounded search may not visit all relevant
reachable states; it visits only those that are reachable within
the bound. If a search explores all relevant states reachable
within the bound, then it provides bounded coverage.

An algorithm that could describe a bounded search would be the following:

\begin{algorithm}[H]
    \caption{Bounded-DPOR}
    \KwResult{Explore the whole statespace}
    Explore($\emptyset$)\;
    \Fn{Explore($E$)}{
        T = Sufficient\_set($final(E)$)
     \For{all $t \in T$}{
         \If{$B_v(E.t) \leq c$}{
            Explore($E.t$)
         }
        }
    }
\end{algorithm}

\noindent The only difference between the unbounded and the bounded version of the algorithm is the if-statement on line 4 which allows for an interleaving to be explored
only if the bound has not been exceeded.

What is needed next is an appropriate definition of the function $B_v$ that calculates a value that the bounded-DPOR tries to keep bounded, 
and the sufficient set. 

In this thesis, we mainly focus on preemption-bounded search. 

Preemption-bounded search limits the number of preemptive context switches that occur in an execution \cite{Musu07}. The 
preemption bound is defined recursively as follows.

\begin{definition}{(Preemption bound)}
\\
$P_b(\emptyset) = 0$ \\
$P_b(E.t) = 
 \begin{cases} 
    P_b(E) + 1 & \text{ if } t.tid = last(E).tid \text{ and } last(E).tid \in enabled(final(E)) \\
    P_b(E) & \text{ otherwise }
 \end{cases}
$\\
\end{definition}

The previous definition describes what a preemptive context switch is. A preemptive context switch happens when the previously running thread could execute
its next step but it does not due to the scheduling of another thread. Hence, a preemptive switch will increase the preemption bound.

\section{Preemption Bounded Persistent Sets}

A set that has been proposed as a sufficient for preemption bounded search is the preemption bounded persistent set \cite{BPOR}.

An important observation is that the execution of a thread until it gets blocked or terminates will not increase
the bound count.
\begin{definition}{($ext(s,t)$)}
    Given a state $s = final(E)$ and a transition $t \in enabled(s)$,
    $ext(s,t)$ returns the unique sequence of transitions $\beta$ from $s$
    such that
    \begin{enumerate}
        \item $\forall i \in dom(\beta): \beta_i.tid = t.tid$
        \item $t.tid \notin enabled(final(E.\beta))$
    \end{enumerate}
\end{definition}

Next, we need to define preemption bounded persistent sets. We denote with $A_G(P_b,c)$ the generic 
bounded state space with bound function $P_b$ and bound $c$. $last(a)$ denotes the last execution step of
an execution sequence $a$

\begin{definition}{(Preemption bounded persistent set)}

A set $T \subseteq \mathcal{T}$ of transitions enabled in a state $s=final(e)$
is preemption-bound persistent in $s$ iff for all nonempty
sequences $a$ of transitions from $s$ in $A_G(P_b,c)$ such that
$\forall i \in dom(a), a_i \notin T$ for all $t \in T$ ,

\begin{enumerate}
\item $Pb(E.t) \leq Pb(E.a_1)$
\item if $Pb(E.t)<Pb(E.a_{1}) ,$ then $t \leftrightarrow last(a)$ and $t \leftrightarrow  next(final(E.a), last(a).tid)$
\item if $Pb(E.t)=Pb(E.a_{1}),$ then $ext(s,t) \leftrightarrow last(a)$ and $ext(s,t) \leftrightarrow next(final(E.a), last(a).tid)$
\end{enumerate}

\end{definition}

When dealing with preemption bounded DPOR it is useful to introduce the idea of blocks in an execution sequence.

\begin{definition}{(Block of execution sequence)}
    Block in an execution sequence is the maximal subsequence of execution steps that consists of execution steps of the same thread.
\end{definition}

In the Figure \ref{Example of blocks} there are three blocks. The first block is coloured with yellow, the second with green and the third with blue.

\trace{blocks.pdf}{Example of blocks}

Let $P$ be persistent set. A preemption bounded persistent set is a set that contains all $p \in P$ with the addition of all the 
threads that would be added in a block that would be created when $p$ was scheduled. These threads are called conservative threads and their 
goal is to allow the coverage of interleavings that would not exceed the bound. Notice that an interleaving can be both conservative and non-conservative.
Preemption bounded persistent sets extend a persistent set by adding all the threads that will create a new block
after the block that will be created by the persistent set.

