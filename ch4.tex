\chapter{Evaluation of Bounding Techniques}
\label{Chapter 4}

In this chapter, the performance of each implemented technique will be discussed. Firstly, the performance of classic-DPOR is demonstrated in order to prove
its performance, indeed differs from Source-DPOR. The evaluation happens in two parts. In the first part, short synthetic programs are used, while in the second part
real world software is tested. Synthetic programs can be found in the appendix section. One area where Nidhugg is tested is the verification of the Read Copy Update
technique of the Linux kernel.


\section{Synthetic Tests}
There are many tests provided from various sources. Most of these testcases are not complicated at all since their purpose is to demonstrate the performance
difference of the Source-DPOR and classic-DPOR.

\begin{itemize}
\item The writer-Nreaders test: In this test N threads read (readers) the same global variable and one thread (writer) writes that variable. It is important 
to notice that in this case there are some other local operations taking place before the read of the variable. As a result we must expect different results between 
source sets and persistent sets.

\item Account: This test is a small bank account simulation which uses mutex locks to prevent simultaneous operations on the account.
There are three possible operations: The deposit operation increases the balance by an amount. The withdraw operation decreases the balance by a certain amount. The check\_result operation confirms 
$\text{final\_balance} == \text{initial\_balance} + \text{deposit} - \text{withdraw}$ and can only happen after both deposit and withdraw are completed.

\item Micro: In this test three threads are spawned that perform the \verb|x++| operation twice. The \verb|x++| operation
consists of two operations a read operation and a write operation.

\item Last-zero test:The program consists of N+1 threads which operate on an array of
N+1 elements which are all initially zero. In this program, thread 0
searches the array for the zero element with the highest index, while
the other N threads read one of the array elements and update the
next one. The final state of the program is uniquely defined by the
values of i and array[1..N]. Last-zero does not produce more traces when DPOR is used for reasons
that will be explained later. However a modification of the .ll file can expose the difference.

\item Indexer.c: This benchmark uses a compare-and-swap(CAS) primitive instruction to check
whether a specific entry in a matrix is 0 and set it to a new
value. 

\item Indexermod.c: In this benchmark all the threads traverse and try to write the matrix at the same
order and as a result many conflicts emerge.


\end{itemize}

\section{RCU}

Read-Copy-Update (RCU) is a synchronization mechanism, which was invented by McKenney and Slingwine \cite{McKenney98}, based on mutual exclusion.
It was added to the Linux kernel in October of 2002. RCU achieves scalability improvements by allowing reads to occur concurrently 
with updates. In contrast with conventional locking primitives that ensure mutual exclusion among concurrent threads regardless of whether 
they be readers or updaters, or with reader-writer locks that allow concurrent reads but not in the presence of updates, 
RCU supports concurrency between a single updater and multiple readers. 

DPOR was used as an approach to systematically test the code
of the main flavor of RCU used in the Linux kernel (Tree RCU) for
concurrency errors, under sequential consistency. 
The modeling allows Nidhugg, a stateless model checking
tool, to reproduce, within seconds, safety and liveness bugs that
have been reported for RCU \cite{Spin}.

RCU provides an ideal testcase to evaluate the various DPOR and Bounded DPOR algorithms since it is:
\begin{itemize}
\item It is a real world software and not just a synthetic test.
\item The number of traces (different schedulings) is large enough to demonstrate the differences in the performance.
\item Previous work \cite{Spin} enables us to evaluate the correctness of each algorithm's implementation.
\end{itemize}


\section{Evaluation of Persistent Sets}
As it was established in the previous chapter the implementation of the persistent sets
is crucial since they are utilized in every bounding technique. 
In this section we demonstrate performance differences between Source-DPOR and classic-DPOR in both synthetic tests and RCU.

\subsection{Evaluation of Persistent sets on Synthetic tests}
The execution of the synthetic test cases delineated that Source-DPOR is indeed an improvement over Classic-DPOR. As it was expected source-DPOR explores less 
traces than the DPOR. It is important
to notice that this difference is caused by the sleep set blocked traces that are produced by the DPOR algorithm that are omitted by the source DPOR. 
The reduction in the number of traces explored is not the same of all the testcases. For example in some testcase there is a
significant decrease of the explored traces while in others the reduction is not so great. 
It varies due to the different approaches as well as the size of the state space. The results are presented with two different ways. 
The writer-N-readers testcase result is given with a graph, in Figure \ref{writer-N-readers}, in order to demonstrate the escalation of the state space as well as the greater impact the source-DPOR has. The rest of the
results are given in Table \ref{Source-DPOR vs DPOR for synthetic tests} so they can be easily compared.

\graph{/home/yannis/nidhugg/tests/mytests/wNr.png}{writer-N-readers}

\smalltabular{"/home/yannis/nidhugg/tests/mytests/unbounded.tex"}{Source-DPOR vs DPOR for synthetic tests}

\subsection{Evaluation of Persistent sets on RCU}
We noticed that there is no difference between Source sets and persistent sets thus no results are presented since they coincide with \cite{Spin}. 
The reason why the results of DPOR and Source-DPOR are the same may be due to the operations that take place which not allow for the optimization of the Source-DPOR 
to be effective. Another reason is the LLVM IR which is used by Nidhugg and will be explained in the next section.

\section{Comparison with Concuerror results - Why DPOR may be enough for LLVM}
The are many cases where Concuerror's Source-DPOR explores less traces than Classic-DPOR while Nidhugg's Source-DPOR doesn't. 
In fact Nidhugg's Classic-DPOR implementation seems to explore less number of traces than Concuerror's Classic-DPOR \cite{AbdullaAronisJohnssonSagonasDPOR2014}.
which equals with the numbers of traces explored by Source-DPOR. However, a look on the code that Nidhugg tests justifies this behavior.
LLVM produces much more code than the code in the source file. Due to this extra code that is added, the number of conflicting events is less than the 
number of conflicting events in an Erlang program.

At Figure \ref{Padding impact on persistent sets} the reason is visualized. As shown in the figure we would expect that both p and r would be added in the persistent set. However when the p thread is added the write event is no longer visible, 
according to the persistent set definition. As a the r thread is not added. We can compare the LLVM code with the relative Erlang code.

\trace{rwrpersistent.pdf}{Padding impact on persistent sets}

\Code{./code/readers_rwr.erl}{Erlang code for rwr}
%%\Code{"./code/rwr.c"}{C code for rwr}
%%\Code{"./code/rwr.ll"}{Produced LLVM code for rwr}
\Side{./code/rwr.c}{C code for writer and reader}{./code/rwr.ll}{LLVM code for writer and reader}{Comparison between C and LLVM}


\section{Evaluation of Bounding Techniques}
The evaluation of the techniques takes into account two aspects. The number of traces explored and the soundness. The former is closely related with the amount
of time required for a bug to be found or the whole state space to be explored. The second is important because it demonstrates the tradeoff between time and accuracy
of the results. It is intelligible that a faster algorithm may compromise the soundness of the state space.
\subsection{Evaluation of Bounding Techniques on Synthetic tests}

The results for the testcases are demonstrated below. Again they are presented in two different ways.

%%\begin{center}
%%    \begin{tabular}{ |c|c|c|c|c|c|c|}
%%    \hline
%%    \multicolumn{1}{|c|}{Technique:} & \multicolumn{2}{c|}{Vanilla-BPOR} & \multicolumn{2}{c|}{BPOR} & \multicolumn{2}{c|}{Source-BPOR} \\
%%    \hline
%%    Bound: & 0 & 1 & 0 & 1 & 0 & 1 \\
%%    \hline \hline
%%    account.c & 1 & 6 & 6 & 1 & 27 & 27 \\
%%    \hline
%%    lazy.c & 1 & 6 & 6 & 1 & 27 & 27 \\
%%    \hline
%%    \end{tabular}
%%\end{center}
\graph{/home/yannis/nidhugg/tests/mytests/wNrB.png}{writer-N-readers bounded}
\smalltabular{"/home/yannis/nidhugg/tests/mytests/bounded.tex"}{Traces for various bound limits}

As it was expected the Vanilla-BPOR explores significantly less traces than the BPOR and the source-DPOR. However, as it was previously discussed, the whole
state space is not explored. The number of traces explored by the sound algorithms is significantly greater and it caused by the many conservative branches that are
added in order to achieve soundness. Surprisingly, there is no difference between the other two bounding techniques. An explanation is given later.

\subsection{Evalution of Bounding Techniques on RCU}
The results are demonstrated below. Notice that since the Source-DPOR did not resulted less traces than the DPOR we could not expect from the Source-BPOR and BPOR
to differentiate. Moreover tests did not show any difference. For these reasons only the performance of Vanilla-BPOR and BPOR is examined. In each table
the results with a given bound are demonstrated. Specifically the exploration time and the number of traces are shown. 
Moreover there is a cell indicating whether the assertion was found (We note F for found and NF for not found).


\bigtabular{"/home/yannis/rcu/valtree/nobound.tex"}{RCU results without bound}
\bigtabular{"/home/yannis/rcu/valtree/bound_0.tex"}{RCU results for bound $b=0$}
\bigtabular{"/home/yannis/rcu/valtree/bound_1.tex"}{RCU results for bound $b=1$}
\bigtabular{"/home/yannis/rcu/valtree/bound_2.tex"}{RCU results for bound $b=2$}
\bigtabular{"/home/yannis/rcu/valtree/bound_3.tex"}{RCU results for bound $b=3$}
\bigtabular{"/home/yannis/rcu/valtree/bound_4.tex"}{RCU results for bound $b=4$}
\bigtabular{"/home/yannis/rcu/valtree/comp_traces.tex"}{Comparison between DPOR and BPOR}


We notice that some assertions are found significantly faster. The most spectacular result is the \verb|-DFORCE_FAILURE_3| which is found in only 6 seconds for bound b=3 whereas it requires 464.77 seconds in the unbounded version. Moreover we notice for bound b=4 all the errors that are found in the unbounded version are
found. As a result the empirical observation that errors occur in low bound count seems to be confirmed. However, we have to underline that these are contrived
errors aiming to verify the correctness of the rcu and as a result they cannot be regarded as substantial evidences. As it is expected for larger bounds (b=4) the number of traces
grows exponentially. An other impressive result is that when the bound grows larger the errors takes longer to be found. If we take a look at \verb|-DFORCE_FAILURE_3| again we notice that the error
takes significantly longer to be tracked even through it exposed for the first time at bound b=2. For b=4 the exploration will was stopped since it exceeded 100,000 traces.
On the other hand many assertions are found faster with source-DPOR.

\subsection{A known bug}
As it was discussed in previous section, the scheduling priorities of Nidhugg should be changed in order for the running thread to be prioritize since it does not
increase the bound count. However this alternation in the priority causes Nidhugg to explore many more traces in unbounded search for an unknown reason. In order to deal with
this problem alternation in priority occurs only when bound is applied. As a result the comparison between DPOR and BPOR is not fair. Looking at table \ref{Comparison between DPOR and BPOR with the bug}
we can clearly see that the minimum traces required for BPOR to track the bug for the first time are always less than DPOR

\bigtabular{"/home/yannis/rcu/valtree/comp_trace_priority.tex"}{Comparison between DPOR and BPOR with the bug}

\section{Equivalence between BPOR and Source-BPOR (Correctness of Source-BPOR)}
Surprisingly the results of BPOR and Source-BPOR always coincide. However, further investigation of this behavior can reveal that these two techniques
are actually equivalent. 

It can be proved that a branch which was rejected by the Source-DPOR but accepted by the BPOR algorithm as a non-conservative one will be added as 
conservative by the source-bpor algorithm.

Let us assume a branch of the thread $b$ that is added as a non-conservative by the BPOR algorithm.

By the definition of persistent-sets this means that there is a $t \in T$ which
conflicts with an execution step of b. 

This non-conservative branch is rejected by the Source-BPOR. We know that there must be
 a trace such that thread b occurs before t. Since b was rejected there must be another branch s which shares the same initials
  with b, 
  
When s is scheduled another block will be created.
 
\begin{itemize}
\item Case 1: s is an execution step which conflicts with b. Hence, there must be a trace where b happens before s. As a result b belongs to the source set.
As shown in the Figure \ref{Source-BPOR and BPOR equivalence Case 1}, the branch which seems to be initially rejected, will finally be added by the Source-DPOR and as a result belongs to the source-set.
\trace{equivalence_case1.pdf}{Source-BPOR and BPOR equivalence Case 1}
   
\item Case 2: 
   s doesn’t conflict with b (both b and s are read operations). There must b an trace s.b.t (where s,b,t is the execution
   of all the steps of s,b,t). 
   Since t conflicts with an execution step of s the first step of b is an initial for t and it will be added both as non-conservative branch and 
   as conservative at the beginning of the block where it was rejected by the source-dpor. For Figure \ref{Source-BPOR and BPOR equivalence Case 2}, both q and r belong to the persistent set. However,
   the r thread will be rejected since it shares the same initials with the q thread. However it will be added as a conservative set. Notice that it would be added as a
   non-conservative as well but we have already shown that when both conservative and non-conservative branches of the same thread are added we must keep the conservative one.

   \trace{equivalence_case2.pdf}{Source-BPOR and BPOR equivalence Case 2}
\end{itemize}
   
A more intuitive explanation of the equivalence of the two techniques would be this: Persistent sets add threads in points higher in the trace. As a result, an equivalent trace
may have already been explored when these branches are scheduled, leading to sleep set blocked sets. However, it obvious that these branches would lead to equivalent traces of
lower bound count and thus, they would be add as conservative branches by the BPOR algorithm.

We have proved that Source-Bpor is sound since the traces explored by the bpor are subset of the traces explored by the Source-BPOR.
