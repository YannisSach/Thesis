\chapter{Further Discussion on Bounding Problem}
\label{Chapter 5}

In this chapter we discuss alternative ideas of approaching the preemption bounding problem of the DPOR. 
We suggest a new approach which is shown to be equivalent to the addition of conservative branches but does not require the addition of conservative
branches.

\section{Techniques without the Addition of Conservative Branches}

In previous chapters we discussed the challenges one have to deal with when designing a Bounded DPOR algorithm. We saw that there there is 
no apparent significant improvement can be made with the use of conservative branches and that other optimizations used for the unbouded versions of DPOR
cannot have analogous results. 

\subsection{Motivation}
The only algorithm that does not add any conservative branch is the Naive-BPOR. For a sufficient bound an erroneous trace would have still be found using this technique.
The drawback of this algorithm is its unsoundness. In this algorithm a function which calculates the number of preemptive switches in the current thread is used.
However many of the preemptive switches that are counted would be avoided.

An example is given in Figure \ref{Motivation} further explaining this idea. Let us assume a program consisting of two processes $p$ and $q$. 
Process $p$ writes a shared variable $x$ and process $q$ reads the shared variable $y$ (which is not modified at any point by any other process), 
and the shared variable $x$. There are only two possible schedulings i.e. one that the writing of $x$ precedes the reading and one that these two
commands are reversed. However, in order to make the reversion a preemptive switch must be introduced. Suppose that the first execution sequence is 
$q.q.p$, the other will be $q.p.q$ since the first step of $q$ has no race with step of $p$.

\trace{motivation.pdf}{Motivation}

In Figure \ref{Motivation} the preemptive switch that takes places would have been easily avoided there is an obvious inversion of the two blocks $q:read(y)$ and $p$.
But what allows such an inversion?

The answer lies to the events of block (in this particular case the block consists of one step only). The first block reads a variable which is not used by any other block. 
It is fathomable that this block can be switched with the
next block since there is no a happens before relationship with the two blocks.

This observation leads to the next question: Which of the preemption switches are compulsory? (Or equivalently which traces cannot be produced without a preemption switch?)
Moreover, is it possible for a given trace to calculate the minimum number of preemptive switches among all the equivalent traces?

\subsection{An Algorithm without Conservative Branches}
An algorithm that would preform such a bounded search would be different from the Naive-BPOR only concerning the function that calculates the bound count 
of the traces. This function $f$ would have to be constantly ascending i.e. it would not be possible to calculate a lower bound later for the same traces.
Given a suffix $E$, $f(E) \leq f(E.E')$ for any $E'$.

The general form of the algorithm is given in Algorithm \ref{NBBPOR}.

\begin{algorithm}[H]
    \SetAlgoLined
    \caption{General form of the BPOR without branch addition}
    \label{NBBPOR}
    \KwResult{Explore the whole state space within the bound}
    Explore($\emptyset$)\;
    \Fn{Explore($S$)}{
        T = Sufficient\_set($final(S)$)
     \For{all $t \in T$}{
         \If{$min\{B_v([S.t])\} \leq c$}{
            Explore($S.t$)
         }
        }
    }
\end{algorithm}

We notice that instead of calculating the $Bv(S.t)$ i.e., the bound count of the current trace value, 
we calculate the $min(Bv[S.t])$ i.e., minimum of all $Bv$ values of the traces that are equivalent with $S.t$.

\subsection{Calculating Minimum Bound Count}
The only thing left is the construction of this function f.
For a given trace $E$ which consists of blocks many happens-before relations hold. Each equivalent trace should also compensate to these relations.
It is also possible for different instructions in one block different happens-before relations hold true. For this section only, we will consider that 
a happens-before relation is a relation that holds between blocks. This is done for two main reasons:

\begin{itemize}
    \item The algorithm described later is simplified.
    \item We are not interested in further breaking each block and as a result we can regard is block as an entity.
\end{itemize}

These happens-before relations form a graph. This graph consists of nodes which are the blocks and edges which are these
relations. Obviously blocks of the same thread have a happens-before relation. We can also move from one block to another as long as these blocks happen
concurrently. We add weights to each edge. The edges that connect to blocks of the same thread weigh 0. Edges that start from a block that 
is blocked or the most recently added block of each thread weigh 0 since blocked blocks do not increase the bound count and we do not know if the last block of
each thread is indeed the last one. All the other edges which represent preemption points have weight 1.


Any traversal of the graph would not allow to traverse a block A that happens-before B before B, thus, all the happens-before relations should still hold true.
All traversals that cover the whole graph passing from each node only once are equivalent traces.

\noindent An algorithm on how to add a block to a given graph is given at Algorithm \ref{Adding a new block to the dependencies graph}. 
The algorithm works using induction.
Initially the graph consists of the first block. When a block of the trace is completed then we add it to the dependency graph. We connect the new block with each 
concurrent block with double edges with the new block. Moreover we connect the most recent block of each thread that happens before the new block with a directed edge
ending to the new block. 

\SetKw{Return}{return \;}
\SetKw{Break}{break \;}
\begin{algorithm}[H]
    \caption{Adding a new block to the dependencies' graph}
    \label{Adding a new block to the dependencies graph}
    \Fn{AddBlock(block,graph)}{
        \If{previous block of the same thread was not blocked}{
            increase the weigh of the edges coming from the previous block to 1 \;
        }

        \For{each thread t}{
            list:= preceding blocks t\;
            \For{l in reversed(list)}{
                \If{$l \leftrightarrow block$}{
                    add edge from $block$ to $l$ with weight $0$ \;
                    \If{$l$ is not last}{
                        add edge from $l$ to $block$ with weight $1$ \;
                    }
                    \Else{
                        add edge from $l$ to $block$ with weight $0$ \; 
                    }
                }
                \If{$l \rightarrow block$}{
                    \If{$l$ is not last}{
                        add edge from $l$ to $block$ with weight $1$ \;
                    }
                    \Else{
                        add edge from $l$ to block with weight $0$ \; 
                    }
                    \Break
                }
            }
        }
    }
\end{algorithm}


\trace{compulsoryswitch.pdf}{Graph example}

In Figure \ref{Graph example} a simple example of such a graph is demonstrated. For this trace we notice that $w(y)$ of $q$ thread is concurrent with $r(x)$ while 
it happens before $w(y)$ of the $p$ thread. Each transition costs 1 preemption switch that is why the weight is 1. Moreover transitions between the same thread cost 0.
The most important fact, however, is that if for any reason we try to violate the happen before relation (e.g. starting from $r(x)$ we jump to $w(x)$) there is no way to 
traverse all the nodes. 

We can see that there is a hamiltonian path with weight 1 for the given trace. In fact, this is the minimum hamiltonian path of the graph. We can easily realize that 
there is no equivalent trace with the initial one that has bound count less than 1. 

We can infer that the calculation of this bound count is reduced to the weight of the minimum hamiltonian path of this graph. This problem it is known to be NP-complete. 
As a result any algorithm that would calculate this weight would not be significantly better than a DFS-exploration. 

This is an extremely interesting indication of the difficulty of the DPOR bounding problem since the addition of the conservative sets imply this DFS
exploration at the state space. As a result this algorithm would not be better than the already proposed BPOR algorithm.

Now that the difficulty of this approach is clear a new question arises. Is it possible to approximate the total weight of the minimum hamiltonian path?
Such an algorithm would cover a greater state space than the Naive-BPOR without the explosion caused by the conservative branches.

\subsection{Approximating Bound Count}
There are two approaches examined in order to approximate the value of the bound count. The idea of both algorithms is based on this observation: 
A preemption switch is compulsory
when for two blocks of the same process A a block of another process B must intervene in order for the happens-before relations to hold true. As a result, the execution of 
the first A block should stop so the execution of the B block takes place followed by the execution of the A block again. Hence it should hold $e_1(A) \rightarrow e(B) \rightarrow e_2(A)$.
In case of $e_1(A) \not \rightarrow e(B)$ or $e(B) \not \rightarrow e_2(A)$ we could invert the blocks without affecting the happens-before relations and, thus construct an
equivalent trace with lower bound count.

The algorithm is presented here:\\

\SetKwProg{Fn}{Function}{}{}
\begin{algorithm}[H]
    \caption{First Approximation Algorithm}
    \label{First Approximation Algorithm}
    \Fn{BoundCount($E$,$current\_bound$)}{
        \For{$i=0 \text{ to } len(E)-1$}{
            \If{$E[i].pid = last(E).pid$}{
                $higher\_block = i$ \;
                \Break
            }
        }
        \For{$i = higher\_block+1 \text{ to } len(E)-1$}{
            \If{$ E[higher\_block] \rightarrow E[i] \rightarrow last(E)$}{
                current\_bound++ \;
                \Return
            }
        }
    }
\end{algorithm}

In Algorithm \ref{First Approximation Algorithm} we find the most recent block with the same pid as with the last block. We, then try to find if there is an event that happens before the first
and after the last event. If exists such an event we increase the counter.
Notice that for establishing the happens-before relation vector clocks can be used.
Moreover, it is obvious that more happens-before relations can be counted.


Th second algorithm explores more state space than it is required.\\

\begin{algorithm}[H]
    \caption{Second Approximation Algorithm}
    \Fn{BoundCount($E$,$current\_bound$)}{
        \For{$i=len(E)-1 \text{ to } 0$}{
            \If{$E[i].pid = last(E).pid$}{
                $lower\_block = i$ \;
                \Break
            }
        }

        \For{$i = lower\_block+1 \text{ to } len(E)-1$}{
            \If{$ E[lower\_block] \rightarrow E[i] \rightarrow last(E)$}{
                current\_bound++\;
                \Return
            }
        }
    }
\end{algorithm}

This algorithm starts the search for an event that intervenes the two events of the same id from the immediately previous block 
with the same id. 

\subsection{Evaluation}
The previous discussed approaches were tested and produced some interesting results. The both estimation algorithms seem to be "more sound" than the BPOR and
may explore traces that exceed the bound. This stems from the fact that they tend to underestimate the bound count since there are more complex relations 
between blocks that result traces with higher bound count than the one estimated. We notice that in writer-N-readers example the number of traces explored
is stable for every bound. In fact, each trace of this test has an equivalent trace with zero bound count since in each thread only a command related to
a shared variable is executed.

\subsection{Evaluation of Approximating algorithms}

\graph{/home/yannis/nidhugg/tests/mytests/wNrL1B.png}{writer-N-readers bounded by the first estimation algorithm}
\smalltabular{"/home/yannis/nidhugg/tests/mytests/lazy1_bounded.tex"}{Traces for the first estimation algorithm for various bound limits}


\graph{/home/yannis/nidhugg/tests/mytests/wNrL2B.png}{writer-N-readers bounded by the second estimation algorithm}
\smalltabular{"/home/yannis/nidhugg/tests/mytests/lazy2_bounded.tex"}{Traces for the second estimation algorithm for various bound limits}

\subsection{Implementation of Lazy-BPOR}

Some of the testcases such as lastzero or writer-N-readers made clear that an implementation of a bound count function which does not simply counts the preemptive switches in traces can prevent
the state space explosion caused by the conservative branches added. The next step is to implement the Lazy-BPOR, an algorithm that calculates the number of compulsory
preemptive switches (preemptive switches that cannot be avoided in any equivalent trace with the one examined). The main difference from the Naive-BPOR is that
the Lazy-BPOR maintains throughout the execution of the DPOR a graph of the blocks that are contained in the traces. When a new block is created, it is added by the
algorithm previously described. When it comes to the calculation of the bound count, the minimum hamiltonian path is calculated. The weight of this path corresponds
to the bound count taken into consideration.

\begin{algorithm}
    \caption{Lazy-BPOR}
    \label{Lazy-BPOR}
    \Let{$G =: \emptyset$}
    Explore($\langle \rangle$,$\emptyset$,$G$,$b$)\;
    \Fn{Explore($E$,$Sleep$,$G$,$b$)}{
        \If{$\exists p \in (enabled(s_{[E]}) \backslash Sleep)$ such that $B_v(E.p) <= b$ }{
            backtrack(E) $:={p}$ \;
            \While{$\exists p \in (backtrack(E) \backslash Sleep $}{
                \ForEach{$e \in dom(E)$ such that $e \lesssim_{E.p} next_{[E]}(p) $}{
                    \Let{$E' = pre(E,e)$}
                    \Let{$u = notdep(e,E).p$}
                    \If{$I_{E'}(u) \cap backtrack(E') = \emptyset$}{
                        add some $q' \in I_{[E']}(u) to backtrack(E') $ \;
                    }
                }
                \Let{$Sleep' := \{q \in Sleep \mid E \models p \diamondsuit q \} $}
                \If{$p$ creates a new block}{
                    \Let{$block$ = $last\_block(E)$}
                    \Let{$G'$ = add\_block($block$,$G$)}
                }
                \If{$min \{ Hamiltonian\_path(G') \} <= b $}{
                    $Explore(E.p, Sleep,G',b)$ \;
                    add $p$ to $Sleep$ \;
                }
            }
        }
    }
\end{algorithm}


\subsection{Lazy-BPOR - RCU Evaluation}

The results are demonstrated below. Since we have to compare Lazy-BPOR with BPOR the bugged versions of the DPOR must be used. The bugged version
of DPOR is that where the last running thread is prioritized.

Here we present the evaluation of the algorithm on RCU.

\bigtabular{"/home/yannis/rcu/valtree/lazy_comp.tex"}{Comparison between DPOR and Lazy-BPOR}

\bigtabular{"/home/yannis/rcu/valtree/lazy_buged_comp.tex"}{Comparison between DPOR and Lazy-BPOR without the bug}


We compare the algorithm with BPOR. We notice that Lazy-BPOR examines less traces but requires longer time since the cost of the lazy bound count is significantly
increased.

\bigtabular{"/home/yannis/rcu/valtree/preem_lazy_comp.tex"}{Comparison between BPOR and Lazy-BPOR(buged)}

