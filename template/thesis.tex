\documentclass[diploma, english]{softlab-thesis}

\usepackage{amsmath}
\usepackage{float}
\usepackage{color}
\usepackage[table,xcdraw]{xcolor}
\usepackage{siunitx}
\usepackage{multirow}
\usepackage{colortbl}
\usepackage{subcaption}
%\usepackage{refcheck}

\usepackage{url}
\usepackage[colorlinks]{hyperref}
\hypersetup{
  bookmarksnumbered,
  citecolor={blue},
  linkcolor={blue},
  urlcolor={blue},
  pdfpagemode={UseOutlines}
}

\usepackage{amsthm}
\theoremstyle{definition}
\newtheorem{definition}{Definition}[chapter]

\usepackage{listings}
\usepackage{lstlinebgrd}
\lstset{
    numbers=left,
    numberstyle=\tiny\color{black},
    basicstyle=\ttfamily\footnotesize,
    basewidth=0.59em,
    keywordstyle=[3]{},
    commentstyle=\itshape\footnotesize,
    tabsize=8,
    frame=single,
    frameround=tttt,
    showstringspaces=false,
    breaklines=false,
    captionpos=b,
    aboveskip=\bigskipamount,
    belowskip=\bigskipamount,
    escapechar=^
}
\lstdefinestyle{custom}{
    numbers=left,
    numberstyle=\tiny\color{black},
    basicstyle=\ttfamily\footnotesize,
    basewidth=0.59em,
    keywordstyle=[3]{},
    commentstyle=\itshape\footnotesize,
    tabsize=8,
    frame=single,
    frameround=tttt,
    showstringspaces=false,
    rulecolor=\color{black},
    breaklines=false,
    captionpos=b,
    aboveskip=\bigskipamount,
    belowskip=\bigskipamount,
    escapechar=^,
    moredelim=**[is][\color{blue}]{@}{@}
}

\lstnewenvironment{code}[2]{
  \nopagebreak
  \lstset{language=C, label={#1}, caption={#2}, style=custom}
}{}

\lstnewenvironment{code_appendix}{
  \nopagebreak
  \lstset{language=C, style=custom, numbers=none,rulecolor=\color{black}}
}{}

\lstnewenvironment{console}[2]{
  \nopagebreak
  \lstset{label={#1}, caption={#2}, numbers=none, style=custom}
}{}
 
\lstnewenvironment{assembly}[2]{
  \nopagebreak
  \lstset{language={[x86masm]Assembler}, label={#1}, caption={#2}, style=custom}
}{}

\usepackage{etoolbox}
\expandafter\patchcmd\csname \string\lstinline\endcsname{%
  \leavevmode
  \bgroup
}{%
  \leavevmode
  \ifmmode\hbox\fi
  \bgroup
}{}{%
  \typeout{Patching of \string\lstinline\space failed!}%
}
\newcommand{\ccode}[1]{\lstset{language=C}\protect\lstinline!#1!}

%%%
%%%  The document
%%%

\begin{document}

%%%  Title page

\frontmatter

\title{Systematic Concurrency Testing of Read-Copy-Update under Sequentially Consistent and Weak Memory Models}
\author{Michalis Kokologiannakis}
\date{October 2016}
\datedefense{13}{10}{2016}

\supervisor{Konstantinos Sagonas}
\supervisorpos{Associate Professor NTUA}

\committeeone{Konstantinos Sagonas}
\committeeonepos{Associate Professor NTUA}
\committeetwo{Nikolaos S. Papaspyrou}
\committeetwopos{Associate Professor NTUA}
\committeethree{Nectarios Koziris}
\committeethreepos{Professor NTUA}

\TRnumber{CSD-SW-TR-1-16}  % number-year, ask nickie for the number
\department{Division of Computer Science}

\maketitle


%%%  Abstract, in Greek

\iffalse
\begin{abstractgr}%
  
H διεξοδική επαλήθευση και έλεγχος προγραμμάτων γραμμένων στο πρότυπο του ταυτοχρονισμού είναι ένα μιά τόσο σημαντική
όσο και απαιτητική εργασία. Προκειμένουν να επαληθεύσουμε την ορθότητα ενός τέτοιου προγραμμάτος θα πρέπει να εξετάσουμε
όλες τις δυνατές δρομολογήσεις που αυτό μπορεί να παράξει. Το Stateless model checking με τη χρήση αναγωγής σε δυναμικές
σχέσεις μερική διάταξης (Dynamic Partial Order Reduction ή DPOR) είναι μια τεχνική η οποία αντιμετωπίζει το πρόβλημα του της
έκρηξης του μεγέθους του χώρου καταστάσεων. Παρολ' αυτά η επαλήθευση μεγάλων προγραμμάτων μπορεί να διαρκέσει
περισσότερο από όσο θα επιθυμούσαν οι developers. Σε αυτές τις περιπτώσεις ο περιορισμός (οριοθέτηση) της αναζήτησης
(bounded search) με τη χρήση κάποιου ορίου μπορεί να αποδειχθεί αρκετά χρήσιμη. Η οριοθετημένη αναζήτηση, σε αντίθεση με την DPOR 
απαλείφει το πρόβλημα της έκρηξης του μεγέθους το χώρου καταστάσεων αγνοώντας δρομολογήσεις που ξεπερνούν κάποιο όριο

Στην παρούσα διπλωματική περιγράφεται η υλοποίηση του preemption bounded DPOR (BPOR) στο Nidhugg ένα εργαλία για την εύρεση λαθών (bugs)
που στοχεύει στην ανεύρεση σφαλμάτων που προκείπτουν από το μοντέλο του ταυτοχρονισμού και τα χαλαρά μοντέλα μνήμης (relaxed memory models).
Συγκεκριμένα υλοποιήθηκαν τρεις τεχνικές περιορισμού: ο Naive-BPOR, o Nidhugg-BPOR και ο Source-BPOR. Οι τεχνικές αυτές αξιολογήθηκαν τόσο σε 
συνθετικά τεστ όσο και σε λογισμικό που χρησιμοποιείται στην πραγματικότητα. Συγκεκριμένα ο μηχανισμός Read-Copy-Update του πυρήνα του Linux
επαληθεύτηκε ξανά με τη χρήση αυτών των τεχνικών. Επιπλέον εξετάστηκε κατα πόσο βελτιστοποιήσεις που εφαρμόζονται στον μη περιορισμένο DPOR μπορούν
να βελτιώσουν την επίδοση του BPOR.
  
\begin{keywordsgr}
\input{./keywords_gr.tex}
\end{keywordsgr}
\end{abstractgr}
\fi

%%%  Abstract, in English

\begin{abstracten}%
  
Thorough verification and testing of concurrent programs is an important, but also
challenging task. In order to verify a concurrent program one must examine all possible 
different interleavings the scheduler can produce. Stateless model checking with Dynamic Partial Order Reduction 
is a technique proposed to deal with state space explosion. Nevertheless, for larger programs the verification 
takes longer than the developers are willing to wait. In these cases, bounded search can be proved useful. Bounded search,
in contrast to the DPOR, alleviates state-space explosion by pruning the executions that exceed a bound. 

This thesis describes the implementation of the preemption bounded DPOR (BPOR) on Nidhugg, a bug finding tool which targets bugs caused by concurrency
and relaxed memory consistency in concurrent programs. Specifically three bounding techniques were implemented: the Naive-BPOR, the BPOR,
and the Source-BPOR. The three techniques were evaluated both in synthetic and in real world software. Specifically Read-Copy-Update mechanism of Linux Kernel
was verified again. Moreover it is examined whether optimizations that have been suggested for the 
unbounded DPOR can improve the efficiency of BPOR.
  
\begin{keywordsen}
Formal Verification, Stateless Model Checking, Systematic Concurrency Testing, RCU, Read-Copy-Update, Bounded Dynamic Partial Order Reduction,
\end{keywordsen}
\end{abstracten}


%%%  Acknowledgements

\iffalse
\begin{acknowledgementsgr}
\end{acknowledgementsgr}
\fi

\begin{acknowledgementsen}
  First of all, I am most grateful to my advisor, Kostis Sagonas, for all his help
  and support during this effort. With his ceaseless enthusiasm and avid interest
  in research he managed to motivate and inspire me, while his essential advice
  and guidance throughout this process have been invaluable. I absolutely enjoyed
  every minute of working with him for my thesis during the last year,
  and I am looking forward to working with him again in the future.

  I am also much obliged to Paul McKenney, for dedicating a lot of his free time
  to answer all of my questions willingly and promptly. With his 
  profound insight into RCU-related issues he helped me in numerous occasions,
  and his suggestions and input were extremely helpful.

  Finally, I would like to extend my thanks to my parents and brother for
  their patience, love and support throughout these years; after all, they are
  the ones who made this endeavour possible.
  
\end{acknowledgementsen}

%%%  Various tables

\tableofcontents
%\listoftables
\listoffigures
\listoflistings


%%%  Main part of the book

\mainmatter


Hello World

\chapter{Background}
\label{Chapter 2}

\section{Concurrent programming}

Concurrent computing, which is implemented by concurrent programming paradigm, is a form of computing in which several 
computations are executed during overlapping time 
periods—concurrently—instead of sequentially (one completing before the next starts). 
This is a property of a system—this may be an individual program, a computer, or a network—and there is a separate execution point 
or "thread of control" for each computation ("process"). A concurrent system is one where a computation can advance without waiting for 
all other computations to complete.
The main challenge in designing concurrent programs is concurrency control: ensuring the correct sequencing of the 
interactions or communications between different computational executions, and coordinating access to resources that are shared among executions.
Potential problems include race conditions, deadlocks, livelocks and resource starvation. 
The scheduler is usually responsible for running a thread. Due to this scheduling non-determinism the programmer cannot always be aware of which thread
will be scheduled next.

An important aspect of a concurrent program is the notion of the set of interleavings which is the set of all the execution paths a program can follow.
Intuitively, if we imagine a process as a (possibly infinite) sequence/trace of statements (e.g. obtained by loop unfolding),
then the set of possible interleavings of several processes consists of all possible sequences of statements of any of these processes.

As it can be inferred, debugging this kind of programs can be proved extremely challenging. The challenge mainly emerges from the fact that it is 
not always clear which thread command will be executed. Moreover the error may not always occur during debugging since there may be only a limited
number of interleavings that produce an error. 

\section{Testing, Model Checking and Verification}

Dynamic software model checking consists of adapting model check-ing into a form of systematic testing that is applicable to industrial-size software. 
Over the last two decades, dozens of tools following this paradigm have been de-veloped for checking concurrent and data-driven software. Compared to 
traditionalsoftware testing, dynamic software model checking provides better coverage, butis more computationally expensive. Compared to more general 
forms of programverification like interactive theorem proving, this approach provides more limitedverification guarantees, but is cheaper due to its higher
level of automation. Dynamic software model checking thus offers an attractive practical trade-off between testing and formal verification. 

A graph that greatly demonstrates the differences between these terms is show in 

\trace{testmodver.png}{Comparing Testing, Model Checking and Verification}


\section{Stateless model checking and partial order reduction}

In order to find an error of a concurrent program, one must examine every possible interleaving this program can produce. Usually the error 
would occur only under some interleaving the programmer did not take into consideration, making its detection extremely difficult. Stateless model checking is based on the idea of driving 
the program along all these possible interleavings. However, this approach suffers from state explosion, i.e., the number of all possible interleavings grows 
exponentially with the size of the program and the number of threads. Several approaches to this problem have been proposed in order to deal with this challenge: 
partial order reduction \cite{Godefroid1996} and bounded search \cite{BPOR}. Partial order reduction is aiming to reduce the number of interleavings explored by eliminating equivalent interleavings.
These equivalent traces are produced by the inversion of independent events which do not affect the results of the program. For example, the scheduling of two threads that read a local
variable can be inverted since the result of the operation is affected by the order under which each operation occurs. 
There are two ways that a partial order reduction algorithm can be implemented. The first is a
static partial order reduction algorithm \cite{Static1997} where the dependencies between two threads are tracked before the execution of the concurrent program. 
The second is the Dynamic partial order reduction (DPOR) which observes the program's dependencies of runtime. It is important to notice that the size of
the state space still grows exponentially. 

For larger programs DPOR often runs longer than developers are willing to wait. In these cases, bounded search can be proved useful. Bounded search,
in contrast to DPOR, alleviates state-space explosion by pruning the executions that exceed a bound \cite{Thomson}. There have been proposed many bounded techniques
such as preemption bounded search \cite{BPOR} or delay bounded search \cite{Delay11}. All bounded search techniques are based on the notion that many of the concurrency bugs can be
tracked even when the bound limit is set to be small, thus the time required for a bug to be found is significantly smaller.


\section{Vector Clocks}

A vector clock is an algorithm for generating a partial ordering of events in a distributed or concurrent system and detecting causality violations. 
Just as in Lamport timestamps \cite{Lamp}, interprocess messages contain the state of the sending process's logical clock. 
A vector clock of a system of N processes is an array/vector of N logical clocks, one clock per process; 
a local "smallest possible values" copy of the global clock-array is kept in each process, with the following rules for clock updates:

\begin{enumerate}
    \item Each process experiencing an internal event, it increments its own logical clock in the vector by one.
    \item Each time a process receives a message or performs an action on a shared variable, it increments its own logical clock in the vector by one and updates each element in its vector 
    by taking the maximum of the value in its own vector clock and the value in the vector in the received message or the maximum value of all processes that share
    the same shared variable.
\end{enumerate}

An example execution of the algorithm is shown in Figure \ref{Clock example} where both the source code and an explored trace are given.
As we can easily notice for each command the clock of the main thread increases. The thread <0.0> starts to run its clock for the thread <0> is 8 since 
that is the moment when the thread was spawned. When the value of y is read the clock for <0> increases again so it corresponds with the y=1 event. When the first thread is
scheduled again then its clock for the <0.0> is 7 since \verb|pthread_join()| command takes place.


\Side{./code/clocks.c}{Vector Clock example}{./code/clocks.out}{Vector Clock output}{Clock example}

Every algorithm that is presented in this thesis is based on vector clocks.


\section{Notation}

Before examining delving in the problem of dynamic partial order reduction it is crucial to explain the notation that will be used.
An execution sequence $E$ of a system is a finite sequence of
execution steps of its processes that is performed from the initial
state which we denote as $s_0$. Since each execution step is deterministic, an execution
sequence $E$ is uniquely characterized by the sequence of processes
that perform steps in $E$. For instance, $p.p.q$ denotes the execution
sequence where first $p$ performs two steps, followed by a step of $q$.
The sequence of processes that perform steps in $E$ also uniquely
determine the (global) state of the system after $E$, which is denoted
$s_{[E]}$. For a state $s$, let $enabled(s)$ denote the set of processes $p$ that
are enabled in $s$ (i.e., for which execute $p(s)$ is defined). We use $.$ to
denote concatenation of sequences of processes. Thus, if $p$ is not
blocked after $E$, then $E.p$ is an execution sequence.
An event of $E$ is a particular occurrence of a process in $E$.
We use $\langle p,i \rangle$ to denote the ith event of process $p$ in the execution
sequence $E$. In other words, the event $\langle p,i \rangle$ is the ith execution step
of process $p$ in the execution sequence $E$. We use $dom(E)$ to denote
the set of events $\langle p,i \rangle$ which are in $E$, i.e., $\langle p,i \rangle \in dom(E)$ iff $E$
contains at least $i$ steps of $p$. We will use $e,e',...$ , to range over
events. We use $proc(e)$ to denote the process $p$ of an event $e = \langle p, i \rangle$.
If $E.w$ is an execution sequence, obtained by concatenating $E$ and
$w$, then $dom_{[E]}(w)$ denotes $dom(E.w) \ dom(E)$, i.e. the events in
$E.w$ which are in $w$. As a special case, we use $next_{[E]}(p)$ to denote
$dom_{[E]}(p)$.
We use $<_E$ to denote the total order between events in $E$, i.e.
$e <_E e'$  denotes that $e$ occurs before $e'$  in $E$. We use $E'\leq E$ to
denote that the sequence $E'$ is a prefix of the sequence $E$.

\section{Event Dependencies}

One of the most important concepts when we have to deal with an algorithm that searches the whole state space of the different schedulings is the 
happens-before relation in an execution sequence. Usually this relation is denoted with $\rightarrow$ symbol. For example, if the relation $\rightarrow$ 
for two events $e,e'$ in $dom(E)$ holds true then the event $e$ happens-before $e'$. This relation usually appears in the message exchange, when $e$ is the message
transmission and $e'$ is the event when the message is received. For the context of Nidhugg $e \rightarrow e'$ would hold true when at least one of the two events
is a write operation on the same shared variable. It is fathomable that any DPOR algorithm should be able to assign this happens-before relations. 
In practice, the happens-before assignment is implemented with the use of vector clocks.

\begin{definition}{(happens-before assignment)}
    A happens-before assignment, which assigns a
    unique happens-before relation $\rightarrow E$ to any execution sequence
    $E$, is valid if it satisfies the following properties for all execution
    sequences $E$.
    \begin{enumerate}
        \item $\rightarrow_{E}$ is a partial order on $dom(E)$, which is included in $<_E$. In other words every scheduling is part of the set of all possible
        partial order of the program.
        \item The execution steps of each process are totally ordered, i.e. 
        $\langle p,i \rangle \rightarrow_E \langle p,i+1 \rangle$ whenever $\langle p, i+1 \rangle \in dom(E)$.
        \item If $E'$ is a prefix of $E$ then $\rightarrow_E$ and $\rightarrow_{E'}$ are the same on $dom(E')$.
        \item Any linearization $E'$ of $\rightarrow_E$ on $dom(E)$ is an execution sequence which has exactly the same “happens-before” relation
$\rightarrow_{E'}$ as $\rightarrow_E$. This means that the relation $\rightarrow_E$ induces a set
of equivalent execution sequences, all with the same “happens-before” relation. 
We use $E \simeq E'$ to denote that $E$ and $E'$ are
linearizations of the same “happens-before” relation, and $[E] \simeq$ 
to denote the equivalence class of E.
    \item If $E \simeq E'$ then $s_{[E]} = s_{[E']}$ (i.e. two equivalent traces will lead to the same state).
    \item For any sequences $E, E'$ and $w$, such that $E.w$ is an execution
sequence, we have $E \simeq E'$  if and only if $E.w \simeq' E'.w$.
    \end{enumerate}
\end{definition}

The first six properties should be obvious for any reasonable
happens-before relation. The only non-obvious is the
last. Intuitively, if the next step of p happens before the next step
of $r$ after the sequence $E$, then the step of $p$ still happens before
the step of $r$ even when some step of another process, which is not
dependent with $p$, is inserted between $p$ and $r$. This property holds
in any reasonable computation model that we can think of. As
examples, one situation is when $p$ and $q$ read a shared variable that
is written by $r$. Another situation is that $p$ sends a message that is
received by $r$. If an intervening process $q$ is independent with $p$, it
cannot affect this message, and so $r$ still receives the same message.
Properties 4 and 5 together imply, as a special case, that if $e$
and $e'$ are two consecutive events in E with $e \not \rightarrow_{E} e'$, then they can
be swapped and the (global) state after the two events remains the
same.

\section{Independence and races}

We now define independence between events of a computation. If
$E.p$ and $E.w$ are both execution sequences, then $E \models p\diamondsuit w$ denotes
that $E.p.w$ is an execution sequence such that $next_{[E]}(p) \not \rightarrow_{E.p.w} e$
for any $e \in dom([E.p])(w)$. In other words, $E \models p \diamondsuit w$ states that
the next event of $p$ would not “happen before” any event in $w$
in the execution sequence $E.p.w$. Intuitively, it means that $p$ is
independent with $w$ after $E$. In the special case when $w$ contains
only one process $q$, then $E \models p \diamondsuit q$ denotes that the next steps of
$p$ and $q$ are independent after $E$. We use $E'\models p \diamondsuit w$ to denote that
$E \not \models p \diamondsuit w$ does not hold.

For a sequence $w$ and $p \in w$, let $w \backslash p$ denote the sequence
$w$ with its first occurrence of $p$ removed, and let $w \uparrow p$ denote the
prefix of w up to but not including the first occurrence of $p$. For
an execution sequence $E$ and an event $e \in  dom(E)$, let $pre(E,e)$
denote the prefix of $E$ up to, but not including, the event $e$. For an
execution sequence $E$ and an event $e \in E$, let $notdep(e, E)$ be the
sub-sequence of $E$ consisting of the events that occur after $e$ but do
not “happen after” $e$ (i.e., the events $e'$ that occur after $e$ such that
$e \not \rightarrow_E e'$).


A central concept in most DPOR algorithms is that of a race.
Intuitively, two events, $e$ and $e'$ in an execution sequence $E$, where
$e$ occurs before $e'$ in $E$, are in a race if
\begin{itemize}
\item $e$ happens-before $e'$ in $E$, and
\item $e$ and $e'$ are “concurrent”, i.e. there is an equivalent execution
sequence $E' \simeq E$ in which $e$ and $e'$ are adjacent.
\end{itemize}
Formally, let $e \lessdot_E e'$ denote that $proc(e) \not = proc(e')$, that $e \rightarrow_E e'$,
and that there is no event $e'' \in dom(E)$, different from $e'$ and $e$,
such that $e \rightarrow_E e'' \rightarrow_E e'$.

Whenever a DPOR algorithm detects a race, then it will check
whether the events in the race can be executed in the reverse order.
Since the events are related by the happens-before relation, this may
lead to a different global state: therefore the algorithm must try to
explore a corresponding execution sequence. Let $e \lesssim_E e'$ denote
that $e \lessdot_E e'$, and that the race can be reversed. Formally, if $E' \lesssim E$
and $e$ occurs immediately before $e'$ in $E'$, then $proc(e')$ was not
blocked before the occurrence of $e$.


\section{Dynamic partial order reduction}

Before explaining the DPOR algorithm it is important to define sufficient sets.

\begin{definition}{(Sufficient Sets)}
A set of transitions is sufficient in a state $s$ if any relevant
state reachable via an enabled transition from s is also reachable from $s$ via at least one of the transitions in the sufficient
set. A search can thus explore only the transitions in the
sufficient set from s because all relevant states still remain
reachable. The set containing all enabled threads is trivially
sufficient in $s$, but smaller sufficient sets enable more state
space reduction.
\end{definition}

Many techniques have been proposed in order to implement a DPOR algorithm. What most of these techniques share in common is the following basic structure:
\SetKwProg{Fn}{Function}{}{}

\SetKwHangingKw{Let}{let}
\begin{algorithm}[H]
    \caption{General form of DPOR}
    Explore($\emptyset$)\;
    \Fn{Explore($E$)}{
     \Let{T = Sufficient\_set($final(E)$)}
     \For{all $t \in T$}{
        Explore($E.t$) \;
    }
    }
\end{algorithm}

where $final(E)$ represents the state that will be reached when the execution sequence $E$ is executed.

The algorithm above describes a DFS search in the state space of all possible interleavings.
As it can be inferred from the algorithm the most important step is that of the calculation of the set $T$.

\begin{definition}{(Enabled sets, $enabled(s)$)}
    Given a state $s$, $enabled(s)$ represents the set of all the threads that can be scheduled immediately after $s$.
\end{definition}

An obvious property that the sufficient sets must hold is that Sufficient\_set$(final(E)) \subseteq enabled(E)$.

Intuitively $enabled(s)$ represents the threads that are not blocked or have already finished their execution.

In bibliography many types of sufficient sets can be found \cite{Godefroid1996}. 
In this thesis we mainly focus on persistent sets and on source sets.


\section{Persistent Sets}

A persistent set in a state $s$ is a sufficient set of transitions to
explore from $s$ while maintaining local state reachability for acyclic state spaces \cite{God97}. A selective search using persistent
sets explores a persistent set of transitions from each state s where $enabled(s) \neq \emptyset$ and prunes enabled transitions that
are not persistent in s.
In a more formal way:\\

\begin{definition}{(Persistent Sets)}
Let $s$ be a state, and let $W \subseteq E(s)$ be a set
of execution sequences from $s$. A set $T$ of transitions is a persistent set for $W$
after $s$ if for each prefix $w$ of some sequence in $W$, which contains no occurrence
of a transition in $T$,  we have $E \vdash t \diamondsuit w$ for each $t \in T$.
\end{definition}

The above definition can be described as follows: If $t \in T$ and there is another thread $t'$ that can be executed until a command which
is in a race with $t$, then $t'$ belongs in the persistent set.

Notice that the definition of persistent sets suggests a way to construct them.

In Figure \ref{Construction of persistent sets} two different examples of persistent set construction are given. We denote the persistent set of branches the execution will take with $BR{}$.
In the first, let a concurrent program contain 3 threads $p$, $q$, and $r$. Thread $p$ changes the value of the variable (writer) and the other ($q$ and $r$) just read this variable (readers).
Let $p.q.r$ be an interleaving. According to the definition of the persistent sets $q$ and $r$ are in a race with $p$, thus, $q$ and $r$ must also be on the persistent set
of the first command of the interleaving. In Figure \ref{Construction of persistent set} we notice that both $r$ and $q$ threads are added to the persistent set of the first
command of the trace since both conflict with the write operation. 
In the second example, let $p$ and $r$  be a readers and $q$ be a writer. We notice that both $r$ and $q$ are added. However, there is no conflict between $p$ and $r$ since both $p$ and $r$
just read the variable $x$. The reason why the thread $r$ is added is the conflict that will be produced by the $q$'s write operation.

\trace{persistent.pdf}{Construction of persistent set}

\section{Source sets}

Before defining source sets, we give some other useful definitions.

\begin{definition}{($dom(E)$)}
    The set of events-transitions happening during the scheduling of $E$.
\end{definition}

\begin{definition}{(Initials after an execution sequence $E.w$, $I_{[E]}(w)$)}
For an execution sequence $E.w$, let $I_{[E]}(w)$ denote the set of
processes that perform events $e$ in $dom_{[E]}(w)$ that have no
“happens-before” predecessors in $dom_{[E]}(w)$. More formally,
$p \in I_{[E]}(w)$ if $p \in w$ and there is no other event $e \in dom_{[E]}(w)$ with
$e \rightarrow_{E.w} next_{[E]}(p)$.
\end{definition}

\begin{definition}{(Source Sets)}
Let $S$ be an execution sequence,
and let $W$ be a set of sequences, such that $E.w$ is an execution
sequence for each $w \in W$. A set $T$ of processes is a source set for
$W$ after $E$ if for each $w \in W$ we have $WI_{[E]}(w) \cap P  = \emptyset$.
\end{definition}

A source set is a set of threads that guarantee that the whole state space will be explored. Notice that their is no requirement related to the races
of the events.
What the above definition implies is that source can be considered every set of threads that contains these threads that are able to cover the whole state-space.
It actually suggests a property for the sufficient sets to hold.

\section{Sleep sets}

Another technique complementary to the persistent or source sets aiming to reduce the number of interleavings is the sleep set technique.
Sleep sets prohibit visited transitions from executing again
until the search explores a dependent transition. Assume that
the search explores transition $t$ from state $s$, backtracks $t$,
then explores $t_0$ from $s$ instead. Unless the search explores
a transition that is dependent with $t$, no states are reachable
via $t_0$ that were not already reachable via t from s. Thus, t
“sleeps” unless a dependent transition is explored.

A short example on sleep sets is the following:
Let us the concurrent program of one writer and two readers.
let w1 <0.0>: w(x) r1 <0.1>: (local operations), r(x) and r2 <0.2>: (local operations), r(x).

The resulted traces are demonstrated in the Listing \ref{Sleep set example}.

\Output{./code/sleep_sets.out}{Sleep set example}

As we can see from the execution of the DPOR algorithm the interleaving which started from r2 was blocked since it would lead to an interleaving which
has already been explored. Notice that this is due to the fact that r1 cannot wakeup since its first transition (local operations) does not conflict with any other transition
in the program. 
It can be proved \cite{Godefroid1996} that sleeps will eventually block all the redundant interleavings and thus the only interleavings that will be explored till their end (where all threads that could be executed, have been executed).
As a result an optimal algorithm should be able to not consider these interleavings whatsoever.

\section{Comparing Persistent sets with Source Sets}

Note that the definition of source sets is much more relaxed than the definition of the persistent sets. 
This relaxation enables the source sets to be much more efficient than the persistent sets. In Figure \ref{Non-minimal persistent sets}
an example is given were source sets and persistent sets differ.

\begin{figure*}
    \begin{lstlisting}[frame=none,numbers=none]
        Initially: x = y = z = 0 
    \end{lstlisting}
    \begin{minipage}{0.3\textwidth}
      \begin{lstlisting}[frame=none, numbers=none]
        p:
        m := x; (p1)
        if (m = 0) then
            z := 1; (p2)
      \end{lstlisting}
    \end{minipage}
    \begin{minipage}{0.3\textwidth}
        \begin{lstlisting}[frame=none, numbers=none]
            q:
            n := y; (q1)
            if (n = 0) then
                x := 1; (q2)
        \end{lstlisting}
      \end{minipage}
      \begin{minipage}{0.3\textwidth}
        \begin{lstlisting}[frame=none, numbers=none]
            r:
            o := z; (r1)
            if (o = 0) then
                y := 1; (r2)
        \end{lstlisting}
      \end{minipage}
      \caption{Program with non-minimal persistent sets}
      \label{Non-minimal persistent sets}
  \end{figure*}

From the example, it is clear that the reason why source sets are an improvement over persistent sets is the fact that minimum source sets can eliminate
sleep set blocked traces i.e. traces that would eventually be blocked by the sleep sets. An algorithm that would only calculate minimal source sets would be optimal \cite{AbdullaAronisJohnssonSagonasDPOR2014}, hence
would never explore two equivalent interleavings.

It is obvious that a single transition cannot be a source set. For
instance, the set $\{ p_1 \}$ does not contain the initials of execution $q_1.q_2.p_1.r_1.r_2$,
since q2 and p1 perform conflicting accesses. On the other hand, any subset
containing two enabled transitions is a source set. To see this, let us choose
$\{p_1, q_1 \}$ as the source set. Obviously, $\{p1, q1 \}$ contains an initial of any execution
that starts with either $p_1$ or $q_1$. Any execution sequence which starts with $r_1$ is
equivalent to an execution obtained by moving the first step of either $p_1$ or $q_1$ to
the beginning:
\begin{itemize}
\item If $q_1$ occurs before $r_2$, then $q_1$ is an initial, since it does not conflict with
any other transition.
\item If $q_1$ occurs after $r_2$, then $p_1$ is independent of all steps, so $p_1$ is an initial.
We claim that $\{p_1, q_1 \}$ cannot be a persistent set. The reason is that the execution
sequence $\{r_1.r_2 \}$ does not contain any transition in the persistent set, but its second
step is dependent with $q_1$. By symmetry, it follows that no other two-transition
set can be a persistent set.
\end{itemize}

In other words, persistent sets have the unpleasant property that adding a process
may disturb the persistent set so that even more process may have to be added.
This property is relevant in the context of DPOR, where the first member of the
persistent set is often chosen rather arbitrarily (it is the next process in the first
exploration after $E$), and where the persistent set is expanded by need.

Continuing the comparison between source sets and persistent sets, we first
note some rather direct properties, including the following.

\begin{itemize}
\item Any persistent set is a source set.
\item Any one-process source set is a persistent set.
\end{itemize}


\section{Bounded search - preemption bounded search}
Bounded search explores only executions that do not exceed
a bound \cite{BPOR,Thomson}. The bound may be any property of a
sequence of transitions. A bound evaluation function $Bv(S)$
computes the bounded value for a sequence of transitions S.
A bound evaluation function $B_v$ and bound $c$ are inputs to
bounded search. Bounded search may not visit all relevant
reachable states; it visits only those that are reachable within
the bound. If a search explores all relevant states reachable
within the bound, then it provides bounded coverage.

An algorithm that could describe a bounded search would be the following:

\begin{algorithm}[H]
    \caption{Bounded-DPOR}
    \KwResult{Explore the whole statespace}
    Explore($\emptyset$)\;
    \Fn{Explore($S$)}{
        T = Sufficient\_set($final(S)$)
     \For{all $t \in T$}{
         \If{$Bv(S.t) \leq c$}{
            Explore($S.t$)
         }
        }
    }
\end{algorithm}

\noindent The only difference between the unbounded and the bounded version of the algorithm is the if statement on line 4 which allows for an interleaving to be explored
only if the bound has not been exceeded.

What is needed next is an appropriate definition of the function $B_v$ that calculates a value that the bounded-DPOR tries to keep bounded, 
and the sufficient set. 

In this thesis, we mainly focus on preemption-bounded search. 

Preemption-bounded search limits the number of preemptive context switches that occur in an execution \cite{Musu07}. The 
preemption bound is defined recursively as follows.

\begin{definition}{Preemption bound}
$P_b(t) = 0$ \\
$P_b(S.t) = 
 \begin{cases} 
    P_b(S) + 1 & \text{ if } t.tid = last(S).tid \text{ and } last(S).tid \in enabled(final(S)) \\
    P_b(S) & \text{ otherwise }
 \end{cases}
$\\
\end{definition}

The previous definition describes what a preemptive context switch is. A preemptive context switch happens when the previously running thread could execute
its next step but it does not due to the scheduling of another thread. Hence, a preemptive switch will increase the preemption bound.

\section{Preemption-bound persistent sets}

A set that has been proposed as a sufficient for preemption bounded search is the preemption bounded persistent set \cite{BPOR}.

An important observation is that the execution of a thread until it gets blocked or terminates will not increase
the bound count.
\begin{definition}{($ext(s,t)$)}
    Given a state $s = final(S)$ and a transition $t \in enabled(s)$,
    $ext(s,t)$ returns the unique sequence of transitions $\beta$ from $s$
    such that
    \begin{enumerate}
        \item $\forall i \in dom(\beta): \beta_i.tid = t.tid$
        \item $t.tid \notin enabled(final(S.\beta))$
    \end{enumerate}
\end{definition}

Next, we need to define preemption bounded persistent sets. We denote with $A_G(P_b,c)$ the generic 
bounded state space with bound function $P_b$ and bound $c$. $last(a)$ denotes the last execution step of
an execution sequence $a$

\begin{definition}{(Preemption bounded persistent set)}

A set $T \subseteq \mathcal{T}$ of transitions enabled in a state $s=final(S)$
is preemption-bound persistent in $s$ iff for all nonempty
sequences $a$ of transitions from $s$ in $A_G(P_b,c)$ such that
$\forall i \in dom(a), a_i \notin T$ for all $t \in T$ ,

\begin{enumerate}
\item $Pb(S.t) \leq Pb(S.a_1)$
\item if $Pb(S.t)<Pb(S.a_{1}) ,$ then $t \leftrightarrow last(a)$ and $t \leftrightarrow  next(final(S.a), last(a).tid)$
\item if $Pb(S.t)=Pb(S.a_{1}),$ then $ext(s,t) \leftrightarrow last(a)$ and $ext(s,t) \leftrightarrow next(final(S.a), last(a).tid)$
\end{enumerate}

\end{definition}

When dealing with preemption bounded DPOR it is useful to introduce the idea of blocks in an execution sequence.

\begin{definition}{(Block of execution sequence}
    Block in an execution sequence is the maximal subsequence of execution steps that consists of execution steps of the same thread.
\end{definition}

In the Figure \ref{Example of blocks} there are three blocks. The first block is coloured with yellow, the second with green and the third with blue.

\trace{blocks.pdf}{Example of blocks}

Let us assume that $P$ is a persistent set. A preemption bounded persistent set is a set that contains all $p \in P$ with the addition of all the 
threads that would be added in a block that would be created when $p$ was scheduled. These threads are called conservative threads and their 
goal is to allow the coverage of interleavings that would not exceed the bound. Notice that an interleaving can be both conservative and non-conservative.
Preemption bounded persistent sets extend a persistent set by adding all the threads that will create a new block
after the block that will be created by the persistent set.


\chapter{Nidhugg}
\label{Chapter 3}

Nidhugg is a bug-finding tool which targets bugs caused by scheduling non-determinism
and relaxed memory consistency in concurrent programs. It works on the
level of LLVM internal representation, which means that it can be used
for programs written in languages such as C or C++ which can compile to llvm and implement shared-memory concurrency using the pthreads
library.

At the time this thesis was written Nidhugg supported the SC, TSO, PSO and POWER memory
models. On the other hand, Nidhugg does not handle data non-determinism and, thus,
thread should be deterministic when run in isolation.

\section{Source-DPOR - The Nidhugg's algorithm}

The algorithm that Nidhugg uses is shown in Algorithm \ref{Source}.

\SetKwProg{Fn}{Function}{}{}
\SetKwHangingKw{Let}{let}
\begin{algorithm}
    \caption{Source-DPOR}
    \label{Source}
    Explore($\langle \rangle$,$\emptyset$)\;
    \Fn{Explore($E$,$Sleep$)}{
        \If{$\exists p \in (enabled(s_{[E]}) \backslash Sleep)$}{
            backtrack(E) $:={p}$ \;
            \While{$\exists p \in (backtrack(E) \backslash Sleep)$}{
                \ForEach{$e \in dom(E)$ such that $e \lesssim_{E.p} next_{[E]}(p)$}{
                    \Let{$E' = pre(E,e)$}
                    \Let{$u = notdep(e,E).p$}
                    \If{$I_{E'}(u) \cap backtrack(E') = \emptyset$}{
                        add some $q' \in I_{[E']}(u) \text{ to } backtrack(E') $ \;
                    }
                }
                let $Sleep' := \{q \in Sleep \mid E \models p \diamondsuit q \} $ \;
                $Explore(E.p, Sleep')$ \;
                add $p$ to $Sleep$ \;

            }
        }
    }
\end{algorithm}

\noindent Initially an arbitrary enabled and not sleeping
process is chosen and added to the $backtrack(E)$. 

Each step of the algorithm consists of two separate phases. 
During the first phase of the algorithm the race detection takes place.
The algorithm picks a process $p$ which can execute its next step i.e, $p \in enabled(s_{[E]})$ and appends it to the explored trace $E$.
Then the algorithm finds every event $e$ which is already contained in the explored trace ($e \in dom(E)$) and can be reversed with the next step of $p$.

This happens in order to explore the execution sequence where $p$ happens-before $e$.
This execution sequence consists of:
\begin{itemize} 
\item $E' \equiv pre(E,e)$: the subsequence $E'$ which consists of events scheduled before $e$ in $E$.
\item $u \equiv notdep(e,E).p$: the concatination $u$ of all the events that are scheduled after $e$ in $E$ but are independent with $e$ and $p$.
\item $proc(e)$ which is the id of the process that caused the event $e$.
\end{itemize}
Then the algorithm checks whether some process in $I_{[E']}(u)$ is already in $backtrack(E')$. 
If not, then a process in $I_{[E']}(u)$ is added to backtrack. 

In the exploration phase, the exploration starts from $E.p$. The important part is the calculation of the new sleep set at that step since some processes 
may have woken up.
If the next step of a process conflicts with the $next(p)$ then this process must wake up. As a result, the sleep set consists of the already sleeping processes
whose next steps do not interfere with the $next(p)$, i.e, $Sleep' := \{q \in Sleep \mid E \models p \diamondsuit q \} $ 
After finishing the exploration of $E.p$, $p$ is added to the sleep set because we want to refrain from executing an equivalent trace.

\section{Implementation of Nidhugg}

Nidhugg works on the level of LLVM intermediate representation (IR). In order for Nidhugg to find a bug it creates an interpreter for the LLVM assembly.
It then schedules and executes the different traces until an error is found such as the violation of an assertion.
Traces play the most important role in Nidhugg as they represent different schedulings. These traces are represented as vectors of Events objects.
The Event object maintains all the useful information about the event such as the pid of the thread that was executed. Branches which cause the exploration
of different interleavings are also stored in the Event object. The scheduling is regulated by the Tracebuilder object which depends on the memory model used.
Tracebuilder is also responsible for checking for races between different threads that access the same memory.

The execution follows in general the flow that is represented in Figure \ref{Nidhugg's Flow Chart}.
As the flow chart suggests, Nidhugg maintains a TraceBuilder object. The trace builder tries to schedule new events according to the schedule() routine. 
After scheduling, the events
are executed and the vector clocks are updated. Then it is checked whether this event is dependent with other events, i.e., accesses the same
memory locations. After that, Nidhugg tries to add branches to the appropriate places of the branch and checks whether any errors were produced. 
In case of error the procedure stops
and the error is reported. Notice that Nidhugg can be set so it can continue the exploration so more errors can be found. In absence of errors trace builder resets to the most recent branch.
Then the whole trace is executed until that point and the next branch is scheduled. When no more resets are available, the execution terminates.

As far as the algorithm is concerned it is clear that the most important part of the flow chart is the detection of dependencies. 

\mediumGraph{flowchartv2.pdf}{Nidhugg's Flow Chart}

\section{Branch addition by Nidhugg}

Once a command, which is likely to cause a concurrent error, is scheduled the see\_accesses vector is created which contains all the accesses that took place
in the same memory location and calls the procedure \verb|see_events()|.

\begin{algorithm}
    \caption{see\_events()}
    \Fn{$see\_events(seen\_access)$}{
        $branches := seen\_access - \{ a \in seen\_access \mid a  \rightarrow last(E)$ or $\exists a' \in seen\_access$ which happens after $ a \}$ \;
        $update\_clocks()$ \;
        \ForEach{$b \in  branches $}{
            $add\_branch(b)$ \;
        }
     }
\end{algorithm}

As it transparent from the algorithm the function's purpose is to filter out all the access that are not in a race with the current access,
i.e., the dependencies that cannot be represented as a trace with no other concurrent event occurring between them. It is important to notice
that this does not suggest that these events cannot be concurrent in another scheduling. The events that were not discarded as not in race events
are stored in the branch vector and checked by the \verb|add_branch()| function.

Another task of \verb|see_events()| is the update of the vector clocks. Two events that are in race will be concurrent for Nidhugg before the execution
of this routine. At the end of the routine, however, the clocks will be updated so the last event happens after the \verb|seen_access| events.

The \verb|add_branch()| function shown in Algorithm \ref{add_branch} is the most crucial for the whole Nidhugg infrastructure.

\begin{algorithm}
    \caption{add\_branch()}
    \label{add_branch}
    \Fn{add\_branch($b$)}{
        $candidates$ = $\emptyset$ \;
        $lc := null$ \;
        $E' := E \text{ starting from } next(b)$ \; 
        \ForEach{$e \in dom(E')$}{
            \If{$b \rightarrow e$ or $\exists c \in candidates : c \rightarrow e $}{
                continue \;
            }

            $lc := e.pid$\;
            \If{$lc \in candidates$}{
                continue \;
            }

            \If{$e.pid \in backtrack(b)$ or $e.pid \in sleep\_set$}{
                    return \;
            }
            $candidates := candidates \cup lc$ \;
        }
        $backtrack(b) := backtrack(b) \cup lc $ \;
    }
\end{algorithm}

Intuitively, add\_branch does the following: Beginning from the event that conflicts with the most recently scheduled event, start traversing
the vector that represents $E$ until the closest to the end of the trace thread is found (if that is possible). In fact what really happens is the 
calculation of the
$I$ (initials) set. As suggested in the algorithm if there is an already added thread to the branch or the sleep set the procedure will be terminated.

As a result \verb|add_branc()| calculates a source set.

\section{Implementation of the preemption-bound counter}

The first step in order to implement any bounding technique is the implementation of a bound counter. Since we are intrested in implementing
a preemption bounded algorithm we should be able to know the bound count of each event i.e., how many preemptive switches happened until the current 
event. 

The first observation we make is that preemption bound count is a property of each event, thus, apart from any other information the event object should
maintain a counter attribute. Moreover the bound counter should be known to the TraceBuilder as well, since it is responsible for the scheduling.
Such an attribute will prove pretty useful later when preemption-bounded algorithm will be implemented.The second step is to track where new events are added to the trace. There are two occasions when new events
are added. New events are added during the scheduling. Here the implementation of the counter is rather straightforward. 
Taking advantage of an already implemented attribute that indicates the availability of the thread we can store whether the pid of the previous event
corresponds to an available thread and thus, conclude if a preemptive switch happened.
The other occasion when an event is added to the trace is during reset. Unfortunately, the availability attribute is not helpful here since it stores
the latest state of thread and it is a property of the trace not the event. This results usually to all threads being marked unavailable when reset takes place.

There are two options on how to implement a bound counter in such a case. The first is to make thread availability a property of the event, hence, we should store
all the threads' availability in each event. This option was rejected due to the overhead that would occur. The overhead would be caused by both the
memory that would be required and by the fact that this vector should be constantly be copied throughout the DPOR execution. 
The other solution is to infer the availability by the counter itself which as it was mentioned must be maintained in each event. Since the available attribute of
a thread will be reset afterwards we can still use this attribute to store the availability of the threads. 
During the reset the event vector is traversed from the end to the beginning until a branch is found. We assume that each thread is available.
We can make the following observation based on the bound counter of each event.

Given two consecutive events $a,b$ , if $a.bound\_count < b.bound\_count$ then $a$ was available.

The pseudo code is given in Algorithm \ref{Should we increase the bound count?}.

\begin{algorithm}
    \caption{Should we increase the bound count?}
    \label{Should we increase the bound count?}
    \Let{$i =\text{the most recent branching point} $}
    $bound\_count := prefix[i].bound\_count$  \;
    \If{$i>0$}{
        \If{$prefix[i].id == prefix[i-1].id$}{
            $prefix[i].bound\_count = ++bound\_count$\;
        }
        \Else{
            $prefix[i].bound\_count = bound\_count$ \;
        }
    }

\end{algorithm}


In order to be able to verify the correct calculation of the bound count, the debug print during the reset was modified appropriately.
The bound counter should work like Listing \ref{Example of bound counter}.

\Output{./code/bound_count.out}{Example of bound counter}

\section{The Vanilla-BPOR algorithm}


\subsection{Description of Vanilla-BPOR}
The first bounded technique to be implemented is the vanilla-bpor. The purpose of the algorithm is to block threads that exceed the bound
limit. The algorithm is presented at Algorithm \ref{Vanilla}.

\begin{algorithm}
    \caption{Vanilla-BPOR}
    \label{Vanilla}
    \SetKwInOut{Input}{input}
    Explore($\langle \rangle$,$\emptyset$,$b$)\;
    \Fn{Explore($E$,$Sleep$,$b$)}{
        \If{$\exists p \in (enabled(s_{[E]}) \backslash Sleep)$ such that $B_v(E.p) \leq b)$ }{
            $backtrack(E) :={p}$ \;
            \While{$\exists p \in (backtrack(E) \backslash Sleep$ and $B_v(E.p) \leq b$}{
                \ForEach{$e \in dom(E)$ such that $e \lesssim_{E.p} next_{[E]}(p) $}{
                    \Let{$E' = pre(E,e)$}
                    \Let{$u = notdep(e,E).p$}
                    \If{$I_{E'}(u) \cap backtrack(E') = \emptyset$}{
                        add some $q' \in I_{[E']}(u) to backtrack(E') $ \;
                    }
                }
                \Let{$Sleep' := \{q \in Sleep \mid E \models p \diamondsuit q \} $}
                $Explore(E.p, Sleep, b)$ \;
                add $p$ to $Sleep$ \;

            }
        }
    }
\end{algorithm}

The Algorithm \ref{Vanilla} is the almost the same with Source-DPOR(Algorithm \ref{Source}). The only additions made are related to the 
thread scheduling. When a a step of a process $p$ added to $E$ result the trace $E.p : B_v(E.p) > b$ then this process is not allowed to be scheduled.
This algorithm is not sound i.e., it does not examine every trace that compensates with the bound

Lets take for example the writer-2 readers example with $b=0$ show in Figure \ref{Vanilla-BPOR for bound=$0$}. 

\label{Vanilla0}    
\trace{w2rvbound.pdf}{Vanilla-BPOR for bound=$0$}

As we can see there are 4 traces that do not exceed the bound. These are:
$p.q.q.r.r$, $q.q.p.r.r$, $r.r.p.q.q$, $q.q.r.r.p$.
However the vanilla-BPOR is not able to explore them all; $r.r.p.q.q$ is not explored.
As it was shown in the comparison of persistent and source sets, r is never registered as the first event of the trace
since this will lead to a sleep set blocked trace. The branch that would lead to an equivalent trace to $r.r.p.q.q$ is rejected
since it would have higher bound count.

\subsection{Implementation of Vanilla-BPOR}
Nidhugg expects only traces blocked due to sleep sets. Again the first step is to locate parts of the Nidhugg's code where 
bound block should take place. The best option for a bound block to occur is during the \verb|schedule()| function. Before any new scheduling
we just need to determine whether that bound was exceeded or not because of a reset.
Moreover in order for the trace builder to know whether the trace was blocked due to the bound the bound\_blocked flag was added.
Finally modifications should be made in the DPORDriver so it can print correct messages about the reason why the trace was blocked.

Running a random program will result the Listing \ref{Vanilla-BPOR output}.
\Output{./code/vanilla_output.out}{Vanilla-BPOR output}

We notice that Nidhugg gives the number of scheduling that were rejected. Nidhugg schedules threads by giving priority to the older ones. As a result, as 
soon as an old thread becomes available it will be scheduled immediately. This will cause an increase of the bound count since it will probably stop the execution of another thread and maybe if the bound
leading to the increase of the bound count which would cause the to exploration to stop if the bound count was exceeded. In order to explore as many interleavings as possible the priority of the threads was modified.
Specifically, the thread executed most recently has the highest priority. If that thread is unavailable then the priority remains as it 
used to be with the oldest thread being prioritized.

\trace{w2rscheduling.pdf}{Execution without the scheduling optimization}

In the Figure \ref{Execution without the scheduling optimization} an example of two of a program with two readers and a writer is demonstrated.
In the second trace of the example we notice that when the read operation of $q$ takes place $p$ wakes up since $r(x)$ conflicts with $w(x)$. Since $p$
is older than $q$, the execution of $q$ stops and $p$ is scheduled. Since $q$ was enabled when $p$ was scheduled the bound counter is increased. However,
if $q$ was scheduled again instead of $p$ then the $q$ would be blocked after the return command and the scheduling of $p$ would not have increased the 
bound count. We can infer that, had the most recently running thread given the highest priority at least one more interleaving would have
been explored and as a result more interleavings would have been explored.

\section{The DPOR using persistent sets}

The implementation of persistent sets proved to be one of the most challenging tasks in Nidhugg. As it will become clear later, such an implementation
of persistent sets is a prerequisite for BPOR's soundness. Many options were considered. The definition of persistent sets implies one 
way to implement them i.e., for every execution step we should
check all the other threads and add them to the branch if they contain a command which conflicts with the execution step. This kind of approach
completely contradicts with the whole philosophy of Nidhugg which stems from the nature of source sets and, thus, would not be feasible. 
The option that was finally chosen was an implementation based on the DPOR using Vector Clocks.

\subsection{Description of DPOR using persistent sets}

Since Nidhugg uses vector clocks to track events the DPOR using Clock Vectors variation will be used which is shown in Algorithm \ref{DPORV}

\begin{algorithm}
    \caption{DPOR using Clock Vectors}
    \label{DPORV}
    \SetKwInOut{Input}{input}
    \SetKwInput{Initialization}{Explore($\emptyset, \lambda x. \bot$)}
    \SetKwHangingKw{Let}{let}
    \Fn{Explore($E$,$C$)}{
        \Let{$s := last(S)$}
        \For{all process $p$}{
            \If{$\exists i = max(\{ i \in dom(S) \mid S_i$ is dependent and may be co-enabled with $next(s,p)$ and $i \not \leq  C(p)(proc(S_i)) \} $}{
                \uIf{$p \in enabled(pre(S,i)))$}{
                    add $p$ to $backtrack(pre(S,i))$ \;
                }
                \Else{add $enabled(pre(S,i))$ to $backtrack(pre(S,i))$ \;}
            }
        }
        \If{$\exists p \in enabled(s)$}{
            $backtrack(s) := {p}$ \;
            \Let{$done = \emptyset$}
            \While{$\exists p \in (backtrack(s) \backslash done)$}{
               add $p$ to $done$ \;
               \Let{$t = next(s,p)$} 
                \Let{$S' = S.t$} 
                \Let{$cu = max \{ C(i) \mid i \in 1.. | S |$ and $S_i$ dependent with $t \}$ 
                \Let{$cu2 = cu [ p := | S' | ] $} 
               \Let{$C' = C [p:= cu2, |S'| := cu2 ]$} 
                $Explore(S',C')$ \;
            }
        }
    }
\end{algorithm}

In order to take advantage of Nidhugg's infrastructure Algorithm \ref{NBPOR} will be used. As a result there is no need to add all
available threads when $p$ is not enabled. If no sufficient candidate is found
a candidate suggested by the Source-DPOR algorithm is added. This way of calculating persistent sets is considered to be more 
complex and thus a bad option \cite{Gode05}. However, source sets algorithm is closer to this approach.

\begin{algorithm}
    \caption{Nidhugg DPOR}
    \label{NDPOR}
    \SetKwInOut{Input}{input}
    \SetKwHangingKw{Let}{let}
    Explore($\langle \rangle$,$\emptyset$)\;
    \Fn{Explore($E$,$Sleep$)}{
        \If{$\exists p \in (enabled(s_{[E]}) \backslash Sleep$}{
            backtrack(E) $:={p}$ \;
            \While{$\exists p \in (backtrack(E) \backslash Sleep)$}{
                \ForEach{$e \in dom(E)$ such that $e \lesssim_{E.p} next_{[E]}(p)$}{
                    \Let{$E' = pre(E,e)$}
                    \Let{$u = notdep(e,E).p$}
                    \Let{$CI = \{ i \in I_{E'}(u) \mid i \rightarrow p \}$}
                    \If{$CI \cap backtrack(E') = \emptyset$}{
                        \If{$CI \neq \emptyset$}{
                            add some $q' \in CI \text{ to } backtrack(E') $ \;
                        }
                        \Else{add some $q' I_{E'}(u) \text{ to } backtrack(E')$}
                    }
                }
                \Let{ $Sleep' := \{q \in Sleep \mid E \models p \diamondsuit q \} $ } 
                $Explore(E.p, Sleep)$ \;
                add $p$ to $Sleep$ \;
            }
        }
    }
\end{algorithm}

The implemented Algorithm \ref{NDPOR} differs from Source-DPOR (Alg. \ref{Source}) in the calculation of the initials.
It is clear that persistent algorithm implemented differentiates from Source-DPOR in the calculation of the initials. 
Specifically a subset of the initials that happen before $p$ is used. 
Intuitively in the case of a writer and two readers, both readers will be added to the branch since the first read does not
happen before the second one.
To generalize this idea: since Nidhugg does not enable us to create branches for $last(E)$ when it is scheduled we add the 
branches later as in Source-DPOR (Alg. \ref{Source}). When a race is considered usually only the thread that causes the race will be added since 
$CI$ contains this thread only.

We will prove that DPOR calculates a persistent set or that when the algorithm finishes a persistent set will have been calculated in each step.

Let us assume two processes that are in race with $last(S)$.
\begin{itemize}

\item Case 1: at least one of them is a write process.
We know that the Nidhugg's DPOR should calculate a superset of the Source DPOR branches, thus, we know that the read and the write processes at some point
will be inverted. Moreover we will the $CI$ set will consider both ignoring padding (see Figure \ref{Construction of persistent sets in Nidhugg}.As a result both processes will be considered and will be added to the persistent set.

\trace{nidhuggpersistent.pdf}{Construction of persistent sets in Nidhugg}

\item Case 2: both processes are read operations.
Since we do not calculate $I$ but $CI$ the first read operation will not be considered as it does not happen before the second read operation and as result
both processes will be added to $backtrack$. 

\end{itemize}
It is clear that no process that does not belong to the backtrack(S) has race with a process that belongs to backtrack(S).

\subsection{Implementation of DPOR using persistent-sets}

The implementation of the persistent sets is based on the already implemented infrastructure of vector clocks. Specifically the \verb|include()| function of vector clocks
lets as determine whether $i \rightarrow p$.

To calculate the $CI$ set we just need to prevent the \verb|add\_branch()| function from rejecting branches due to threads that belong to $I$ but not to $CI$.
An example of the modification is presented at Algorithm \ref{addbranch routine for persistent sets}.

\begin{algorithm}
    \caption{add\_branch() routine for persistent sets}
    \label{addbranch routine for persistent sets}
    \Fn{add\_branch($b$)}{
        $candidates = \emptyset$ \;
        $lc = null$
        $E'$ := $E text{starting from} next(b)$ \; 
        \ForEach{$e in dom(E')$}{
            \If{$b \rightarrow e \text{ or } \exists c \in candidates \mid c \rightarrow e $}{
                continue \;
            }

            \If{$e \rightarrow last(E)$}{
                $lpc := e.pid$\;
            }

            $lc := e.pid$\;

            \If{$e.pid \in backtrack(b) or e.pid \in sleep\_set(b)$}{
                \If{$lc \rightarrow last(E)$}{
                    return \;
                }
            }
        }
        \If{$lpc$}{
            $backtrack(b) := backtrack(b) \cup lpc $
        }
        \Else{$backtrack(b) := backtrack(b) \cup lc $}
    }
\end{algorithm}

In case that E is empty we can just use a candidate that it is suggested from $I$ set.

\section{The BPOR}
Having implemented persistent sets correctly the next task is the implementation of a BPOR algorithm. The novelty of the BPOR is the introduction of conservative branches. These
are branches that are introduced in order to guarantee the exploration of the whole state space. It is common for a trace to exceed the bound limit whereas there is
an equivalent trace which does not. The conservative branches are used for this purpose.

\begin{definition}{(Trace block)}
For a trace $T$ a sequence $B$ of consecutive events is a trace block iff all events happen in the same thread i.e. all the events have the same thread id.
\end{definition}

The idea behind conservative branches is quit simple. When a branch is added a conservative branch is added at the beginning of the corresponding block.
Usually concurrent events take place inside a block. As a result when a branch is taken then the preemption count will most probably increased. However had this
branch been added at the beginning of the block the preemption count would not have been increased. 

\subsection{Description of BPOR}
The algorithm implemented is presented here \cite{BPOR} in detail. A modification of this algorithm is used in order to take advantage of the Nidhugg's infrastructure.
The algorithm is presented in Algorithm \ref{Nidhugg BPOR}.


\begin{algorithm}
    \caption{Nidhugg BPOR}
    \label{Nidhugg BPOR}
    \SetKwInOut{Input}{input}
    \SetKwHangingKw{Let}{let}
    Explore($\langle \rangle$,$\emptyset$,$b$)\;
    \Fn{Explore($E$,$Sleep$,$b$)}{
        \If{$\exists p \in ((enabled(s_{[E]}) \backslash Sleep)$ and $B_v(E.p) <=b$}{
            backtrack(E) $:={p}$ \;
            \While{$\exists p \in (backtrack(E) \backslash Sleep $ and $B_v(E.p) <=b$}{
                \ForEach{$e \in dom(E)$ such that $e \lesssim_{E.p} next_{[E]}(p)$}{
                    \Let{$E' = pre(E,e)$}
                    \Let{$u = notdep(e,E).p$}
                    \Let{$CI = \{ i \in I_{E'}(u) \mid i \rightarrow p \}$}
                    \If{$CI \cap backtrack(E') = \emptyset$}{
                        \If{$CI \neq \emptyset$}{
                            add some $q' \in CI$ to $backtrack(E') $ \;
                        }
                        \Else{
                             add some $q' \in I_{[E']}(u)$ to $backtrack(E') $ \;}
                        }
                    \Let{$E''= pre\_block(e,E)$}
                    \Let{$u = notdep(e,E).p$}
                    \Let{$CI = \{ i \in I_{E''}(u) \mid i \rightarrow p \}$}
                    \If{$CI \cap backtrack(E') = \emptyset$}{
                        \If{$CI \neq \emptyset$}{
                            add some $q' \in CI$ to $backtrack(E') $ \;
                        }
                        \Else{
                            add some $c(q') \in I_{[E'']}(u)$ to $backtrack(E'') $ \;
                        }
                    } 
                }
                \Let{ $Sleep' := \{q \in Sleep \mid E \models p \diamondsuit q \}$ } 
                $Explore(E.p, Sleep)$ \;
                \If{$p$ is not conservative}{
                    add $p$ to $Sleep$ \;
                }
            }
        }
    }
\end{algorithm}

A critical challenge arises when a DPOR algorithm is used in tandem with sleep sets. This stems from the fact that conservative branches are not added due to a
concurrent event. By observing the sleep set algorithm we notice that if we follow the same strategy as with non-conservative branches many traces will end up being blocked.

Let us take the writer-2readers example as shown in Figure \ref{Usage of non-conservative branches}.

\trace{w2rpersistent.pdf}{Usage of non-conservative branches}

We notice that the last trace is sleep set blocked while it should be examined. The algorithm is unaware that the thread r should be removed from the sleep set since there is no
or will ever be found any conflict with the first command of the thread which is related with a non shared variable. In order to deal with this problem when a conservative
branch is chosen then it should not be added to the sleep set. However there must be a set recording all the branches that where added at this certain point of the trace
so no thread is added twice. The solution is based on the notion of the conservative sets where every thread that was added to the branch is recorded. 

Intuitively the algorithm is the same with the Source-DPOR with the addition of the conservative branches. The solution is based on the notion of the conservative sets where every thread that was added to the branch is recorded.  However many challenges araise which are discussed 
in the implementation section.

\tracelong{w2rbpor.pdf}{Example of BPOR execution}

\subsection{Implementation of BPOR}
As it was made clear in previous sections for any algorithm to be implemented, the main modifications should take place in the see\_events and add\_branch procedures. 
The pseudo code for the see\_events procedure is demonstrated at Algorithm \ref{seeevents routine for BPOR}.

During see\_events procedure we add another branch at the beginning of the block if that is possible. In order to do that we have to check whether the added thread
is available or not in that place. We use the available thread field for this purpose.

\begin{algorithm}[H]
    \caption{see\_events() for BPOR}
    \label{seeevents routine for BPOR}
    Explore($\langle \rangle$,$\emptyset$)\;
    \Fn{see\_events($seen\_access$)}{
        $vector branches = \emptyset$ \;
        $branches := seen\_access - \{ a \in seen\_access \mid a$ happens before $last(E) or \exists a' \in seen\_access$ which happens after $a \}$ \;
        $update\_clocks()$ \;
        \ForEach{$b \in branches$}{
            $add\_branch(b)$ \;
            \If{$b \in enabled(last(E))$}{
                $add\_branch($ at the beginning of block $b)$ \;
            }
        }
     }
\end{algorithm}

In case the \verb|add_branch()| invokes directly then the \verb|add_conservative_branch()| is called which works as the "conservative"
part of the \verb|see_events()|. The pseudo code for this procedure is given here:

During \verb|add_branch()| procedure we add branches at the appropriate places using the algorithm for the persistent sets suggested in the previous section.
It was made clear that two different types of branches are used. However, the Nidhugg's infrastructure takes into account only the non-conservative ones when it comes
to searching for threads in set of threads such as sleep sets. As a result, at some points of the code we have to look for both the conservative and the non conservative
branches in the set. 
Another important problem arises when both conservative and non conservative branches are added at the same point. In that case the conservative branch prevails.
Looking at the writer-2readers example if we have chosen the non-conservative branch then the trace that begins with the r2 would have been blocked by the sleep sets.

\section{The Source-BPOR algorithm}
Having implemented a BPOR algorithm the next step is to try combine source sets with the algorithm. The first observation we have to make is that
source sets and thus the algorithm for creating these sets is not suitable for adding conservative branches. A quick explanation is given in the next writer-2readers example
even though the problem will be further discussed later. Let us assume that we have followed the source set algorithm for adding conservative sets. 
The results are shown at Figure \ref{Following source sets for conservative branches}.

\trace{w2rsourceconservative.pdf}{Following source sets for conservative branches}

It is clear that some traces are not explored. Specifically, the trace which start with r has been rejected. The reason is that it shares the same initials with r1 even at the
beginning of that block. As a result the algorithm must create to persistent sets when conservative threads are added. 

Having made the preceding observations the algorithm used for Source-BPOR is the following:

\subsection{Description of Source-BPOR}

We can notice that the algorithm is equivalent to the source-DPOR for non-conservative traces and equivalent to BPOR for conservative traces.

\subsection{Implementation of Source-BPOR}
The implementation is again based on modifications on the procedure \verb|add_branch()| since every change in the \verb|see_events()| procedure is still necessary for this algorithm.
We can infer from the algorithm that will choose the same candidate with the Source-DPOR when we are dealing with a non-conservative branch and the same candidate as in BPOR
in case the branch is conservative. In order to differentiate for the two cases we add a second parameter at the routine so we can be aware of the nature of the
branch (conservative or non-conservative). The pseudo code for \verb|add_branch()| is given at Algorithm \ref{addbranch routine for Source-BPOR}.

\begin{algorithm}[H]
    \caption{add\_branch() routine for Source-BPOR}
    \label{addbranch routine for Source-BPOR}
    \Fn{add\_branch($b,is\_conservative$)}{
        $candidates = \emptyset$ \;
        $lc = null$ \;
        $lpc = null$ \;
        $E'$ := $E$ starting from $next_{E}(b)$ \; 
        \ForEach{$e \in dom(E')$}{
            \If{$b \rightarrow e$ or $\exists c \in candidates \mid c \rightarrow e $}{
                continue \;
            }

            \If{$e \rightarrow last(E)$}{
                $lpc := e.pid$\;
            }

            $lc := e.pid$\;

            \If{$e.pid \in backtrack(b)$ or $e.pid \in sleep\_set(b)$}{
                \If{not is\_conservative or $ lc \rightarrow last(E)$}{
                    return \;
                }
            }
        }
        \If{$lpc$ and $is\_conservative$}{
            $backtrack(b) := backtrack(b) \cup lpc $
        }
        \Else{$backtrack(b) := backtrack(b) \cup lc $}
    }
\end{algorithm}

\section{Modifications in test suite}
For any implementation to be verified the test suit already available with Nidhugg was used. However in the test suit there are many limitations related to the source-DPOR
that do not hold true in the BPOR and Source-DPOR. For example the test suit driver would report equivalent traces as errors even though that these traces cannot be eliminated 
when bounded DPOR takes place. The reasons of this behavior have already been explained. In the section the changes on the test driver are reported.
The modification took place was rather straightforward since we just had to mute warnings when the number of traces exceeded the anticipated or equivalent traces were explored more than
once. However, in two cases (Atomic\_9,Intrinsic\_2) the only check that takes place concerns the number of the traces. In these cases only the test suit will report an error. 
The report of the test suit when bounded DPOR is executed is shown below.


\chapter{Evaluation of Bounding Techniques}
\label{Chapter 4}

In this chapter, the performance of each implemented technique will be discussed. Firstly, the performance of classic-DPOR is demonstrated in order to prove
its performance, indeed differs from Source-DPOR. The evaluation happens in two parts. In the first part, short synthetic programs are used, while in the second part
real world software is tested. Synthetic programs can be found in the appendix section. One area where Nidhugg is tested is the verification of the Read Copy Update
technique of the Linux kernel.


\section{Synthetic Tests}
There are many tests provided from various sources. Most of these testcases are not complicated at all since their purpose is to demonstrate the performance
difference of the Source-DPOR and classic-DPOR.

\begin{itemize}
\item The writer-Nreaders test: In this test N threads read (readers) the same global variable and one thread (writer) writes that variable. It is important 
to notice that in this case there are some other local operations taking place before the read of the variable. As a result we must expect different results between 
source sets and persistent sets.

\item Account: This test is a small bank account simulation which uses mutex locks to prevent simultaneous operations on the account.
There are three possible operations: The deposit operation increases the balance by an amount. The withdraw operation decreases the balance by a certain amount. The check\_result operation confirms 
$\text{final\_balance} == \text{initial\_balance} + \text{deposit} - \text{withdraw}$ and can only happen after both deposit and withdraw are completed.

\item Micro: In this test three threads are spawned that perform the \verb|x++| operation twice. The \verb|x++| operation
consists of two operations a read operation and a write operation.

\item Last-zero test:The program consists of N+1 threads which operate on an array of
N+1 elements which are all initially zero. In this program, thread 0
searches the array for the zero element with the highest index, while
the other N threads read one of the array elements and update the
next one. The final state of the program is uniquely defined by the
values of i and array[1..N]. Last-zero does not produce more traces when DPOR is used for reasons
that will be explained later. However a modification of the .ll file can expose the difference.

\item Indexer.c: This benchmark uses a compare-and-swap(CAS) primitive instruction to check
whether a specific entry in a matrix is 0 and set it to a new
value. 

\item Indexermod.c: In this benchmark all the threads traverse and try to write the matrix at the same
order and as a result many conflicts emerge.


\end{itemize}

\section{RCU}

Read-Copy-Update (RCU) is a synchronization mechanism, which was invented by McKenney and Slingwine \cite{McKenney98}, based on mutual exclusion.
It was added to the Linux kernel in October of 2002. RCU achieves scalability improvements by allowing reads to occur concurrently 
with updates. In contrast with conventional locking primitives that ensure mutual exclusion among concurrent threads regardless of whether 
they be readers or updaters, or with reader-writer locks that allow concurrent reads but not in the presence of updates, 
RCU supports concurrency between a single updater and multiple readers. 

DPOR was used as an approach to systematically test the code
of the main flavor of RCU used in the Linux kernel (Tree RCU) for
concurrency errors, under sequential consistency. 
The modeling allows Nidhugg, a stateless model checking
tool, to reproduce, within seconds, safety and liveness bugs that
have been reported for RCU \cite{Spin}.

RCU provides an ideal testcase to evaluate the various DPOR and Bounded DPOR algorithms since it is:
\begin{itemize}
\item It is a real world software and not just a synthetic test.
\item The number of traces (different schedulings) is large enough to demonstrate the differences in the performance.
\item Previous work \cite{Spin} enables us to evaluate the correctness of each algorithm's implementation.
\end{itemize}


\section{Evaluation of Persistent Sets}
As it was established in the previous chapter the implementation of the persistent sets
is crucial since they are utilized in every bounding technique. 
In this section we demonstrate performance differences between Source-DPOR and classic-DPOR in both synthetic tests and RCU.

\subsection{Evaluation of Persistent sets on Synthetic tests}
The execution of the synthetic test cases delineated that Source-DPOR is indeed an improvement over Classic-DPOR. As it was expected source-DPOR explores less 
traces than the DPOR. It is important
to notice that this difference is caused by the sleep set blocked traces that are produced by the DPOR algorithm that are omitted by the source DPOR. 
The reduction in the number of traces explored is not the same of all the testcases. For example in some testcase there is a
significant decrease of the explored traces while in others the reduction is not so great. 
It varies due to the different approaches as well as the size of the state space. The results are presented with two different ways. 
The writer-N-readers testcase result is given with a graph, in Figure \ref{writer-N-readers}, in order to demonstrate the escalation of the state space as well as the greater impact the source-DPOR has. The rest of the
results are given in Table \ref{Source-DPOR vs DPOR for synthetic tests} so they can be easily compared.

\graph{/home/yannis/nidhugg/tests/mytests/wNr.png}{writer-N-readers}

\smalltabular{"tables/synthetic_unbounded.tex"}{Source-DPOR vs Classic-DPOR for synthetic tests}

\subsection{Evaluation of Persistent sets on RCU}
We noticed that there is no difference between Source sets and persistent sets thus no results are presented since they coincide with \cite{Spin}. 
The reason why the results of DPOR and Source-DPOR are the same may be due to the operations that take place which not allow for the optimization of the Source-DPOR 
to be effective. Another reason is the LLVM IR which is used by Nidhugg and will be explained in the next section.

\section{Comparison with Concuerror results}
The are many cases where Concuerror's Source-DPOR explores less traces than Classic-DPOR while Nidhugg's Source-DPOR doesn't. 
In fact Nidhugg's Classic-DPOR implementation seems to explore less number of traces than Concuerror's Classic-DPOR \cite{AbdullaAronisJohnssonSagonasDPOR2014}.
which equals with the numbers of traces explored by Source-DPOR. There are many reasons that justify this behavior.

\begin{itemize}
  \item The implementation of the persistent set calculation. In Concuerror this calclation is more relaxed than the in Nidhugg as a result 
  Nidhugg calculates sometimes smaller persistent sets. In Figure \ref{Lastzero Concuerror} the example of lastzero testcase in Concuerror is given when a 
  larger persistent set is calculated. In fact, the process $q$ should never have been added to the persistent set since it does not conflict with any other
  process.

  \item The number of traces explored is closely related by scheduling of the events. Let a program consist of a process which writes on 
  variable $x$ and two process that read variable $x$. In Figure \ref{Scheduling Effect reader-writer-reader} the exploration is demonstrated when
  one reader is scheduled first. We notice that exactly 4 traces are explored. On the other hand, provided that the writer was scheduled first 
  \ref{Scheduling Effect writer-reader-reader} 5 traces with one sleep set blocked traces would have been explored.

\end{itemize}

\tracelong{lastzero.pdf}{Lastzero Concuerror}

\trace{schedulingrwr.pdf}{Scheduling Effect reader-writer-reader}
\trace{schedulingwrr.pdf}{Scheduling Effect writer-reader-reader}


\section{Evaluation of Bounding Techniques}
The evaluation of the techniques takes into account two aspects. The number of traces explored and the soundness. The former is closely related with the amount
of time required for a bug to be found or the whole state space to be explored. The second is important because it demonstrates the tradeoff between time and accuracy
of the results. It is intelligible that a faster algorithm may compromise the soundness of the state space.
\subsection{Evaluation of Bounding Techniques on Synthetic tests}

The results for the testcases are demonstrated below. Again they are presented in two different ways.

%%\begin{center}
%%    \begin{tabular}{ |c|c|c|c|c|c|c|}
%%    \hline
%%    \multicolumn{1}{|c|}{Technique:} & \multicolumn{2}{c|}{Vanilla-BPOR} & \multicolumn{2}{c|}{BPOR} & \multicolumn{2}{c|}{Source-BPOR} \\
%%    \hline
%%    Bound: & 0 & 1 & 0 & 1 & 0 & 1 \\
%%    \hline \hline
%%    account.c & 1 & 6 & 6 & 1 & 27 & 27 \\
%%    \hline
%%    lazy.c & 1 & 6 & 6 & 1 & 27 & 27 \\
%%    \hline
%%    \end{tabular}
%%\end{center}

\graph{img/wNrB.png}{writer-N-readers bounded}
\smalltabular{"/home/yannis/nidhugg/tests/mytests/bounded.tex"}{Traces for various bound limits}

As it was expected the Naive-BPOR explores significantly less traces than the BPOR and the source-DPOR. However, as it was previously discussed, the whole
state space is not explored. The number of traces explored by the sound algorithms is significantly greater and it caused by the many conservative branches that are
added in order to achieve soundness. Surprisingly, there is no difference between the other two bounding techniques. An explanation is given later.

\subsection{Evaluation of Bounding Techniques on RCU}
The results are demonstrated below. Notice that since the Source-DPOR did not resulted less traces than the DPOR we could not expect from the Source-BPOR and BPOR
to differentiate. Moreover tests did not show any difference. For these reasons only the performance of Naive-BPOR and BPOR is examined. In each table
the results with a given bound are demonstrated. Specifically the exploration time and the number of traces are shown. 
Moreover there is a cell indicating whether the assertion was found (We note F for found and NF for not found).


\bigtabular{"/home/yannis/rcu/valtree/nobound.tex"}{RCU results without bound}
\bigtabular{"tables/naivevsbpor0.tex"}{RCU results for bound $b=0$}
\bigtabular{"tables/naivevsbpor1.tex"}{RCU results for bound $b=1$}
\bigtabular{"tables/naivevsbpor2.tex"}{RCU results for bound $b=2$}
\bigtabular{"tables/naivevsbpor3.tex"}{RCU results for bound $b=3$}
\bigtabular{"tables/naivevsbpor4.tex"}{RCU results for bound $b=4$}
\bigtabular{"tables/dporvsbpor.tex"}{Comparison between DPOR and BPOR}


We notice that some assertions are found significantly faster. The most spectacular result is the \verb|-DFORCE_FAILURE_3| which is found in only 6 seconds for bound b=3 whereas it requires 464.77 seconds in the unbounded version. Moreover we notice for bound b=4 all the errors that are found in the unbounded version are
found. As a result the empirical observation that errors occur in low bound count seems to be confirmed. However, we have to underline that these are contrived
errors aiming to verify the correctness of the rcu and as a result they cannot be regarded as substantial evidences. As it is expected for larger bounds (b=4) the number of traces
grows exponentially. An other impressive result is that when the bound grows larger the errors takes longer to be found. If we take a look at \verb|-DFORCE_FAILURE_3| again we notice that the error
takes significantly longer to be tracked even through it exposed for the first time at bound b=2. For b=4 the exploration will was stopped since it exceeded 100,000 traces.
On the other hand many assertions are found faster with source-DPOR.

\subsection{A known bug}
As it was discussed in previous section, the scheduling priorities of Nidhugg should be changed in order for the running thread to be prioritize since it does not
increase the bound count. However this alternation in the priority causes Nidhugg to explore many more traces in unbounded search for an unknown reason. In order to deal with
this problem alternation in priority occurs only when bound is applied. As a result the comparison between DPOR and BPOR is not fair. Looking at table \ref{Comparison between DPOR and BPOR with the bug}
we can clearly see that the minimum traces required for BPOR to track the bug for the first time are always less than DPOR

\bigtabular{"tables/dporvsbporpriority.tex"}{Comparison between DPOR and BPOR with the bug}

\section{Equivalence between Classic-BPOR and Source-BPOR (Correctness of Source-BPOR)}
Surprisingly the results of Classic-BPOR and Source-BPOR always coincide. However, further investigation of this behavior can reveal that these two techniques
are actually equivalent. 

It can be proved that a branch which was rejected by the Source-DPOR but accepted by the Classic-BPOR algorithm as a non-conservative one will be added as 
conservative by the source-bpor algorithm.

Let us assume a branch of the thread $b$ that is added as a non-conservative by the BPOR algorithm. Let $T$ be the persistent set at that point.

By the definition of persistent-sets this means that there is a $t \in T$ which
conflicts with an execution step of $b$. 

This non-conservative branch is rejected by the Source-BPOR. We know that there must be
 a trace such that thread $b$ occurs before $t$. Since $b$ was rejected there must be another branch $s$ which shares the same initials
  with $b$, 
  
When $s$ is scheduled another block will be created.
 
\begin{itemize}

\item Case 1: $s$ has an execution step which conflicts with $b$. Hence, there must be a trace where $b$ happens before some step of $s$ and $s$ 
happens before $t$. Since $b$ happens-before some execution step of $s$ but share the same initials with $s$, $b$ must be added as a conservative
branch at the point where it was rejected at the initially.
As shown in the Figure \ref{Source-BPOR and Classic-BPOR equivalence Case 1}, the branch which seems to be initially rejected, 
will finally be added by the Source-BPOR and as a result belongs to the source-set. 
\trace{equivalence_case1w.pdf}{Source-BPOR and BPOR equivalence Case 1}
   
\item Case 2: 
   $s$ doesn’t conflict with $b$ (both $b$ and $s$ are read operations). There must $b$ a trace s.b.t (where s,b,t is the execution
   of all the steps of s,b,t). 
   Since $t$ conflicts with an execution step of $s$ the first step of $b$ is an initial for $t$ and it will be added both as non-conservative branch and 
   as conservative at the beginning of the block where it was rejected by the Source-DPOR. For Figure \ref{Source-BPOR and Classic-BPOR equivalence Case 2}, both $s$ and $b$ belong to the persistent set. However,
   the $b$ thread will be rejected since it shares the same initials with the $s$ thread. However it will be added as a conservative set. Notice that it would be added as a
   non-conservative as well but we have already shown that when both conservative and non-conservative branches of the same thread are added we must keep the conservative one.

   \trace{equivalence_case2.pdf}{Source-BPOR and BPOR equivalence Case 2}
\end{itemize}
   
A more intuitive explanation of the equivalence of the two techniques would be based on the following two observation:
\begin{itemize}
  \item Let $B_v$ be a function that calculates the bound count then $B_v(pre(E,e)) \leq B_v(E)$ for every $e \in E$.
  \item The points in the trace where the bound count increases are those where branches are added.
\end{itemize}

As a result a Classic-BPOR algorithm would have to add conservative branches at points where the bound increases. The non-conservative branches that would have been
rejected by the Source-DPOR are added as conservative ones to since the lead to already explored traces but with a smaller bound count.

We have proved that Source-BPOR is sound since the traces explored by the Classic-BPOR are subset of the traces explored by the Source-BPOR.

\chapter{Further Discussion on Bounding Problem}
\label{Chapter 5}

In this chapter we discuss alternative ideas of approaching the preemption bounding problem of the DPOR. 
We suggest a new approach which is shown to be equivalent to the addition of conservative branches but does not require the addition of conservative
branches.

\section{Techniques without the Addition of Conservative Branches}

In previous chapters we discussed the challenges one have to deal with when designing a Bounded DPOR algorithm. We saw that there there is 
no apparent significant improvement can be made with the use of conservative branches and that other optimizations used for the unbouded versions of DPOR
cannot have analogous results. 

\subsection{Motivation}
The only algorithm that does not add any conservative branch is the Naive-BPOR. For a sufficient bound an erroneous
trace would have still be found using this technique. The drawback of this algorithm is its unsoundness. In this
algorithm a function which calculates the number of preemptive switches in the current thread is used. However many of
the preemptive switches that are counted would be avoided.

An example is given in Figure \ref{An example of avoidable preemption-switch} further explaining this idea. Let
a program consisting of two processes $p$ and $q$. Process $p$ writes a shared variable $x$ and process $q$ reads
the shared variable $y$ (which is not modified at any point by any other process), and the shared variable $x$. There
are only two possible schedulings i.e. one that the writing of $x$ precedes the reading and one that these two commands
are reversed. However, in order to make the reversion a preemptive switch must be introduced. Suppose that the first
execution sequence is $q.q.p$, the other will be $q.p.q$ since the first step of $q$ has no race with step of $p$.

\trace{motivation.pdf}{An example of avoidable preemption-switch}

In Figure \ref{An example of avoidable preemption-switch} the preemptive switch that takes places would have been easily
avoided there is an obvious inversion of the two blocks $q:read(y)$ and $p$. But what allows such an inversion?

The answer lies to the events of block (in this particular case the block consists of one step only). The first block
reads a variable which is not used by any other block. It is fathomable that this block can be switched with the next
block since there is no a happens before relationship with the two blocks.

This observation leads to the next question: Which of the preemption switches are compulsory? (Or equivalently, which
traces cannot be produced without a preemption switch?) Moreover, is it possible for a given trace to calculate the
minimum number of preemptive switches among all the equivalent traces?

\subsection{An Algorithm without Conservative Branches}
An algorithm that would preform such a bounded search would be different from the Naive-BPOR only concerning the
function that calculates the bound count of the traces. This function $f$ would have to be constantly ascending i.e. it
would not be possible to calculate a lower bound later for the same traces. Given a prefix $E$, $f(E) \leq f(E.E')$ for
any $E'$.

The general form of the algorithm is given in Algorithm \ref{NBBPOR}.

\begin{algorithm}[H]
    \SetAlgoLined
    \caption{General form of the BPOR without branch addition}
    \label{NBBPOR}
    \KwResult{Explore the whole state space within the bound}
    Explore($\emptyset$)\;
    \Fn{Explore($S$)}{
        T = Sufficient\_set($final(S)$)
     \For{all $t \in T$}{
         \If{$min\{B_v([S.t])\} \leq c$}{
            Explore($S.t$)
         }
        }
    }
\end{algorithm}

Comparing Algorithm \ref{NBBPOR} with Algorithm \ref{GeneralDPOR} we notice that instead of calculating the $B_v(S.t)$ i.e., the
bound count of the current trace value, we calculate the $min(B_v[S.t])$ i.e., minimum of all $B_v$ values of the traces
that are equivalent with $S.t$.

\subsection{Calculating Minimum Bound Count}
The only thing left is the construction of this function $f$. For a given trace $E$ which consists of blocks many
happens-before relations hold. Each equivalent trace should also compensate with these relations. It is also possible
for different instructions in one block different happens-before relations hold true. For this section only, we will
consider that a happens-before relation is a relation that holds between blocks. This is done for two main reasons:

\begin{itemize}
    \item The algorithm described later is simplified.
    \item We are not interested in further breaking each block and as a result we can regard is block as an entity.
\end{itemize}

These happens-before relations form a graph. This graph consists of nodes which are the blocks and edges which are these
relations. Obviously blocks of the same thread have a happens-before relation. We can also move from one block to
another as long as these blocks happen concurrently. We add weights to each edge. The edges that connect to blocks of
the same thread weigh 0. Edges that start from a block that is blocked or the most recently added block of each thread
weigh 0 since blocked blocks do not increase the bound count and we do not know if the last block of each thread is
indeed the last one. All the other edges which represent preemption points have weight 1.

In order to find the minimum bound count we have to traverse this all the blocks of this graph without breaking at any
point the happens-before relations. Hence, the minimum bound count corresponds to the minimum hamiltonian path that
compensates with all happens-before relations of the trace that is explored.

Since the calculation of the minimum hamiltonian path is demanding we have to create a graph that limits as much as
possible traverses that break the happens-before relations. All traversals that cover the whole graph passing from each
node only once are equivalent traces. 

\noindent An algorithm on how to add a block to a given graph is given at Algorithm \ref{Adding a new block to the
dependencies graph}. The algorithm works using induction. Initially the graph consists of the first block. When a block
of the trace is completed then we add it to the dependency graph. We connect the new block with each concurrent block
with double edges with the new block. Moreover we connect the most recent block of each thread that happens before the
new block with a directed edge ending to the new block. 

\SetKw{Return}{return \;}
\SetKw{Break}{break \;}
\begin{algorithm}[H]
    \caption{Adding a new block to the dependencies' graph}
    \label{Adding a new block to the dependencies graph}
    \Fn{AddBlock(block,graph)}{
        \If{previous block of the same thread was not blocked}{
            increase the weigh of the edges coming from the previous block to 1 \;
        }

        \For{each thread t}{
            list:= preceding blocks t\;
            \For{l in reversed(list)}{
                \If{$l \leftrightarrow block$}{
                    add edge from $block$ to $l$ with weight $0$ \;
                    \If{$l$ is not last}{
                        add edge from $l$ to $block$ with weight $1$ \;
                    }
                    \Else{
                        add edge from $l$ to $block$ with weight $0$ \; 
                    }
                }
                \If{$l \rightarrow block$}{
                    \If{$l$ is not last}{
                        add edge from $l$ to $block$ with weight $1$ \;
                    }
                    \Else{
                        add edge from $l$ to block with weight $0$ \; 
                    }
                    \Break
                }
            }
        }
    }
\end{algorithm}


\trace{compulsoryswitch.pdf}{Graph example}

In Figure \ref{Graph example} a simple example of such a graph is demonstrated. For this trace we notice that $w(y)$ of
$q$ thread is concurrent with $r(x)$ while it happens before $w(y)$ of the $p$ thread. Each transition costs 1
preemption switch that is why the weight is 1. Moreover transitions between the same thread cost 0. The most important
fact, however, is that if for any reason we try to violate the happen before relation (e.g. starting from $r(x)$ we jump
to $w(x)$) there is no way to traverse all the nodes. 

We can see that there is a hamiltonian path with weight 1 for the given trace. In fact, this is the minimum hamiltonian
path of the graph. We can easily realize that there is no equivalent trace with the initial one that has bound count
less than 1. 

We can infer that the calculation of this bound count is reduced to the weight of the minimum hamiltonian path of this
graph. This problem it is known to be NP-complete. As a result any algorithm that would calculate this weight would not
be significantly better than a DFS-exploration. 

This is an extremely interesting indication of the difficulty of the DPOR bounding problem since the addition of the
conservative sets imply this DFS exploration at the state space. As a result this algorithm would not be better than the
already proposed BPOR algorithm.

Now that the difficulty of this approach is clear a new question arises. Is it possible to approximate the total weight of the minimum hamiltonian path?
Such an algorithm would cover a greater state space than the Naive-BPOR without the explosion caused by the conservative branches.

\subsection{Approximating Bound Count}
There are two approaches examined in order to approximate the value of the bound count. The idea of both algorithms is
based on this observation: A preemption switch is compulsory when for two blocks of the same process A a block of
another process B must intervene in order for the happens-before relations to hold true. As a result, the execution of
the first A block should stop so the execution of the B block takes place followed by the execution of the A block
again. Hence it should hold $e_1(A) \rightarrow e(B) \rightarrow e_2(A)$. In case of $e_1(A) \not \rightarrow e(B)$ or
$e(B) \not \rightarrow e_2(A)$ we could invert the blocks without affecting the happens-before relations and, thus
construct an equivalent trace with lower bound count.

The algorithm is presented here:\\

\SetKwProg{Fn}{Function}{}{}
\begin{algorithm}[H]
    \caption{First Approximation Algorithm}
    \label{First Approximation Algorithm}
    \Fn{BoundCount($E$,$current\_bound$)}{
        \For{$i=0 \text{ to } len(E)-1$}{
            \If{$E[i].pid = last(E).pid$}{
                $higher\_block = i$ \;
                \Break
            }
        }
        \For{$i = higher\_block+1 \text{ to } len(E)-1$}{
            \If{$ E[higher\_block] \rightarrow E[i] \rightarrow last(E)$}{
                current\_bound++ \;
                \Return
            }
        }
    }
\end{algorithm}

In Algorithm \ref{First Approximation Algorithm} we find the most recent block with the same pid as with the last block. We, then try to find if there is an event that happens before the first
and after the last event. If exists such an event we increase the counter.
Notice that for establishing the happens-before relation vector clocks can be used.
Moreover, it is obvious that more happens-before relations can be counted.


Th second algorithm explores more state space than it is required.\\

\begin{algorithm}[H]
    \caption{Second Approximation Algorithm}
    \Fn{BoundCount($E$,$current\_bound$)}{
        \For{$i=len(E)-1 \text{ to } 0$}{
            \If{$E[i].pid = last(E).pid$}{
                $lower\_block = i$ \;
                \Break
            }
        }

        \For{$i = lower\_block+1 \text{ to } len(E)-1$}{
            \If{$ E[lower\_block] \rightarrow E[i] \rightarrow last(E)$}{
                current\_bound++\;
                \Return
            }
        }
    }
\end{algorithm}

This algorithm starts the search for an event that intervenes the two events of the same id from the immediately previous block 
with the same id. 

\subsection{Evaluation of Approximating Algorithms}
The previous discussed approaches were tested and produced some interesting results. The both estimation algorithms seem
to be ``more sound'' than the BPOR and may explore traces that exceed the bound. This stems from the fact that they tend
to underestimate the bound count since there are more complex relations between blocks that result traces with higher
bound count than the one estimated. We notice that in writer-N-readers example the number of traces explored is stable
for every bound. In fact, each trace of this test has an equivalent trace with zero bound count since in each thread
only a command related to a shared variable is executed.

\graph{img/wNrLB.png}{writer-N-readers bounded by the first estimation algorithm}
\smalltabular{tables/lazy1_bounded.tex}{Traces for the first estimation algorithm for various bound limits}

\subsection{Implementation of Lazy-BPOR}

Some of the testcases such as lastzero or writer-N-readers made clear that an implementation of a bound count function
which does not simply counts the preemptive switches in traces can prevent the state space explosion caused by the
conservative branches added. The next step is to implement the Lazy-BPOR, an algorithm that calculates the number of
compulsory preemptive switches (preemptive switches that cannot be avoided in any equivalent trace with the one
examined). The main difference from the Naive-BPOR is that the Lazy-BPOR maintains throughout the execution of the DPOR
a graph of the blocks that are contained in the traces. When a new block is created, it is added by the algorithm
previously described. When it comes to the calculation of the bound count, the minimum hamiltonian path is calculated.
The weight of this path corresponds to the bound count taken into consideration.

\begin{algorithm}
    \caption{Lazy-BPOR}
    \label{Lazy-BPOR}
    \Let{$G =: \emptyset$}
    Explore($\langle \rangle$,$\emptyset$,$G$,$b$)\;
    \Fn{Explore($E$,$Sleep$,$G$,$b$)}{
        \If{$\exists p \in (enabled(s_{[E]}) \backslash Sleep)$ such that $B_v(E.p) \leq b$ }{
            backtrack(E) $:={p}$ \;
            \While{$\exists p \in (backtrack(E) \backslash Sleep $}{
                \ForEach{$e \in dom(E)$ such that $e \lesssim_{E.p} next_{[E]}(p) $}{
                    \Let{$E' = pre(E,e)$}
                    \Let{$u = notdep(e,E).p$}
                    \If{$I_{E'}(u) \cap backtrack(E') = \emptyset$}{
                        add some $q' \in I_{[E']}(u) to backtrack(E') $ \;
                    }
                }
                \Let{$Sleep' := \{q \in Sleep \mid E \models p \diamondsuit q \} $}
                \If{$p$ creates a new block}{
                    \Let{$block$ = $last\_block(E)$}
                    \Let{$G'$ = add\_block($block$,$G$)}
                }
                \If{$min \{ Ham\_path(G') \text{ which compensate with all happens-before relations of } E \} \leq b $}{
                    $Explore(E.p, Sleep,G',b)$ \;
                    add $p$ to $Sleep$ \;
                }
            }
        }
    }
\end{algorithm}



\subsection{Lazy-BPOR - RCU Evaluation}

The results are demonstrated below. Since we have to compare Lazy-BPOR with BPOR the bugged versions of the DPOR must be used. The bugged version
of DPOR is that where the last running thread is prioritized.

At Figures \ref{Comparison between DPOR and Lazy-BPOR} and \ref{Comparison between BPOR and Lazy-BPOR} we present the results for the various
testcases of RCU test suite.


\landscapetabular{"tables/lazy_comp.tex"}{Comparison between DPOR and Lazy-BPOR}

\landscapetabular{"tables/lazy_buged_comp.tex"}{Comparison between DPOR and Lazy-BPOR without the bug}


We compare the algorithm with BPOR. We notice that Lazy-BPOR examines less traces but requires longer time since the cost of the lazy bound count is significantly
increased. This is due to the calculation of the minimum hamiltonian path.

\landscapetabular{"tables/hline_pandas_lazy_preep.tex"}{Comparison between BPOR and Lazy-BPOR}

\section{Conclusion}
Even though the Lazy-BPOR is not proved to be a more efficient way than all the other sound BPOR algorithms examined in this thesis it
provides some interesting results.

\begin{itemize}
    \item It is possible to explore a preemption-bounded state space without the addition of conservative branches.
    \item It provides an upper bound for the number of traces explored in BPOR no matter the bound. In fact the number of traces
    explored by Lazy-BPOR at worst case equal to the number of traces explored by the unbounded DPOR. This is true since no conservative
    branches are added.
    \item The most important is that provides a reduction of the preemption-bounded search to a well known graph problem where many heuristics can
    be applied in order to expedite the calculation of the minimum hamiltonian path.
\end{itemize}


\chapter{Concluding Remarks}
\label{Chapter 6}

In this thesis we have implemeted persistent set based DPOR on Nidhugg and used it to implement BPOR.
We combined source-DPOR and BPOR and showed that both approaches are equivalent. We used this approach 
verify RCU and count the minimum preemptive-switches of each injection and showed that bounded DPOR can
find all these injections in a shorter period of time exploring less traces. Moreover we explored other
techniques that could reduce the number of traces explored showing that are not feasible or are equivalent
to the already proposed techniques.

However, this exploration is far from over. Our tasks for the future include:

\begin{itemize}
    \item The examination and implementation of other bounding techniques and their evaluation compared to the premption bounded dynamic
    partial order reduction.
    \item The implementation of optimal DPOR for Nidhugg and the implications of such an implementation to the bounded DPOR.
    \item The examination of novel techniques such Observers can reduce the state space of the exploration.
    \item The further useage of Nidhugg in the verification of concurrent software.
    \item The parallelization of Nidhugg and its effects on performance on both unbounded and bounded search.
\end{itemize}
\nocite{*}

%%%  Bibliography

\bibliographystyle{softlab-thesis}
\bibliography{thesis}


%%%  Appendices

\backmatter

\appendix

\chapter{Tree RCU Modified Functions \label{appendix_a}}

Below are listed some functions from \ccode{kernel/rcu/tree.c} file
that have been modified for the bug injection procedure (see Section
\ref{valtree_gp}). The code relevant to the bug injections has been
colored blue.

%\lstinputlisting[linerange={187-199,1530-1572,1597-1671,1958-2011,3554-3581},nolol=true,style=custom}]{/home/michalis/Dropbox/sxolh/thesis/rcu/valtree/v3.19/tree.c}

\begin{code_appendix}
  void rcu_sched_qs(void) 
  {@
  #ifdef LIVENESS_CHECK_2
	  return;
  #endif@
	  if (!rcu_sched_data[get_cpu()].passed_quiesce) {
		  trace_rcu_grace_period(TPS("rcu_sched"),
				         rcu_sched_data[get_cpu()].gpnum,
				         TPS("cpuqs"));
		  rcu_sched_data[get_cpu()].passed_quiesce = 1;
	  }
  }

  static bool __note_gp_changes(struct rcu_state *rsp, struct rcu_node *rnp,
			      struct rcu_data *rdp)
  {
	  bool ret;

	  /* Handle the ends of any preceding grace periods first. */
	  if (rdp->completed == rnp->completed) {

		  /* No grace period end, so just accelerate recent callbacks. */
		  ret = rcu_accelerate_cbs(rsp, rnp, rdp);

	  } else {

		  /* Advance callbacks. */
		  ret = rcu_advance_cbs(rsp, rnp, rdp);

		  /* Remember that we saw this grace-period completion. */
		  rdp->completed = rnp->completed;
		  trace_rcu_grace_period(rsp->name, rdp->gpnum, TPS("cpuend"));
	  }

	  if (rdp->gpnum != rnp->gpnum) {
		  /*
		   * If the current grace period is waiting for this CPU,
		   * set up to detect a quiescent state, otherwise don't
		   * go looking for one.
		   */
		  rdp->gpnum = rnp->gpnum;
		  trace_rcu_grace_period(rsp->name, rdp->gpnum, TPS("cpustart"));
		  rdp->passed_quiesce = 0;@
  #ifdef LIVENESS_CHECK_1
		  rdp->qs_pending = 0;
  #else
		  rdp->qs_pending = !!(rnp->qsmask & rdp->grpmask);
  #endif
  #ifdef FORCE_FAILURE_5
		  rnp->qsmask &= ~rdp->grpmask;
  #endif@
		  zero_cpu_stall_ticks(rdp);
	  }
	  return ret;
  }      

  static int rcu_gp_init(struct rcu_state *rsp)
  {
	  struct rcu_data *rdp;
	  struct rcu_node *rnp = rcu_get_root(rsp);

	  rcu_bind_gp_kthread();
	  raw_spin_lock_irq(&rnp->lock);
	  smp_mb__after_unlock_lock();
	  if (!ACCESS_ONCE(rsp->gp_flags)) {
		  /* Spurious wakeup, tell caller to go back to sleep.  */
		  raw_spin_unlock_irq(&rnp->lock);
		  return 0;
	  }
	  ACCESS_ONCE(rsp->gp_flags) = 0; /* Clear all flags: New grace period. */

	  if (WARN_ON_ONCE(rcu_gp_in_progress(rsp))) {
		  /*
		   * Grace period already in progress, don't start another.
		   * Not supposed to be able to happen.
		   */
		  raw_spin_unlock_irq(&rnp->lock);
		  return 0;
	  }

	  /* Advance to a new grace period and initialize state. */
	  record_gp_stall_check_time(rsp);
	  /* Record GP times before starting GP, hence smp_store_release(). */
	  smp_store_release(&rsp->gpnum, rsp->gpnum + 1);
	  trace_rcu_grace_period(rsp->name, rsp->gpnum, TPS("start"));
	  raw_spin_unlock_irq(&rnp->lock);

	  /* Exclude any concurrent CPU-hotplug operations. */
	  mutex_lock(&rsp->onoff_mutex);
	  smp_mb__after_unlock_lock(); /* ->gpnum increment before GP! */

	  /*
	   * Set the quiescent-state-needed bits in all the rcu_node
	   * structures for all currently online CPUs in breadth-first order,
	   * starting from the root rcu_node structure, relying on the layout
	   * of the tree within the rsp->node[] array.  Note that other CPUs
	   * will access only the leaves of the hierarchy, thus seeing that no
	   * grace period is in progress, at least until the corresponding
	   * leaf node has been initialized.  In addition, we have excluded
	   * CPU-hotplug operations.
	   *
	   * The grace period cannot complete until the initialization
	   * process finishes, because this kthread handles both.
	   */
	  rcu_for_each_node_breadth_first(rsp, rnp) {
		  raw_spin_lock_irq(&rnp->lock);
		  smp_mb__after_unlock_lock();
		  rdp = &rsp->rda[get_cpu()];
		  rcu_preempt_check_blocked_tasks(rnp);
  @#ifdef FORCE_FAILURE_3
		  rnp->qsmask &= ~rdp->grpmask;
  #else
		  rnp->qsmask = rnp->qsmaskinit;
  #endif@
		  ACCESS_ONCE(rnp->gpnum) = rsp->gpnum;
		  WARN_ON_ONCE(rnp->completed != rsp->completed);
		  ACCESS_ONCE(rnp->completed) = rsp->completed;
		  if (rnp == rdp->mynode)
			  (void)__note_gp_changes(rsp, rnp, rdp);
		  rcu_preempt_boost_start_gp(rnp);
		  trace_rcu_grace_period_init(rsp->name, rnp->gpnum,
					      rnp->level, rnp->grplo,
					      rnp->grphi, rnp->qsmask);
		  raw_spin_unlock_irq(&rnp->lock);
		  cond_resched_rcu_qs();
	  }

	  mutex_unlock(&rsp->onoff_mutex);
	  return 1;
  }

  static void
  rcu_report_qs_rnp(unsigned long mask, struct rcu_state *rsp,
		    struct rcu_node *rnp, unsigned long flags)
	  __releases(rnp->lock)
  {
	  struct rcu_node *rnp_c;

@  #ifdef LIVENESS_CHECK_3
	  return;
  #endif@
	  /* Walk up the rcu_node hierarchy. */
	  for (;;) {
		  if (!(rnp->qsmask & mask)) {

			  /* Our bit has already been cleared, so done. */
			  raw_spin_unlock_irqrestore(&rnp->lock, flags);
			  return;
		  }
		  rnp->qsmask &= ~mask;
		  trace_rcu_quiescent_state_report(rsp->name, rnp->gpnum,
						   mask, rnp->qsmask, rnp->level,
						   rnp->grplo, rnp->grphi,
						   !!rnp->gp_tasks);
@  #ifndef FORCE_FAILURE_6
		  if (rnp->qsmask != 0 || rcu_preempt_blocked_readers_cgp(rnp)) {

			  /* Other bits still set at this level, so done. */
			  raw_spin_unlock_irqrestore(&rnp->lock, flags);
			  return;
		  }
  #endif@
		  mask = rnp->grpmask;
		  if (rnp->parent == NULL) {

			  /* No more levels.  Exit loop holding root lock. */

			  break;
		  }
		  raw_spin_unlock_irqrestore(&rnp->lock, flags);
		  rnp_c = rnp;
		  rnp = rnp->parent;
		  raw_spin_lock_irqsave(&rnp->lock, flags);
		  smp_mb__after_unlock_lock();
		  WARN_ON_ONCE(rnp_c->qsmask);
	  }

	  /*
	   * Get here if we are the last CPU to pass through a quiescent
	   * state for this grace period.  Invoke rcu_report_qs_rsp()
	   * to clean up and start the next grace period if one is needed.
	   */
	  rcu_report_qs_rsp(rsp, flags); /* releases rnp->lock. */
  }

  static int __init rcu_spawn_gp_kthread(void)
  {
	  unsigned long flags;
	  struct rcu_node *rnp;
	  struct rcu_state *rsp;
	  struct task_struct *t;

	  rcu_scheduler_fully_active = 1;
@  #ifdef ENABLE_RCU_BH
	  for_each_rcu_flavor(rsp) {
		  t = kthread_run(rcu_gp_kthread, rsp, "%s", rsp->name);
		  BUG_ON(IS_ERR(t));
		  rnp = rcu_get_root(rsp);
		  raw_spin_lock_irqsave(&rnp->lock, flags);
		  rsp->gp_kthread = t;
		  raw_spin_unlock_irqrestore(&rnp->lock, flags);
	  }
  #else
	  t = kthread_run(rcu_gp_kthread, &rcu_sched_state, "%s",rcu_sched_state.name);
	  rnp = rcu_get_root(&rcu_sched_state);
	  raw_spin_lock_irqsave(&rnp->lock, flags);
	  rcu_sched_state.gp_kthread = t;
	  raw_spin_unlock_irqrestore(&rnp->lock, flags);
  #endif@
	rcu_spawn_nocb_kthreads();
	rcu_spawn_boost_kthreads();
	return 0;
  }
  early_initcall(rcu_spawn_gp_kthread);
\end{code_appendix}
%%%  End of document

\end{document}
