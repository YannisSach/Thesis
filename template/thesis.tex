\documentclass[diploma, english]{softlab-thesis}

\usepackage{amsmath}
\usepackage{float}
\usepackage{color}
\usepackage[table,xcdraw]{xcolor}
\usepackage{siunitx}
\usepackage{multirow}
\usepackage{colortbl}
\usepackage{subcaption}
%\usepackage{refcheck}

\usepackage{url}
\usepackage[colorlinks]{hyperref}
\hypersetup{
  bookmarksnumbered,
  citecolor={blue},
  linkcolor={blue},
  urlcolor={blue},
  pdfpagemode={UseOutlines}
}

\usepackage{amsthm}
\theoremstyle{definition}
\newtheorem{definition}{Definition}[chapter]

\usepackage{listings}
\usepackage{lstlinebgrd}
\lstset{
    numbers=left,
    numberstyle=\tiny\color{black},
    basicstyle=\ttfamily\footnotesize,
    basewidth=0.59em,
    keywordstyle=[3]{},
    commentstyle=\itshape\footnotesize,
    tabsize=8,
    frame=single,
    frameround=tttt,
    showstringspaces=false,
    breaklines=false,
    captionpos=b,
    aboveskip=\bigskipamount,
    belowskip=\bigskipamount,
    escapechar=^
}
\lstdefinestyle{custom}{
    numbers=left,
    numberstyle=\tiny\color{black},
    basicstyle=\ttfamily\footnotesize,
    basewidth=0.59em,
    keywordstyle=[3]{},
    commentstyle=\itshape\footnotesize,
    tabsize=8,
    frame=single,
    frameround=tttt,
    showstringspaces=false,
    rulecolor=\color{black},
    breaklines=false,
    captionpos=b,
    aboveskip=\bigskipamount,
    belowskip=\bigskipamount,
    escapechar=^,
    moredelim=**[is][\color{blue}]{@}{@}
}

\lstnewenvironment{code}[2]{
  \nopagebreak
  \lstset{language=C, label={#1}, caption={#2}, style=custom}
}{}

\lstnewenvironment{code_appendix}{
  \nopagebreak
  \lstset{language=C, style=custom, numbers=none,rulecolor=\color{black}}
}{}

\lstnewenvironment{console}[2]{
  \nopagebreak
  \lstset{label={#1}, caption={#2}, numbers=none, style=custom}
}{}
 
\lstnewenvironment{assembly}[2]{
  \nopagebreak
  \lstset{language={[x86masm]Assembler}, label={#1}, caption={#2}, style=custom}
}{}

\usepackage{etoolbox}
\expandafter\patchcmd\csname \string\lstinline\endcsname{%
  \leavevmode
  \bgroup
}{%
  \leavevmode
  \ifmmode\hbox\fi
  \bgroup
}{}{%
  \typeout{Patching of \string\lstinline\space failed!}%
}
\newcommand{\ccode}[1]{\lstset{language=C}\protect\lstinline!#1!}

%%%
%%%  The document
%%%

\begin{document}

%%%  Title page

\frontmatter

\title{Systematic Concurrency Testing of Read-Copy-Update under Sequentially Consistent and Weak Memory Models}
\author{Michalis Kokologiannakis}
\date{October 2016}
\datedefense{13}{10}{2016}

\supervisor{Konstantinos Sagonas}
\supervisorpos{Associate Professor NTUA}

\committeeone{Konstantinos Sagonas}
\committeeonepos{Associate Professor NTUA}
\committeetwo{Nikolaos S. Papaspyrou}
\committeetwopos{Associate Professor NTUA}
\committeethree{Nectarios Koziris}
\committeethreepos{Professor NTUA}

\TRnumber{CSD-SW-TR-1-16}  % number-year, ask nickie for the number
\department{Division of Computer Science}

\maketitle


%%%  Abstract, in Greek

\iffalse
\begin{abstractgr}%
  
H διεξοδική επαλήθευση και έλεγχος προγραμμάτων γραμμένων στο πρότυπο του ταυτοχρονισμού είναι ένα μιά τόσο σημαντική
όσο και απαιτητική εργασία. Προκειμένουν να επαληθεύσουμε την ορθότητα ενός τέτοιου προγραμμάτος θα πρέπει να εξετάσουμε
όλες τις δυνατές δρομολογήσεις που αυτό μπορεί να παράξει. Το Stateless model checking με τη χρήση αναγωγής σε δυναμικές
σχέσεις μερική διάταξης (Dynamic Partial Order Reduction ή DPOR) είναι μια τεχνική η οποία αντιμετωπίζει το πρόβλημα του της
έκρηξης του μεγέθους του χώρου καταστάσεων. Παρολ' αυτά η επαλήθευση μεγάλων προγραμμάτων μπορεί να διαρκέσει
περισσότερο από όσο θα επιθυμούσαν οι developers. Σε αυτές τις περιπτώσεις ο περιορισμός (οριοθέτηση) της αναζήτησης
(bounded search) με τη χρήση κάποιου ορίου μπορεί να αποδειχθεί αρκετά χρήσιμη. Η οριοθετημένη αναζήτηση, σε αντίθεση με την DPOR 
απαλείφει το πρόβλημα της έκρηξης του μεγέθους το χώρου καταστάσεων αγνοώντας δρομολογήσεις που ξεπερνούν κάποιο όριο

Στην παρούσα διπλωματική περιγράφεται η υλοποίηση του preemption bounded DPOR (BPOR) στο Nidhugg ένα εργαλία για την εύρεση λαθών (bugs)
που στοχεύει στην ανεύρεση σφαλμάτων που προκείπτουν από το μοντέλο του ταυτοχρονισμού και τα χαλαρά μοντέλα μνήμης (relaxed memory models).
Συγκεκριμένα υλοποιήθηκαν τρεις τεχνικές περιορισμού: ο Naive-BPOR, o Nidhugg-BPOR και ο Source-BPOR. Οι τεχνικές αυτές αξιολογήθηκαν τόσο σε 
συνθετικά τεστ όσο και σε λογισμικό που χρησιμοποιείται στην πραγματικότητα. Συγκεκριμένα ο μηχανισμός Read-Copy-Update του πυρήνα του Linux
επαληθεύτηκε ξανά με τη χρήση αυτών των τεχνικών. Επιπλέον εξετάστηκε κατα πόσο βελτιστοποιήσεις που εφαρμόζονται στον μη περιορισμένο DPOR μπορούν
να βελτιώσουν την επίδοση του BPOR.
  
\begin{keywordsgr}
\input{./keywords_gr.tex}
\end{keywordsgr}
\end{abstractgr}
\fi

%%%  Abstract, in English

\begin{abstracten}%
  
Thorough verification and testing of concurrent programs is an important, but also
challenging task. In order to verify a concurrent program one must examine all possible 
different interleavings the scheduler can produce. Stateless model checking with Dynamic Partial Order Reduction 
is a technique proposed to deal with state space explosion. Nevertheless, for larger programs the verification 
takes longer than the developers are willing to wait. In these cases, bounded search can be proved useful. Bounded search,
in contrast to the DPOR, alleviates state-space explosion by pruning the executions that exceed a bound. 

This thesis describes the implementation of the preemption bounded DPOR (BPOR) on Nidhugg, a bug finding tool which targets bugs caused by concurrency
and relaxed memory consistency in concurrent programs. Specifically three bounding techniques were implemented: the Vanilla-BPOR, the BPOR,
and the Source-BPOR. The three techniques were evaluated both in synthetic and in real world software. Specifically Read-Copy-Update mechanism of Linux Kernel
was verified again. Moreover it is examined whether optimizations that have been suggested for the 
unbounded DPOR can improve the efficiency of BPOR.
  
\begin{keywordsen}
Formal Verification, Stateless Model Checking, Systematic Concurrency Testing, RCU, Read-Copy-Update, Bounded Dynamic Partial Order Reduction,
\end{keywordsen}
\end{abstracten}


%%%  Acknowledgements

\iffalse
\begin{acknowledgementsgr}
\end{acknowledgementsgr}
\fi

\begin{acknowledgementsen}
  First of all, I am most grateful to my advisor, Kostis Sagonas, for all his help
  and support during this effort. With his ceaseless enthusiasm and avid interest
  in research he managed to motivate and inspire me, while his essential advice
  and guidance throughout this process have been invaluable. I absolutely enjoyed
  every minute of working with him for my thesis during the last year,
  and I am looking forward to working with him again in the future.

  I am also much obliged to Paul McKenney, for dedicating a lot of his free time
  to answer all of my questions willingly and promptly. With his 
  profound insight into RCU-related issues he helped me in numerous occasions,
  and his suggestions and input were extremely helpful.

  Finally, I would like to extend my thanks to my parents and brother for
  their patience, love and support throughout these years; after all, they are
  the ones who made this endeavour possible.
  
\end{acknowledgementsen}

%%%  Various tables

\tableofcontents
%\listoftables
\listoffigures
\listoflistings


%%%  Main part of the book

\mainmatter


 c\chapter{Introduction}

Moore's Law, named after Intel's co-founder Gordon Moore, states that the number of transistors that can be placed on an integrated circuit doubles
roughly every two years. For decades, chipmakers have succeeded in shrinking chip geometries, allowing Moore's Law to remain on track and consumers to
get their hands on even more powerful laptops, tablets, and smartphones. Software developers could just lay back and wait for Moore's Law to take effect.
However, constraints such as heat, clock speeds have largely stood still, and the incremental increase of the performance of each individual 
processor core impedes the further acceleration of software execution. In order for developers to compensate with the demand for efficient software, programming paradigms such
as concurrent programming have become a necessity. However, new challenges arise from concurrent programming, since this type of programming is harder and more
error-prone than its sequential counterpart. When programming with multiple processes/threads many errors may occur due to the fact that the threads
may access and edit the shared memory or require to execute lines of code excluding other threads.

More specifically, the typical problems with concurrency can be outlined as follows:
\begin{description}
\item[Race conditions] A strange interleaving of processes has an unintended effect.
\item[Deadlocks] Two or more processes stop and wait for each other.
\item[Livelocks] Two or more processes keep executing without making any progress.
\item[Resource starvations] Two or more processes are stuck in circular waiting for the resources.
\end{description}
What's worse is that these problems are usually Heisenbugs~\cite{Musu08}; i.e., they can alter their behavior or completely
disappear when one tries to isolate them, since they go hand in hand with the order of
execution of the processes involved.

\section{Testing and Verification of Concurrent Programs}

Testing and verifying the correctness of a concurrent program is a demanding task. 
A technique for the systematic exploration of a program's state space is model checking~\cite{WikipediaModelChecking}.
Model checking is a method for formally verifying concurrent systems through specifications about the system expressed as 
temporal logic formulas and then employing efficient algorithms that can traverse the model defined by the system and check whether
the specifications hold. The major problem model checking tools have to face is the combinatorial explosion of the state space since
a vast number of global states have to be captured and stored. Many techniques have been proposed in order to tackle this problem.
Stateless model checking, for example, avoids storing global states. This technique has been implemented in tools such as Verisoft~\cite{SMC,Gode05}, 
CHESS~\cite{Musu08}, Concuerror~\cite{Chri13}, Nidhugg~\cite{AbdullaAronisAtigJonssonLeonardssonSagonasSMC2015} and RCMC~\cite{RCMC}. The observation that two 
interleavings are equivalent if one can be obtained from the other by swapping adjacent, independent execution steps is the core of the partial
order reduction \cite{Valmari1991, Peled1993, Godefroid1996,POR,JACM} techniques used by many of these tools. Dynamic Partial Order Reduction (DPOR) techniques
capture dependencies between operations of concurrent threads while the program is running~\cite{FlanaganDPOR,JACM}. The exploration begins with an arbitrary interleaving whose steps are then
used to identify operations and points where alternative interleavings need to
be explored in order to capture all program behaviors. Another approach is bounded model checking \cite{BoundedModelChecking} where the finite state
machine is unrolled for a fixed number of steps and the specifications are checked within these steps. Bounded model checking can be combined with the partial
order reduction for modeling executions, and has been effectively implemented in tools such as CBMC~\cite{CBMC}, Nidhugg~\cite{AbdullaAronisJohnssonSagonasDPOR2014} and RCMC~\cite{RCMC}.
Unfortunately, all these techniques still have to deal with the problem of the state space explosion. In order to deal with this problem further
bounding of the exploration is required. Many different bounding techniques have been examined \cite{Thomson}, such as preemption bounding, delay bounding, a controlled random scheduler, and probabilistic concurrency testing (PCT).

\section{Aim of this Thesis}

The goals of this thesis are to: 
\begin{itemize}
\item Implement a preemption bounding technique for DPOR~\cite{BPOR} in the Nidhugg tool. 
\item Examine whether the Source-DPOR and Optimal-DPOR techniques~\cite{AbdullaAronisJohnssonSagonasDPOR2014} can
  be enhanced to support an efficient implementation of bounded partial order reduction,
  preferably one that gives some sort of guarantees.
\item Confirm or disapprove the ability of bounded dynamic partial order reduction to locate errors faster than unbounded partial order reduction.
\item Examine whether the empirical observation that most concurrency errors can manifest themselves in a small number of preemptions~\cite{Musu07} is correct.
\item Explore alternative algorithms that can perform preemption-bounded partial order reduction.
\end{itemize}

\iffalse
The purpose of this thesis is the implementation of a preemption bounding technique \cite{BPOR} for Nidhugg and the combination
of this technique with the a novel technique \cite{AbdullaAronisJohnssonSagonasDPOR2014} suggested for better coverage of the state space.
The bounded-DPOR was used to verify the linux kernel \cite{LinuxKernel} and specifically RCU \cite{Spin}. RCU is a synchronization
mechanism used heavily in Linux kernel, and many of the kernel’s subsystems rely on RCU’s correct operation. By using BPOR the minimum preemptive
switches required to track failure injections were counted. As a result the empirical observation that errors occur in a small bound count was confirmed.
Moreover, the possible application of various optimizations used for unbounded DPOR on bounded DPOR are examined. 
\fi

\section{Overview}
In Chapter \ref{sec:background} we review the theoretical background for both unbounded and bounded DPOR.
In Chapters \ref{unbounded} and \ref{bounded} the unbounded DPOR and bounded DPOR algorithms implemented and evaluated are
described in further detail. In Chapter~\ref{implementations} the technical details of the implementations are
discussed. The evaluation of the each algorithm is given in Chapter \ref{Chapter 4} where the algorithms are tested
using both synthetic tests and a code base of significant size (RCU) which is part of the Linux kernel. In Chapter \ref{Chapter 5} we present and evaluate an
alternative approach to the bounding problem.
Finally, in Chapter \ref{Chapter 6} we summarize the previous chapters, draw some conclusions, and present some possible extensions to our work.

\chapter{Background}
\label{Chapter 2}

\section{Concurrent programming}

Concurrent computing, which is implemented by concurrent programming paradigm, is a form of computing in which several 
computations are executed during overlapping time 
periods—concurrently—instead of sequentially (one completing before the next starts). 
This is a property of a system—this may be an individual program, a computer, or a network—and there is a separate execution point 
or "thread of control" for each computation ("process"). A concurrent system is one where a computation can advance without waiting for 
all other computations to complete.
The main challenge in designing concurrent programs is concurrency control: ensuring the correct sequencing of the 
interactions or communications between different computational executions, and coordinating access to resources that are shared among executions.
Potential problems include race conditions, deadlocks, livelocks and resource starvation. 
The scheduler is usually responsible for running a thread. Due to this scheduling non-determinism the programmer cannot always be aware of which thread
will be scheduled next.

An important aspect of a concurrent program is the notion of the set of interleavings which is the set of all the execution paths a program can follow.
Intuitively, if we imagine a process as a (possibly infinite) sequence/trace of statements (e.g. obtained by loop unfolding),
then the set of possible interleavings of several processes consists of all possible sequences of statements of any of these processes.

As it can be inferred, debugging this kind of programs can be proved extremely challenging. The challenge mainly emerges from the fact that it is 
not always clear which thread command will be executed. Moreover the error may not always occur during debugging since there may be only a limited
number of interleavings that produce an error. 

\section{Testing, Model Checking and Verification}

Dynamic software model checking consists of adapting model check-ing into a form of systematic testing that is applicable to industrial-size software. 
Over the last two decades, dozens of tools following this paradigm have been de-veloped for checking concurrent and data-driven software. Compared to 
traditionalsoftware testing, dynamic software model checking provides better coverage, butis more computationally expensive. Compared to more general 
forms of programverification like interactive theorem proving, this approach provides more limitedverification guarantees, but is cheaper due to its higher
level of automation. Dynamic software model checking thus offers an attractive practical trade-off between testing and formal verification. 

A graph that greatly demonstrates the differences between these terms is show in 

\trace{testmodver.png}{Comparing Testing, Model Checking and Verification}


\section{Stateless model checking and partial order reduction}

In order to find an error of a concurrent program, one must examine every possible interleaving this program can produce. Usually the error 
would occur only under some interleaving the programmer did not take into consideration, making its detection extremely difficult. Stateless model checking is based on the idea of driving 
the program along all these possible interleavings. However, this approach suffers from state explosion, i.e., the number of all possible interleavings grows 
exponentially with the size of the program and the number of threads. Several approaches to this problem have been proposed in order to deal with this challenge: 
partial order reduction \cite{Godefroid1996} and bounded search \cite{BPOR}. Partial order reduction is aiming to reduce the number of interleavings explored by eliminating equivalent interleavings.
These equivalent traces are produced by the inversion of independent events which do not affect the results of the program. For example, the scheduling of two threads that read a local
variable can be inverted since the result of the operation is affected by the order under which each operation occurs. 
There are two ways that a partial order reduction algorithm can be implemented. The first is a
static partial order reduction algorithm \cite{Static1997} where the dependencies between two threads are tracked before the execution of the concurrent program. 
The second is the Dynamic partial order reduction (DPOR) which observes the program's dependencies of runtime. It is important to notice that the size of
the state space still grows exponentially. 

For larger programs DPOR often runs longer than developers are willing to wait. In these cases, bounded search can be proved useful. Bounded search,
in contrast to DPOR, alleviates state-space explosion by pruning the executions that exceed a bound \cite{Thomson}. There have been proposed many bounded techniques
such as preemption bounded search \cite{BPOR} or delay bounded search \cite{Delay11}. All bounded search techniques are based on the notion that many of the concurrency bugs can be
tracked even when the bound limit is set to be small, thus the time required for a bug to be found is significantly smaller.


\section{Vector Clocks}

A vector clock is an algorithm for generating a partial ordering of events in a distributed or concurrent system and detecting causality violations. 
Just as in Lamport timestamps \cite{Lamp}, interprocess messages contain the state of the sending process's logical clock. 
A vector clock of a system of N processes is an array/vector of N logical clocks, one clock per process; 
a local "smallest possible values" copy of the global clock-array is kept in each process, with the following rules for clock updates:

\begin{enumerate}
    \item Each process experiencing an internal event, it increments its own logical clock in the vector by one.
    \item Each time a process receives a message or performs an action on a shared variable, it increments its own logical clock in the vector by one and updates each element in its vector 
    by taking the maximum of the value in its own vector clock and the value in the vector in the received message or the maximum value of all processes that share
    the same shared variable.
\end{enumerate}

An example execution of the algorithm is shown in Figure \ref{Clock example} where both the source code and an explored trace are given.
As we can easily notice for each command the clock of the main thread increases. The thread <0.0> starts to run its clock for the thread <0> is 8 since 
that is the moment when the thread was spawned. When the value of y is read the clock for <0> increases again so it corresponds with the y=1 event. When the first thread is
scheduled again then its clock for the <0.0> is 7 since \verb|pthread_join()| command takes place.


\Side{./code/clocks.c}{Vector Clock example}{./code/clocks.out}{Vector Clock output}{Clock example}

Every algorithm that is presented in this thesis is based on vector clocks.


\section{Notation}

Before examining delving in the problem of dynamic partial order reduction it is crucial to explain the notation that will be used.
An execution sequence $E$ of a system is a finite sequence of
execution steps of its processes that is performed from the initial
state which we denote as $s_0$. Since each execution step is deterministic, an execution
sequence $E$ is uniquely characterized by the sequence of processes
that perform steps in $E$. For instance, $p.p.q$ denotes the execution
sequence where first $p$ performs two steps, followed by a step of $q$.
The sequence of processes that perform steps in $E$ also uniquely
determine the (global) state of the system after $E$, which is denoted
$s_{[E]}$. For a state $s$, let $enabled(s)$ denote the set of processes $p$ that
are enabled in $s$ (i.e., for which execute $p(s)$ is defined). We use $.$ to
denote concatenation of sequences of processes. Thus, if $p$ is not
blocked after $E$, then $E.p$ is an execution sequence.
An event of $E$ is a particular occurrence of a process in $E$.
We use $\langle p,i \rangle$ to denote the ith event of process $p$ in the execution
sequence $E$. In other words, the event $\langle p,i \rangle$ is the ith execution step
of process $p$ in the execution sequence $E$. We use $dom(E)$ to denote
the set of events $\langle p,i \rangle$ which are in $E$, i.e., $\langle p,i \rangle \in dom(E)$ iff $E$
contains at least $i$ steps of $p$. We will use $e,e',...$ , to range over
events. We use $proc(e)$ to denote the process $p$ of an event $e = \langle p, i \rangle$.
If $E.w$ is an execution sequence, obtained by concatenating $E$ and
$w$, then $dom_{[E]}(w)$ denotes $dom(E.w) \ dom(E)$, i.e. the events in
$E.w$ which are in $w$. As a special case, we use $next_{[E]}(p)$ to denote
$dom_{[E]}(p)$.
We use $<_E$ to denote the total order between events in $E$, i.e.
$e <_E e'$  denotes that $e$ occurs before $e'$  in $E$. We use $E'\leq E$ to
denote that the sequence $E'$ is a prefix of the sequence $E$.

\section{Event Dependencies}

One of the most important concepts when we have to deal with an algorithm that searches the whole state space of the different schedulings is the 
happens-before relation in an execution sequence. Usually this relation is denoted with $\rightarrow$ symbol. For example, if the relation $\rightarrow$ 
for two events $e,e'$ in $dom(E)$ holds true then the event $e$ happens-before $e'$. This relation usually appears in the message exchange, when $e$ is the message
transmission and $e'$ is the event when the message is received. For the context of Nidhugg $e \rightarrow e'$ would hold true when at least one of the two events
is a write operation on the same shared variable. It is fathomable that any DPOR algorithm should be able to assign this happens-before relations. 
In practice, the happens-before assignment is implemented with the use of vector clocks.

\begin{definition}{(happens-before assignment)}
    A happens-before assignment, which assigns a
    unique happens-before relation $\rightarrow E$ to any execution sequence
    $E$, is valid if it satisfies the following properties for all execution
    sequences $E$.
    \begin{enumerate}
        \item $\rightarrow_{E}$ is a partial order on $dom(E)$, which is included in $<_E$. In other words every scheduling is part of the set of all possible
        partial order of the program.
        \item The execution steps of each process are totally ordered, i.e. 
        $\langle p,i \rangle \rightarrow_E \langle p,i+1 \rangle$ whenever $\langle p, i+1 \rangle \in dom(E)$.
        \item If $E'$ is a prefix of $E$ then $\rightarrow_E$ and $\rightarrow_{E'}$ are the same on $dom(E')$.
        \item Any linearization $E'$ of $\rightarrow_E$ on $dom(E)$ is an execution sequence which has exactly the same “happens-before” relation
$\rightarrow_{E'}$ as $\rightarrow_E$. This means that the relation $\rightarrow_E$ induces a set
of equivalent execution sequences, all with the same “happens-before” relation. 
We use $E \simeq E'$ to denote that $E$ and $E'$ are
linearizations of the same “happens-before” relation, and $[E] \simeq$ 
to denote the equivalence class of E.
    \item If $E \simeq E'$ then $s_{[E]} = s_{[E']}$ (i.e. two equivalent traces will lead to the same state).
    \item For any sequences $E, E'$ and $w$, such that $E.w$ is an execution
sequence, we have $E \simeq E'$  if and only if $E.w \simeq' E'.w$.
    \end{enumerate}
\end{definition}

The first six properties should be obvious for any reasonable
happens-before relation. The only non-obvious is the
last. Intuitively, if the next step of p happens before the next step
of $r$ after the sequence $E$, then the step of $p$ still happens before
the step of $r$ even when some step of another process, which is not
dependent with $p$, is inserted between $p$ and $r$. This property holds
in any reasonable computation model that we can think of. As
examples, one situation is when $p$ and $q$ read a shared variable that
is written by $r$. Another situation is that $p$ sends a message that is
received by $r$. If an intervening process $q$ is independent with $p$, it
cannot affect this message, and so $r$ still receives the same message.
Properties 4 and 5 together imply, as a special case, that if $e$
and $e'$ are two consecutive events in E with $e \not \rightarrow_{E} e'$, then they can
be swapped and the (global) state after the two events remains the
same.

\section{Independence and races}

We now define independence between events of a computation. If
$E.p$ and $E.w$ are both execution sequences, then $E \models p\diamondsuit w$ denotes
that $E.p.w$ is an execution sequence such that $next_{[E]}(p) \not \rightarrow_{E.p.w} e$
for any $e \in dom([E.p])(w)$. In other words, $E \models p \diamondsuit w$ states that
the next event of $p$ would not “happen before” any event in $w$
in the execution sequence $E.p.w$. Intuitively, it means that $p$ is
independent with $w$ after $E$. In the special case when $w$ contains
only one process $q$, then $E \models p \diamondsuit q$ denotes that the next steps of
$p$ and $q$ are independent after $E$. We use $E'\models p \diamondsuit w$ to denote that
$E \not \models p \diamondsuit w$ does not hold.

For a sequence $w$ and $p \in w$, let $w \backslash p$ denote the sequence
$w$ with its first occurrence of $p$ removed, and let $w \uparrow p$ denote the
prefix of w up to but not including the first occurrence of $p$. For
an execution sequence $E$ and an event $e \in  dom(E)$, let $pre(E,e)$
denote the prefix of $E$ up to, but not including, the event $e$. For an
execution sequence $E$ and an event $e \in E$, let $notdep(e, E)$ be the
sub-sequence of $E$ consisting of the events that occur after $e$ but do
not “happen after” $e$ (i.e., the events $e'$ that occur after $e$ such that
$e \not \rightarrow_E e'$).


A central concept in most DPOR algorithms is that of a race.
Intuitively, two events, $e$ and $e'$ in an execution sequence $E$, where
$e$ occurs before $e'$ in $E$, are in a race if
\begin{itemize}
\item $e$ happens-before $e'$ in $E$, and
\item $e$ and $e'$ are “concurrent”, i.e. there is an equivalent execution
sequence $E' \simeq E$ in which $e$ and $e'$ are adjacent.
\end{itemize}
Formally, let $e \lessdot_E e'$ denote that $proc(e) \not = proc(e')$, that $e \rightarrow_E e'$,
and that there is no event $e'' \in dom(E)$, different from $e'$ and $e$,
such that $e \rightarrow_E e'' \rightarrow_E e'$.

Whenever a DPOR algorithm detects a race, then it will check
whether the events in the race can be executed in the reverse order.
Since the events are related by the happens-before relation, this may
lead to a different global state: therefore the algorithm must try to
explore a corresponding execution sequence. Let $e \lesssim_E e'$ denote
that $e \lessdot_E e'$, and that the race can be reversed. Formally, if $E' \lesssim E$
and $e$ occurs immediately before $e'$ in $E'$, then $proc(e')$ was not
blocked before the occurrence of $e$.


\section{Dynamic partial order reduction}

Before explaining the DPOR algorithm it is important to define sufficient sets.

\begin{definition}{(Sufficient Sets)}
A set of transitions is sufficient in a state $s$ if any relevant
state reachable via an enabled transition from s is also reachable from $s$ via at least one of the transitions in the sufficient
set. A search can thus explore only the transitions in the
sufficient set from s because all relevant states still remain
reachable. The set containing all enabled threads is trivially
sufficient in $s$, but smaller sufficient sets enable more state
space reduction.
\end{definition}

Many techniques have been proposed in order to implement a DPOR algorithm. What most of these techniques share in common is the following basic structure:
\SetKwProg{Fn}{Function}{}{}

\SetKwHangingKw{Let}{let}
\begin{algorithm}[H]
    \caption{General form of DPOR}
    Explore($\emptyset$)\;
    \Fn{Explore($E$)}{
     \Let{T = Sufficient\_set($final(E)$)}
     \For{all $t \in T$}{
        Explore($E.t$) \;
    }
    }
\end{algorithm}

where $final(E)$ represents the state that will be reached when the execution sequence $E$ is executed.

The algorithm above describes a DFS search in the state space of all possible interleavings.
As it can be inferred from the algorithm the most important step is that of the calculation of the set $T$.

\begin{definition}{(Enabled sets, $enabled(s)$)}
    Given a state $s$, $enabled(s)$ represents the set of all the threads that can be scheduled immediately after $s$.
\end{definition}

An obvious property that the sufficient sets must hold is that Sufficient\_set$(final(E)) \subseteq enabled(E)$.

Intuitively $enabled(s)$ represents the threads that are not blocked or have already finished their execution.

In bibliography many types of sufficient sets can be found \cite{Godefroid1996}. 
In this thesis we mainly focus on persistent sets and on source sets.


\section{Persistent Sets}

A persistent set in a state $s$ is a sufficient set of transitions to
explore from $s$ while maintaining local state reachability for acyclic state spaces \cite{God97}. A selective search using persistent
sets explores a persistent set of transitions from each state s where $enabled(s) \neq \emptyset$ and prunes enabled transitions that
are not persistent in s.
In a more formal way:\\

\begin{definition}{(Persistent Sets)}
Let $s$ be a state, and let $W \subseteq E(s)$ be a set
of execution sequences from $s$. A set $T$ of transitions is a persistent set for $W$
after $s$ if for each prefix $w$ of some sequence in $W$, which contains no occurrence
of a transition in $T$,  we have $E \vdash t \diamondsuit w$ for each $t \in T$.
\end{definition}

The above definition can be described as follows: If $t \in T$ and there is another thread $t'$ that can be executed until a command which
is in a race with $t$, then $t'$ belongs in the persistent set.

Notice that the definition of persistent sets suggests a way to construct them.

In Figure \ref{Construction of persistent sets} two different examples of persistent set construction are given. We denote the persistent set of branches the execution will take with $BR{}$.
In the first, let a concurrent program contain 3 threads $p$, $q$, and $r$. Thread $p$ changes the value of the variable (writer) and the other ($q$ and $r$) just read this variable (readers).
Let $p.q.r$ be an interleaving. According to the definition of the persistent sets $q$ and $r$ are in a race with $p$, thus, $q$ and $r$ must also be on the persistent set
of the first command of the interleaving. In Figure \ref{Construction of persistent set} we notice that both $r$ and $q$ threads are added to the persistent set of the first
command of the trace since both conflict with the write operation. 
In the second example, let $p$ and $r$  be a readers and $q$ be a writer. We notice that both $r$ and $q$ are added. However, there is no conflict between $p$ and $r$ since both $p$ and $r$
just read the variable $x$. The reason why the thread $r$ is added is the conflict that will be produced by the $q$'s write operation.

\trace{persistent.pdf}{Construction of persistent set}

\section{Source sets}

Before defining source sets, we give some other useful definitions.

\begin{definition}{($dom(E)$)}
    The set of events-transitions happening during the scheduling of $E$.
\end{definition}

\begin{definition}{(Initials after an execution sequence $E.w$, $I_{[E]}(w)$)}
For an execution sequence $E.w$, let $I_{[E]}(w)$ denote the set of
processes that perform events $e$ in $dom_{[E]}(w)$ that have no
“happens-before” predecessors in $dom_{[E]}(w)$. More formally,
$p \in I_{[E]}(w)$ if $p \in w$ and there is no other event $e \in dom_{[E]}(w)$ with
$e \rightarrow_{E.w} next_{[E]}(p)$.
\end{definition}

\begin{definition}{(Source Sets)}
Let $S$ be an execution sequence,
and let $W$ be a set of sequences, such that $E.w$ is an execution
sequence for each $w \in W$. A set $T$ of processes is a source set for
$W$ after $E$ if for each $w \in W$ we have $WI_{[E]}(w) \cap P  = \emptyset$.
\end{definition}

A source set is a set of threads that guarantee that the whole state space will be explored. Notice that their is no requirement related to the races
of the events.
What the above definition implies is that source can be considered every set of threads that contains these threads that are able to cover the whole state-space.
It actually suggests a property for the sufficient sets to hold.

\section{Sleep sets}

Another technique complementary to the persistent or source sets aiming to reduce the number of interleavings is the sleep set technique.
Sleep sets prohibit visited transitions from executing again
until the search explores a dependent transition. Assume that
the search explores transition $t$ from state $s$, backtracks $t$,
then explores $t_0$ from $s$ instead. Unless the search explores
a transition that is dependent with $t$, no states are reachable
via $t_0$ that were not already reachable via t from s. Thus, t
“sleeps” unless a dependent transition is explored.

A short example on sleep sets is the following:
Let us the concurrent program of one writer and two readers.
let w1 <0.0>: w(x) r1 <0.1>: (local operations), r(x) and r2 <0.2>: (local operations), r(x).

The resulted traces are demonstrated in the Listing \ref{Sleep set example}.

\Output{./code/sleep_sets.out}{Sleep set example}

As we can see from the execution of the DPOR algorithm the interleaving which started from r2 was blocked since it would lead to an interleaving which
has already been explored. Notice that this is due to the fact that r1 cannot wakeup since its first transition (local operations) does not conflict with any other transition
in the program. 
It can be proved \cite{Godefroid1996} that sleeps will eventually block all the redundant interleavings and thus the only interleavings that will be explored till their end (where all threads that could be executed, have been executed).
As a result an optimal algorithm should be able to not consider these interleavings whatsoever.

\section{Comparing Persistent sets with Source Sets}

Note that the definition of source sets is much more relaxed than the definition of the persistent sets. 
This relaxation enables the source sets to be much more efficient than the persistent sets. In Figure \ref{Non-minimal persistent sets}
an example is given were source sets and persistent sets differ.

\begin{figure*}
    \begin{lstlisting}[frame=none,numbers=none]
        Initially: x = y = z = 0 
    \end{lstlisting}
    \begin{minipage}{0.3\textwidth}
      \begin{lstlisting}[frame=none, numbers=none]
        p:
        m := x; (p1)
        if (m = 0) then
            z := 1; (p2)
      \end{lstlisting}
    \end{minipage}
    \begin{minipage}{0.3\textwidth}
        \begin{lstlisting}[frame=none, numbers=none]
            q:
            n := y; (q1)
            if (n = 0) then
                x := 1; (q2)
        \end{lstlisting}
      \end{minipage}
      \begin{minipage}{0.3\textwidth}
        \begin{lstlisting}[frame=none, numbers=none]
            r:
            o := z; (r1)
            if (o = 0) then
                y := 1; (r2)
        \end{lstlisting}
      \end{minipage}
      \caption{Program with non-minimal persistent sets}
      \label{Non-minimal persistent sets}
  \end{figure*}

From the example, it is clear that the reason why source sets are an improvement over persistent sets is the fact that minimum source sets can eliminate
sleep set blocked traces i.e. traces that would eventually be blocked by the sleep sets. An algorithm that would only calculate minimal source sets would be optimal \cite{AbdullaAronisJohnssonSagonasDPOR2014}, hence
would never explore two equivalent interleavings.

It is obvious that a single transition cannot be a source set. For
instance, the set $\{ p_1 \}$ does not contain the initials of execution $q_1.q_2.p_1.r_1.r_2$,
since q2 and p1 perform conflicting accesses. On the other hand, any subset
containing two enabled transitions is a source set. To see this, let us choose
$\{p_1, q_1 \}$ as the source set. Obviously, $\{p1, q1 \}$ contains an initial of any execution
that starts with either $p_1$ or $q_1$. Any execution sequence which starts with $r_1$ is
equivalent to an execution obtained by moving the first step of either $p_1$ or $q_1$ to
the beginning:
\begin{itemize}
\item If $q_1$ occurs before $r_2$, then $q_1$ is an initial, since it does not conflict with
any other transition.
\item If $q_1$ occurs after $r_2$, then $p_1$ is independent of all steps, so $p_1$ is an initial.
We claim that $\{p_1, q_1 \}$ cannot be a persistent set. The reason is that the execution
sequence $\{r_1.r_2 \}$ does not contain any transition in the persistent set, but its second
step is dependent with $q_1$. By symmetry, it follows that no other two-transition
set can be a persistent set.
\end{itemize}

In other words, persistent sets have the unpleasant property that adding a process
may disturb the persistent set so that even more process may have to be added.
This property is relevant in the context of DPOR, where the first member of the
persistent set is often chosen rather arbitrarily (it is the next process in the first
exploration after $E$), and where the persistent set is expanded by need.

Continuing the comparison between source sets and persistent sets, we first
note some rather direct properties, including the following.

\begin{itemize}
\item Any persistent set is a source set.
\item Any one-process source set is a persistent set.
\end{itemize}


\section{Bounded search - preemption bounded search}
Bounded search explores only executions that do not exceed
a bound \cite{BPOR,Thomson}. The bound may be any property of a
sequence of transitions. A bound evaluation function $Bv(S)$
computes the bounded value for a sequence of transitions S.
A bound evaluation function $B_v$ and bound $c$ are inputs to
bounded search. Bounded search may not visit all relevant
reachable states; it visits only those that are reachable within
the bound. If a search explores all relevant states reachable
within the bound, then it provides bounded coverage.

An algorithm that could describe a bounded search would be the following:

\begin{algorithm}[H]
    \caption{Bounded-DPOR}
    \KwResult{Explore the whole statespace}
    Explore($\emptyset$)\;
    \Fn{Explore($S$)}{
        T = Sufficient\_set($final(S)$)
     \For{all $t \in T$}{
         \If{$Bv(S.t) \leq c$}{
            Explore($S.t$)
         }
        }
    }
\end{algorithm}

\noindent The only difference between the unbounded and the bounded version of the algorithm is the if statement on line 4 which allows for an interleaving to be explored
only if the bound has not been exceeded.

What is needed next is an appropriate definition of the function $B_v$ that calculates a value that the bounded-DPOR tries to keep bounded, 
and the sufficient set. 

In this thesis, we mainly focus on preemption-bounded search. 

Preemption-bounded search limits the number of preemptive context switches that occur in an execution \cite{Musu07}. The 
preemption bound is defined recursively as follows.

\begin{definition}{Preemption bound}
$P_b(t) = 0$ \\
$P_b(S.t) = 
 \begin{cases} 
    P_b(S) + 1 & \text{ if } t.tid = last(S).tid \text{ and } last(S).tid \in enabled(final(S)) \\
    P_b(S) & \text{ otherwise }
 \end{cases}
$\\
\end{definition}

The previous definition describes what a preemptive context switch is. A preemptive context switch happens when the previously running thread could execute
its next step but it does not due to the scheduling of another thread. Hence, a preemptive switch will increase the preemption bound.

\section{Preemption-bound persistent sets}

A set that has been proposed as a sufficient for preemption bounded search is the preemption bounded persistent set \cite{BPOR}.

An important observation is that the execution of a thread until it gets blocked or terminates will not increase
the bound count.
\begin{definition}{($ext(s,t)$)}
    Given a state $s = final(S)$ and a transition $t \in enabled(s)$,
    $ext(s,t)$ returns the unique sequence of transitions $\beta$ from $s$
    such that
    \begin{enumerate}
        \item $\forall i \in dom(\beta): \beta_i.tid = t.tid$
        \item $t.tid \notin enabled(final(S.\beta))$
    \end{enumerate}
\end{definition}

Next, we need to define preemption bounded persistent sets. We denote with $A_G(P_b,c)$ the generic 
bounded state space with bound function $P_b$ and bound $c$. $last(a)$ denotes the last execution step of
an execution sequence $a$

\begin{definition}{(Preemption bounded persistent set)}

A set $T \subseteq \mathcal{T}$ of transitions enabled in a state $s=final(S)$
is preemption-bound persistent in $s$ iff for all nonempty
sequences $a$ of transitions from $s$ in $A_G(P_b,c)$ such that
$\forall i \in dom(a), a_i \notin T$ for all $t \in T$ ,

\begin{enumerate}
\item $Pb(S.t) \leq Pb(S.a_1)$
\item if $Pb(S.t)<Pb(S.a_{1}) ,$ then $t \leftrightarrow last(a)$ and $t \leftrightarrow  next(final(S.a), last(a).tid)$
\item if $Pb(S.t)=Pb(S.a_{1}),$ then $ext(s,t) \leftrightarrow last(a)$ and $ext(s,t) \leftrightarrow next(final(S.a), last(a).tid)$
\end{enumerate}

\end{definition}

When dealing with preemption bounded DPOR it is useful to introduce the idea of blocks in an execution sequence.

\begin{definition}{(Block of execution sequence}
    Block in an execution sequence is the maximal subsequence of execution steps that consists of execution steps of the same thread.
\end{definition}

In the following example there are three blocks. The first block is coloured with blue, the second with green and the third with yellow.

Let us assume that $P$ is a persistent set. A preemption bounded persistent set is a set that contains all $p \in P$ with the addition of all the 
threads that would be added in a block that would be created when $p$ was scheduled. These threads are called conservative threads and their 
goal is to allow the coverage of interleavings that would not exceed the bound. Notice that an interleaving can be both conservative and non-conservative.
Preemption bounded persistent sets extend a persistent set by adding all the threads that will create a new block
after the block that will be created by the persistent set.


\input{./ch3.tex}
\chapter{Evaluation of Bounding Techniques}
\label{Chapter 4}

In this chapter, the performance of each implemented technique will be discussed. Firstly, the performance of classic-DPOR is demonstrated in order to prove
its performance, indeed differs from Source-DPOR. The evaluation happens in two parts. In the first part, short synthetic programs are used, while in the second part
real world software is tested. Synthetic programs can be found in the appendix section. One area where Nidhugg is tested is the verification of the Read Copy Update
technique of the Linux kernel.


\section{Synthetic Tests}
There are many tests provided from various sources. Most of these testcases are not complicated at all since their purpose is to demonstrate the performance
difference of the Source-DPOR and classic-DPOR.

\begin{itemize}
\item The writer-Nreaders test: In this test N threads read (readers) the same global variable and one thread (writer) writes that variable. It is important 
to notice that in this case there are some other local operations taking place before the read of the variable. As a result we must expect different results between 
source sets and persistent sets.

\item Account: This test is a small bank account simulation which uses mutex locks to prevent simultaneous operations on the account.
There are three possible operations: The deposit operation increases the balance by an amount. The withdraw operation decreases the balance by a certain amount. The check\_result operation confirms 
$\text{final\_balance} == \text{initial\_balance} + \text{deposit} - \text{withdraw}$ and can only happen after both deposit and withdraw are completed.

\item Micro: In this test three threads are spawned that perform the \verb|x++| operation twice. The \verb|x++| operation
consists of two operations a read operation and a write operation.

\item Last-zero test:The program consists of N+1 threads which operate on an array of
N+1 elements which are all initially zero. In this program, thread 0
searches the array for the zero element with the highest index, while
the other N threads read one of the array elements and update the
next one. The final state of the program is uniquely defined by the
values of i and array[1..N]. Last-zero does not produce more traces when DPOR is used for reasons
that will be explained later. However a modification of the .ll file can expose the difference.

\item Indexer.c: This benchmark uses a compare-and-swap(CAS) primitive instruction to check
whether a specific entry in a matrix is 0 and set it to a new
value. 

\item Indexermod.c: In this benchmark all the threads traverse and try to write the matrix at the same
order and as a result many conflicts emerge.


\end{itemize}

\section{RCU}

Read-Copy-Update (RCU) is a synchronization mechanism, which was invented by McKenney and Slingwine \cite{McKenney98}, based on mutual exclusion.
It was added to the Linux kernel in October of 2002. RCU achieves scalability improvements by allowing reads to occur concurrently 
with updates. In contrast with conventional locking primitives that ensure mutual exclusion among concurrent threads regardless of whether 
they be readers or updaters, or with reader-writer locks that allow concurrent reads but not in the presence of updates, 
RCU supports concurrency between a single updater and multiple readers. 

DPOR was used as an approach to systematically test the code
of the main flavor of RCU used in the Linux kernel (Tree RCU) for
concurrency errors, under sequential consistency. 
The modeling allows Nidhugg, a stateless model checking
tool, to reproduce, within seconds, safety and liveness bugs that
have been reported for RCU \cite{Spin}.

RCU provides an ideal testcase to evaluate the various DPOR and Bounded DPOR algorithms since it is:
\begin{itemize}
\item It is a real world software and not just a synthetic test.
\item The number of traces (different schedulings) is large enough to demonstrate the differences in the performance.
\item Previous work \cite{Spin} enables us to evaluate the correctness of each algorithm's implementation.
\end{itemize}


\section{Evaluation of Persistent Sets}
As it was established in the previous chapter the implementation of the persistent sets
is crucial since they are utilized in every bounding technique. 
In this section we demonstrate performance differences between Source-DPOR and classic-DPOR in both synthetic tests and RCU.

\subsection{Evaluation of Persistent sets on Synthetic tests}
The execution of the synthetic test cases delineated that Source-DPOR is indeed an improvement over Classic-DPOR. As it was expected source-DPOR explores less 
traces than the DPOR. It is important
to notice that this difference is caused by the sleep set blocked traces that are produced by the DPOR algorithm that are omitted by the source DPOR. 
The reduction in the number of traces explored is not the same of all the testcases. For example in some testcase there is a
significant decrease of the explored traces while in others the reduction is not so great. 
It varies due to the different approaches as well as the size of the state space. The results are presented with two different ways. 
The writer-N-readers testcase result is given with a graph, in Figure \ref{writer-N-readers}, in order to demonstrate the escalation of the state space as well as the greater impact the source-DPOR has. The rest of the
results are given in Table \ref{Source-DPOR vs DPOR for synthetic tests} so they can be easily compared.

\graph{/home/yannis/nidhugg/tests/mytests/wNr.png}{writer-N-readers}

\smalltabular{"tables/synthetic_unbounded.tex"}{Source-DPOR vs Classic-DPOR for synthetic tests}

\subsection{Evaluation of Persistent sets on RCU}
We noticed that there is no difference between Source sets and persistent sets thus no results are presented since they coincide with \cite{Spin}. 
The reason why the results of DPOR and Source-DPOR are the same may be due to the operations that take place which not allow for the optimization of the Source-DPOR 
to be effective. Another reason is the LLVM IR which is used by Nidhugg and will be explained in the next section.

\section{Comparison with Concuerror results - Why DPOR may be enough for LLVM}
The are many cases where Concuerror's Source-DPOR explores less traces than Classic-DPOR while Nidhugg's Source-DPOR doesn't. 
In fact Nidhugg's Classic-DPOR implementation seems to explore less number of traces than Concuerror's Classic-DPOR \cite{AbdullaAronisJohnssonSagonasDPOR2014}.
which equals with the numbers of traces explored by Source-DPOR. However, a look on the code that Nidhugg tests justifies this behavior.
LLVM produces much more code than the code in the source file. Due to this extra code that is added, the number of conflicting events is less than the 
number of conflicting events in an Erlang program.

At Figure \ref{Padding impact on persistent sets} the reason is visualized. As shown in the figure we would expect that both p and r would be added in the persistent set. However when the p thread is added the write event is no longer visible, 
according to the persistent set definition. As a the r thread is not added. We can compare the LLVM code with the relative Erlang code.

\trace{rwrpersistent.pdf}{Padding impact on persistent sets}

\Code{./code/readers_rwr.erl}{Erlang code for rwr}
%%\Code{"./code/rwr.c"}{C code for rwr}
%%\Code{"./code/rwr.ll"}{Produced LLVM code for rwr}
\Side{./code/rwr.c}{C code for writer and reader}{./code/rwr.ll}{LLVM code for writer and reader}{Comparison between C and LLVM}


\section{Evaluation of Bounding Techniques}
The evaluation of the techniques takes into account two aspects. The number of traces explored and the soundness. The former is closely related with the amount
of time required for a bug to be found or the whole state space to be explored. The second is important because it demonstrates the tradeoff between time and accuracy
of the results. It is intelligible that a faster algorithm may compromise the soundness of the state space.
\subsection{Evaluation of Bounding Techniques on Synthetic tests}

The results for the testcases are demonstrated below. Again they are presented in two different ways.

%%\begin{center}
%%    \begin{tabular}{ |c|c|c|c|c|c|c|}
%%    \hline
%%    \multicolumn{1}{|c|}{Technique:} & \multicolumn{2}{c|}{Vanilla-BPOR} & \multicolumn{2}{c|}{BPOR} & \multicolumn{2}{c|}{Source-BPOR} \\
%%    \hline
%%    Bound: & 0 & 1 & 0 & 1 & 0 & 1 \\
%%    \hline \hline
%%    account.c & 1 & 6 & 6 & 1 & 27 & 27 \\
%%    \hline
%%    lazy.c & 1 & 6 & 6 & 1 & 27 & 27 \\
%%    \hline
%%    \end{tabular}
%%\end{center}
\graph{img/wNrB.png}{writer-N-readers bounded}
\smalltabular{"/home/yannis/nidhugg/tests/mytests/bounded.tex"}{Traces for various bound limits}

As it was expected the Naive-BPOR explores significantly less traces than the BPOR and the source-DPOR. However, as it was previously discussed, the whole
state space is not explored. The number of traces explored by the sound algorithms is significantly greater and it caused by the many conservative branches that are
added in order to achieve soundness. Surprisingly, there is no difference between the other two bounding techniques. An explanation is given later.

\subsection{Evaluation of Bounding Techniques on RCU}
The results are demonstrated below. Notice that since the Source-DPOR did not resulted less traces than the DPOR we could not expect from the Source-BPOR and BPOR
to differentiate. Moreover tests did not show any difference. For these reasons only the performance of Naive-BPOR and BPOR is examined. In each table
the results with a given bound are demonstrated. Specifically the exploration time and the number of traces are shown. 
Moreover there is a cell indicating whether the assertion was found (We note F for found and NF for not found).


\bigtabular{"/home/yannis/rcu/valtree/nobound.tex"}{RCU results without bound}
\bigtabular{"tables/naivevsbpor0.tex"}{RCU results for bound $b=0$}
\bigtabular{"tables/naivevsbpor1.tex"}{RCU results for bound $b=1$}
\bigtabular{"tables/naivevsbpor2.tex"}{RCU results for bound $b=2$}
\bigtabular{"tables/naivevsbpor3.tex"}{RCU results for bound $b=3$}
\bigtabular{"tables/naivevsbpor4.tex"}{RCU results for bound $b=4$}
\bigtabular{"tables/dporvsbpor.tex"}{Comparison between DPOR and BPOR}


We notice that some assertions are found significantly faster. The most spectacular result is the \verb|-DFORCE_FAILURE_3| which is found in only 6 seconds for bound b=3 whereas it requires 464.77 seconds in the unbounded version. Moreover we notice for bound b=4 all the errors that are found in the unbounded version are
found. As a result the empirical observation that errors occur in low bound count seems to be confirmed. However, we have to underline that these are contrived
errors aiming to verify the correctness of the rcu and as a result they cannot be regarded as substantial evidences. As it is expected for larger bounds (b=4) the number of traces
grows exponentially. An other impressive result is that when the bound grows larger the errors takes longer to be found. If we take a look at \verb|-DFORCE_FAILURE_3| again we notice that the error
takes significantly longer to be tracked even through it exposed for the first time at bound b=2. For b=4 the exploration will was stopped since it exceeded 100,000 traces.
On the other hand many assertions are found faster with source-DPOR.

\subsection{A known bug}
As it was discussed in previous section, the scheduling priorities of Nidhugg should be changed in order for the running thread to be prioritize since it does not
increase the bound count. However this alternation in the priority causes Nidhugg to explore many more traces in unbounded search for an unknown reason. In order to deal with
this problem alternation in priority occurs only when bound is applied. As a result the comparison between DPOR and BPOR is not fair. Looking at table \ref{Comparison between DPOR and BPOR with the bug}
we can clearly see that the minimum traces required for BPOR to track the bug for the first time are always less than DPOR

\bigtabular{"tables/dporvsbporpriority.tex"}{Comparison between DPOR and BPOR with the bug}

\section{Equivalence between BPOR and Source-BPOR (Correctness of Source-BPOR)}
Surprisingly the results of BPOR and Source-BPOR always coincide. However, further investigation of this behavior can reveal that these two techniques
are actually equivalent. 

It can be proved that a branch which was rejected by the Source-DPOR but accepted by the Classic-BPOR algorithm as a non-conservative one will be added as 
conservative by the source-bpor algorithm.

Let us assume a branch of the thread $b$ that is added as a non-conservative by the BPOR algorithm. Let $T$ be the persistent set at that point.

By the definition of persistent-sets this means that there is a $t \in T$ which
conflicts with an execution step of $b$. 

This non-conservative branch is rejected by the Source-BPOR. We know that there must be
 a trace such that thread $b$ occurs before $t$. Since $b$ was rejected there must be another branch $s$ which shares the same initials
  with $b$, 
  
When $s$ is scheduled another block will be created.
 
\begin{itemize}

\item Case 1: $s$ has an execution step which conflicts with $b$. Hence, there must be a trace where $b$ happens before some step of $s$ and $s$ 
happens before $t$. Since $b$ happens-before some execution step of $s$ but share the same initials with $s$, $b$ must be added as a conservative
branch at the point where it was rejected at the initially.
As shown in the Figure \ref{Source-BPOR and BPOR equivalence Case 1}, the branch which seems to be initially rejected, 
will finally be added by the Source-BPOR and as a result belongs to the source-set. 
\trace{equivalence_case1w.pdf}{Source-BPOR and BPOR equivalence Case 1}
   
\item Case 2: 
   $s$ doesn’t conflict with $b$ (both $b$ and $s$ are read operations). There must $b$ a trace s.b.t (where s,b,t is the execution
   of all the steps of s,b,t). 
   Since $t$ conflicts with an execution step of $s$ the first step of $b$ is an initial for $t$ and it will be added both as non-conservative branch and 
   as conservative at the beginning of the block where it was rejected by the Source-DPOR. For Figure \ref{Source-BPOR and BPOR equivalence Case 2}, both $s$ and $b$ belong to the persistent set. However,
   the $b$ thread will be rejected since it shares the same initials with the $s$ thread. However it will be added as a conservative set. Notice that it would be added as a
   non-conservative as well but we have already shown that when both conservative and non-conservative branches of the same thread are added we must keep the conservative one.

   \trace{equivalence_case2.pdf}{Source-BPOR and BPOR equivalence Case 2}
\end{itemize}
   
A more intuitive explanation of the equivalence of the two techniques would be based on the following two observation:
\begin{itemize}
  \item Let $B_v$ be a function that calculates the bound count then $B_v(pre(E,e)) \leq B_v(E)$ for every $e \in E$.
  \item The points in the trace where the bound count increases are those where branches are added.
\end{itemize}

As a result a BPOR algorithm would have to add conservative branches at points where the bound increases. The non-conservative branches that would have been
rejected by the Source-DPOR are added as conservative ones to since the lead to already explored traces but with a smaller bound count.

We have proved that Source-BPOR is sound since the traces explored by the BPOR are subset of the traces explored by the Source-BPOR.

\chapter{Further Discussion on Bounding Problem}
\label{Chapter 5}

In this chapter alternative ideas of approaching the preemption bounding problem of the DPOR are discussed. It is shown that optimizations that have already been used for the DPOR algorithm cannot solve the problem.
Finally a new approach is suggested which is shown to be equivalent to the addition of conservative branches. This approach however can be used to better approximate
a sound solution of the problem. 

\section{Techniques without the Addition of Conservative Branches}

It was shown that no apparent significant improvement can be made with the use of conservative branches. In this section, techniques without the usage of 
conservative branches are discussed.

\subsection{Motivation}
The only algorithm that does not add any conservative branch is the Vanilla-BPOR. For a sufficient bound an erroneous trace would have still be found using this technique.
The drawback of this algorithm is its unsoundness. In this algorithm a function which calculates the number of preemptive switches in the current thread is used.
However many of the preemptive switches that are counted would be avoided.

An example is given further explaining this idea.

\trace{motivation.pdf}{Motivation}

As it is clear the preemptive switch that takes places would have been easily avoided there is an obvious inversion of the two blocks.

But what allows such an inversion?

The answer lies to the events of each block. The first block reads a variable which is not used by any other block. It is fathomable that this block can be switched with the
next block since there is no a happen before relationship with the two blocks.

This observation leads to the next question: Which of the preemption switches are compulsory? (Or equivalently which traces cannot be produced without a preemption switch?)
Moreover is it possible for a given trace to calculate the minimum number of preemptive switches among all the equivalent traces?

\subsection{An Algorithm without Conservative Branches}
An algorithm that would preform such a bounded search would be different from the Vanilla-BPOR only concerning the function calculates the bound count of the traces.
This function of would have to be constantly ascending i.e. it would not be possible to calculate a lower bound later for the same traces.
Given a suffix $E$, $f(E) <= f(E.E')$ for any $E'$.

The general form of the algorithm is given below:

\begin{algorithm}[H]
    \SetAlgoLined
    \caption{General form of the BPOR without branch addition}
    \KwResult{Explore the whole state space within the bound}
    Explore($\emptyset$)\;
    \Fn{Explore($S$)}{
        T = Sufficient\_set($final(S)$)
     \For{all $t \in T$}{
         \If{$min\{B_v([S.t])\} \leq c$}{
            Explore($S.t$)
         }
        }
    }
\end{algorithm}

We notice that instead of calculating the $Bv$ value we calculate minimum of all $Bv$ values of the traces that are equivalent with $S.t$.

\subsection{Calculating Minimum Bound Count}
The only thing left is the construction of this function f.
For a given trace $E$ which consists of blocks many happen-before relations hold. Each equivalent trace should also compensate to these relations.
It is also possible for different instructions in one block different happen before relations hold true. For this section only we will consider that 
a happen before relation is a relation that happens between blocks. This is done for two main reasons:

\begin{itemize}
    \item The algorithm described later is simplified.
    \item We are not interested in further breaking each block and as a result we can regard is block as an entity.
\end{itemize}

The existence of these happen before relations imply the existence of a graph. This graph consists of nodes which are the blocks and edges which are these
relations. Obviously blocks of the same thread have a happen before relation. We can also move from one block to another as long as these blocks happen
concurrently. We add weights to each edge. The edges that connect to blocks of the same thread weigh 0. Edges that start from a block that 
is blocked or the most recently added block of each thread weigh 0 since blocked blocks do not increase the bound count and we do not know if the last block of
each thread is indeed the last one. All the other edges which represent preemptive switch have weight 1.

The construction of the graph would not allow to traverse a block A that happens-before B before B, thus, all the happen before relations should still hold true.
All traversals that cover the whole graph passing from each node only once are equivalent traces.

\noindent An algorithm on how to add a block to a given graph is given at \ref{Adding a new block to the dependencies graph}. The algorithm works using induction.
Initially the graph consists of the first block. When a block of the trace is completed then we add it to the dependency graph. We connect the new block with each 
concurrent block with double edges with the new block. Moreover we connect the most recent block of each thread that happens before the new block with a directed edge
ending to the new block. 

\begin{algorithm}[H]
    \caption{Adding a new block to the dependencies' graph}
    \label{Adding a new block to the dependencies graph}
    \Fn{AddBlock(block,graph)}{
        \If{previous block of the same thread was not blocked}{
            increase the weigh of the edges coming from the previous block to 1 \;
        }

        \For{each thread t}{
            list:= preceding blocks t\;
            \For{l in reversed(list)}{
                \If{$l \leftrightarrow block$}{
                    add edge from block to l with weight 0 \;
                    \If{$l$ is not last}{
                        add edge from l to block with weight 1 \;
                    }
                    \Else{
                        add edge from l to block with weight 0 \; 
                    }
                }
                \If{$l \rightarrow block$}{
                    \If{$l$ is not last}{
                        add edge from l to block with weight 1 \;
                    }
                    \Else{
                        add edge from l to block with weight 0 \; 
                    }
                    break \;
                }
            }
        }
    }
\end{algorithm}


\trace{compulsoryswitch.pdf}{Graph example}

In Figure \ref{Graph example} a simple example of such a graph is demonstrated. For this trace we notice that w(y) of q thread is concurrent with r(x) while 
it happens before w(y) of the p thread. Each transition costs 1 preemption switch that is why the weight is 1. Moreover transitions between the same thread cost 0.
The most important fact, however, is that if for any reason we try to violate the happen before relation (e.g. starting from r(x) we jump to w(x)) there is no way to 
traverse all the nodes. 

We can see that there is a hamiltonian path with weight 1 for the given trace. In fact, this is the minimum hamiltonian path of the graph. We can easily realize that 
there is no equivalent trace with the initial one that has bound count less than 1. 

We can infer that the calculation of this bound count is reduced to the weight of the minimum hamiltonian path of this graph. This problem it is known to be $NP-complete$. 
As a result any algorithm that would calculate this weight would not be significantly better than a DFS-exploration. 

This is an extremely interesting indication of the difficulty of the DPOR bounding problem since the addition of the conservative sets imply this DFS
exploration at the state space. As a result this algorithm would not be better than the already proposed BPOR algorithm.

Now that the difficulty of this approach is clear a new question arises. Is it possible to approximate the total weight of the minimum hamiltonian path?
Such an algorithm would cover a greater state space than the Vanilla-BPOR without the explosion caused by the conservative branches.

\subsection{Approximating Bound Count}
There are two approaches examined in order to approximate a value were considered. The notion of both algorithms is based on this observation: A preemption switch is compulsory
when for two blocks of the same thread A a block of another thread B must intervene in order for the happen before relations to hold true. As a result the execution of 
the first A block should stop so the execution of the B block take place followed by the execution of the A block again. Hence it should hold $e_1(A) \rightarrow e(B) \rightarrow e_2(A)$.
In case of $e_1(A) \not \rightarrow e(B)$ or $e(B) \not \rightarrow e_2(A)$ we could invert the blocks without affecting the happen before relations and, thus construct an
equivalent trace with lower bound count.

The algorithm is presented here:\\

\SetKwProg{Fn}{Function}{}{}
\begin{algorithm}[H]
    \caption{First Estimation Algorithm}
    \Fn{BoundCount($E$,$current\_bound$)}{
        \For{$i=0 \text{ to } len(E)-1$}{
            \If{$E[i].pid = last(E).pid$}{
                $higher\_block = i$ \;
                break \;
            }
        }
        \For{$i = higher\_block+1 \text{ to } len(E)-1$}{
            \If{$ E[higher\_block] \rightarrow E[i] \rightarrow last(E)$}{
                current\_bound++ \;
                return \;
            }
        }
    }
\end{algorithm}

In the above algorithm we find the most recent block with the same pid as with the last block. We, then try to find if there is an event that happens before the first
and after the last event. If exists such an event we increase the counter.
Notice that for establishing the happen before relation vector clocks can be used.
Moreover, it is obvious that more happen before relations can be counted.


Th second algorithm explores more state space than it is required.\\

\begin{algorithm}[H]
    \caption{Second Estimation Algorithm}
    \Fn{BoundCount($E$,$current\_bound$)}{
        \For{$i=len(E)-1 \text{ to } 0$}{
            \If{$E[i].pid = last(E).pid$}{
                $lower\_block = i$ \;
            }
        }

        \For{$i = lower\_block+1 \text{ to } len(E)-1$}{
            \If{$ E[lower\_block] \rightarrow E[i] \rightarrow last(E)$}{
                current\_bound++\;
                return \;
            }
        }
    }
\end{algorithm}

This algorithm starts the search for an event that intervenes the two events of the same the immediately previous block with the same thread as the last one.

\subsection{Evaluation}
The previous discussed approaches were tested and produced some interesting results. The both estimation algorithms seem to be "more sound" than the BPOR and
may explore traces that exceed the bound. This stems from the fact that they tend to underestimate the bound count since there are more complex relations 
between blocks that result traces with higher bound count than the one estimated. We notice that in writer-N-readers example the number of traces explored
is stable for every bound. In fact, each trace of this test has an equivalent trace with zero bound count since in each thread only a command related to
a shared variable is executed.

\subsection{Evaluation of Approximating algorithms}

\graph{/home/yannis/nidhugg/tests/mytests/wNrL1B.png}{writer-N-readers bounded by the first estimation algorithm}
\smalltabular{"/home/yannis/nidhugg/tests/mytests/lazy1_bounded.tex"}{Traces for the first estimation algorithm for various bound limits}


\graph{/home/yannis/nidhugg/tests/mytests/wNrL2B.png}{writer-N-readers bounded by the second estimation algorithm}
\smalltabular{"/home/yannis/nidhugg/tests/mytests/lazy2_bounded.tex"}{Traces for the second estimation algorithm for various bound limits}

\subsection{Implementation of LBPOR}

Some of the testcases made clear that an implementation of a bound count function which does not simply counts the preemptive switches in traces can prevent
the state space explosion caused by the conservative branches added. The next step is to implement the LBPOR, an algorithm that calculates the number of compulsory
preemptive switches (preemptive switches that cannot be avoided in any equivalent trace with the one examined). The main difference from the Vanilla-BPOR is that
the LBPOR maintains throughout the execution of the DPOR a graph of the blocks that are contained in the traces. When a new block is created, it is added by the
algorithm previously described. When it comes to the calculation of the bound count, the minimum hamiltonian path is calculated. The weight of this path corresponds
to the bound count taken into consideration.

\begin{algorithm}
    \caption{LBPOR}
    \label{LBPOR}
    \Let{$G =: \emptyset$}
    Explore($\langle \rangle$,$\emptyset$,$G$,$b$)\;
    \Fn{Explore($E$,$Sleep$,$G$,$b$)}{
        \If{$\exists p \in (enabled(s_{[E]}) \backslash Sleep)$ such that $B_v(E.p) <= b$ }{
            backtrack(E) $:={p}$ \;
            \While{$\exists p \in (backtrack(E) \backslash Sleep $}{
                \ForEach{$e \in dom(E)$ such that $e \lesssim_{E.p} next_{[E]}(p) $}{
                    \Let{$E' = pre(E,e)$}
                    \Let{$u = notdep(e,E).p$}
                    \If{$I_{E'}(u) \cap backtrack(E') = \emptyset$}{
                        add some $q' \in I_{[E']}(u) to backtrack(E') $ \;
                    }
                }
                \Let{$Sleep' := \{q \in Sleep \mid E \models p \diamondsuit q \} $}
                \If{$p$ creates a new block}{
                    \Let{$block$ = $last\_block(E)$}
                    \Let{$G'$ = add\_block($block$,$G$)}
                }
                \If{$min \{ Hamiltonian\_path(G') \} <= b $}{
                    $Explore(E.p, Sleep,G',b)$ \;
                    add $p$ to $Sleep$ \;
                }
            }
        }
    }
\end{algorithm}


\subsection{LBPOR - RCU Evaluation}

The results are demonstrated below. Since we have to compare LBPOR with BPOR the bugged versions of the DPOR must be used. The bugged version
of DPOR is that where the last running thread is prioritized.

Here we present the evaluation of the algorithm on RCU.

\bigtabular{"/home/yannis/rcu/valtree/lazy_comp.tex"}{Comparison between DPOR and LBPOR}

\bigtabular{"/home/yannis/rcu/valtree/lazy_buged_comp.tex"}{Comparison between DPOR and LBPOR without the bug}


We compare the algorithm with BPOR. We notice that LBPOR examines less traces but requires longer time since the cost of the lazy bound count is significantly
increased.

\bigtabular{"/home/yannis/rcu/valtree/preem_lazy_comp.tex"}{Comparison between BPOR and LBPOR(buged)}


\input{./ch6.tex}
\nocite{*}

%%%  Bibliography

\bibliographystyle{softlab-thesis}
\bibliography{thesis}


%%%  Appendices

\backmatter

\appendix

\chapter{Tree RCU Modified Functions \label{appendix_a}}

Below are listed some functions from \ccode{kernel/rcu/tree.c} file
that have been modified for the bug injection procedure (see Section
\ref{valtree_gp}). The code relevant to the bug injections has been
colored blue.

%\lstinputlisting[linerange={187-199,1530-1572,1597-1671,1958-2011,3554-3581},nolol=true,style=custom}]{/home/michalis/Dropbox/sxolh/thesis/rcu/valtree/v3.19/tree.c}

\begin{code_appendix}
  void rcu_sched_qs(void) 
  {@
  #ifdef LIVENESS_CHECK_2
	  return;
  #endif@
	  if (!rcu_sched_data[get_cpu()].passed_quiesce) {
		  trace_rcu_grace_period(TPS("rcu_sched"),
				         rcu_sched_data[get_cpu()].gpnum,
				         TPS("cpuqs"));
		  rcu_sched_data[get_cpu()].passed_quiesce = 1;
	  }
  }

  static bool __note_gp_changes(struct rcu_state *rsp, struct rcu_node *rnp,
			      struct rcu_data *rdp)
  {
	  bool ret;

	  /* Handle the ends of any preceding grace periods first. */
	  if (rdp->completed == rnp->completed) {

		  /* No grace period end, so just accelerate recent callbacks. */
		  ret = rcu_accelerate_cbs(rsp, rnp, rdp);

	  } else {

		  /* Advance callbacks. */
		  ret = rcu_advance_cbs(rsp, rnp, rdp);

		  /* Remember that we saw this grace-period completion. */
		  rdp->completed = rnp->completed;
		  trace_rcu_grace_period(rsp->name, rdp->gpnum, TPS("cpuend"));
	  }

	  if (rdp->gpnum != rnp->gpnum) {
		  /*
		   * If the current grace period is waiting for this CPU,
		   * set up to detect a quiescent state, otherwise don't
		   * go looking for one.
		   */
		  rdp->gpnum = rnp->gpnum;
		  trace_rcu_grace_period(rsp->name, rdp->gpnum, TPS("cpustart"));
		  rdp->passed_quiesce = 0;@
  #ifdef LIVENESS_CHECK_1
		  rdp->qs_pending = 0;
  #else
		  rdp->qs_pending = !!(rnp->qsmask & rdp->grpmask);
  #endif
  #ifdef FORCE_FAILURE_5
		  rnp->qsmask &= ~rdp->grpmask;
  #endif@
		  zero_cpu_stall_ticks(rdp);
	  }
	  return ret;
  }      

  static int rcu_gp_init(struct rcu_state *rsp)
  {
	  struct rcu_data *rdp;
	  struct rcu_node *rnp = rcu_get_root(rsp);

	  rcu_bind_gp_kthread();
	  raw_spin_lock_irq(&rnp->lock);
	  smp_mb__after_unlock_lock();
	  if (!ACCESS_ONCE(rsp->gp_flags)) {
		  /* Spurious wakeup, tell caller to go back to sleep.  */
		  raw_spin_unlock_irq(&rnp->lock);
		  return 0;
	  }
	  ACCESS_ONCE(rsp->gp_flags) = 0; /* Clear all flags: New grace period. */

	  if (WARN_ON_ONCE(rcu_gp_in_progress(rsp))) {
		  /*
		   * Grace period already in progress, don't start another.
		   * Not supposed to be able to happen.
		   */
		  raw_spin_unlock_irq(&rnp->lock);
		  return 0;
	  }

	  /* Advance to a new grace period and initialize state. */
	  record_gp_stall_check_time(rsp);
	  /* Record GP times before starting GP, hence smp_store_release(). */
	  smp_store_release(&rsp->gpnum, rsp->gpnum + 1);
	  trace_rcu_grace_period(rsp->name, rsp->gpnum, TPS("start"));
	  raw_spin_unlock_irq(&rnp->lock);

	  /* Exclude any concurrent CPU-hotplug operations. */
	  mutex_lock(&rsp->onoff_mutex);
	  smp_mb__after_unlock_lock(); /* ->gpnum increment before GP! */

	  /*
	   * Set the quiescent-state-needed bits in all the rcu_node
	   * structures for all currently online CPUs in breadth-first order,
	   * starting from the root rcu_node structure, relying on the layout
	   * of the tree within the rsp->node[] array.  Note that other CPUs
	   * will access only the leaves of the hierarchy, thus seeing that no
	   * grace period is in progress, at least until the corresponding
	   * leaf node has been initialized.  In addition, we have excluded
	   * CPU-hotplug operations.
	   *
	   * The grace period cannot complete until the initialization
	   * process finishes, because this kthread handles both.
	   */
	  rcu_for_each_node_breadth_first(rsp, rnp) {
		  raw_spin_lock_irq(&rnp->lock);
		  smp_mb__after_unlock_lock();
		  rdp = &rsp->rda[get_cpu()];
		  rcu_preempt_check_blocked_tasks(rnp);
  @#ifdef FORCE_FAILURE_3
		  rnp->qsmask &= ~rdp->grpmask;
  #else
		  rnp->qsmask = rnp->qsmaskinit;
  #endif@
		  ACCESS_ONCE(rnp->gpnum) = rsp->gpnum;
		  WARN_ON_ONCE(rnp->completed != rsp->completed);
		  ACCESS_ONCE(rnp->completed) = rsp->completed;
		  if (rnp == rdp->mynode)
			  (void)__note_gp_changes(rsp, rnp, rdp);
		  rcu_preempt_boost_start_gp(rnp);
		  trace_rcu_grace_period_init(rsp->name, rnp->gpnum,
					      rnp->level, rnp->grplo,
					      rnp->grphi, rnp->qsmask);
		  raw_spin_unlock_irq(&rnp->lock);
		  cond_resched_rcu_qs();
	  }

	  mutex_unlock(&rsp->onoff_mutex);
	  return 1;
  }

  static void
  rcu_report_qs_rnp(unsigned long mask, struct rcu_state *rsp,
		    struct rcu_node *rnp, unsigned long flags)
	  __releases(rnp->lock)
  {
	  struct rcu_node *rnp_c;

@  #ifdef LIVENESS_CHECK_3
	  return;
  #endif@
	  /* Walk up the rcu_node hierarchy. */
	  for (;;) {
		  if (!(rnp->qsmask & mask)) {

			  /* Our bit has already been cleared, so done. */
			  raw_spin_unlock_irqrestore(&rnp->lock, flags);
			  return;
		  }
		  rnp->qsmask &= ~mask;
		  trace_rcu_quiescent_state_report(rsp->name, rnp->gpnum,
						   mask, rnp->qsmask, rnp->level,
						   rnp->grplo, rnp->grphi,
						   !!rnp->gp_tasks);
@  #ifndef FORCE_FAILURE_6
		  if (rnp->qsmask != 0 || rcu_preempt_blocked_readers_cgp(rnp)) {

			  /* Other bits still set at this level, so done. */
			  raw_spin_unlock_irqrestore(&rnp->lock, flags);
			  return;
		  }
  #endif@
		  mask = rnp->grpmask;
		  if (rnp->parent == NULL) {

			  /* No more levels.  Exit loop holding root lock. */

			  break;
		  }
		  raw_spin_unlock_irqrestore(&rnp->lock, flags);
		  rnp_c = rnp;
		  rnp = rnp->parent;
		  raw_spin_lock_irqsave(&rnp->lock, flags);
		  smp_mb__after_unlock_lock();
		  WARN_ON_ONCE(rnp_c->qsmask);
	  }

	  /*
	   * Get here if we are the last CPU to pass through a quiescent
	   * state for this grace period.  Invoke rcu_report_qs_rsp()
	   * to clean up and start the next grace period if one is needed.
	   */
	  rcu_report_qs_rsp(rsp, flags); /* releases rnp->lock. */
  }

  static int __init rcu_spawn_gp_kthread(void)
  {
	  unsigned long flags;
	  struct rcu_node *rnp;
	  struct rcu_state *rsp;
	  struct task_struct *t;

	  rcu_scheduler_fully_active = 1;
@  #ifdef ENABLE_RCU_BH
	  for_each_rcu_flavor(rsp) {
		  t = kthread_run(rcu_gp_kthread, rsp, "%s", rsp->name);
		  BUG_ON(IS_ERR(t));
		  rnp = rcu_get_root(rsp);
		  raw_spin_lock_irqsave(&rnp->lock, flags);
		  rsp->gp_kthread = t;
		  raw_spin_unlock_irqrestore(&rnp->lock, flags);
	  }
  #else
	  t = kthread_run(rcu_gp_kthread, &rcu_sched_state, "%s",rcu_sched_state.name);
	  rnp = rcu_get_root(&rcu_sched_state);
	  raw_spin_lock_irqsave(&rnp->lock, flags);
	  rcu_sched_state.gp_kthread = t;
	  raw_spin_unlock_irqrestore(&rnp->lock, flags);
  #endif@
	rcu_spawn_nocb_kthreads();
	rcu_spawn_boost_kthreads();
	return 0;
  }
  early_initcall(rcu_spawn_gp_kthread);
\end{code_appendix}
%%%  End of document

\end{document}
