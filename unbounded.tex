\chapter{Unbounded DPOR}

In this chapter the Classic-DPOR and Source-DPOR algorithms are discussed. These algorithms differ in terms of the sufficient sets they use in order to cover the state space.
DPOR is based on persistent sets while Source-DPOR relies on source sets.

\section{Source-DPOR}
In this section Source-DPOR \cite{AbdullaAronisJohnssonSagonasDPOR2014} is presented. Even though Source-DPOR is posterior to Classic-DPOR, as the name suggests, we choose to present it first since any algorithm
presented in this thesis will be related to Source-DPOR. This is due to the fact that all the algorithms were implemented in Nidhugg which is based on Source-DPOR.

\SetKwProg{Fn}{Function}{}{}
\SetKwHangingKw{Let}{let}
\begin{algorithm}
    \caption{Source-DPOR}
    \label{Source}
    Explore($\langle \rangle$,$\emptyset$)\;
    \Fn{Explore($E$,$Sleep$)}{
        \If{$\exists p \in (enabled(s_{[E]}) \backslash Sleep)$}{
            $backtrack(E) :={p}$ \;
            \While{$\exists p \in (backtrack(E) \backslash Sleep)$}{
                \ForEach{$e \in dom(E)$ such that $e \lesssim_{E.p} next_{[E]}(p)$}{
                    \Let{$E' = pre(E,e)$}
                    \Let{$u = notdep(e,E).p$}
                    \If{$I_{E'}(u) \cap backtrack(E') = \emptyset$}{
                        add some $q' \in I_{[E']}(u) \text{ to } backtrack(E') $ \;
                    }
                }
                \Let{$Sleep' := \{q \in Sleep \mid E \models p \diamondsuit q \} $}
                $Explore(E.p, Sleep')$ \;
                add $p$ to $Sleep$ \;

            }
        }
    }
\end{algorithm}

\noindent Initially an arbitrary enabled and not sleeping
process is chosen and added to the $backtrack(E)$. 

Each step of the algorithm consists of two separate phases. 
During the first phase of the algorithm the race detection takes place.
The algorithm picks a process $p$ which can execute its next step i.e, $p \in enabled(s_{[E]})$ and appends it to the explored trace $E$.
Then the algorithm finds every event $e$ which is already contained in the explored trace ($e \in dom(E)$) and can be reversed with the next step of $p$.

This happens in order to explore the execution sequence where $p$ happens-before $e$.
This execution sequence consists of:
\begin{itemize} 
\item $E' \equiv pre(E,e)$: the subsequence $E'$ which consists of events scheduled before $e$ in $E$.
\item $u \equiv notdep(e,E).p$: the concatination $u$ of all the events that are scheduled after $e$ in $E$ but are independent with $e$ and $p$.
\item $proc(e)$ which is the id of the process that caused the event $e$.
\end{itemize}
Then the algorithm checks whether some process in $I_{[E']}(u)$ is already in $backtrack(E')$. 
If not, then a process in $I_{[E']}(u)$ is added to backtrack. 

In the exploration phase, the exploration starts from $E.p$. The important part is the calculation of the new sleep set at that step since some processes 
may have woken up.
If the next step of a process conflicts with the $next(p)$ then this process must wake up. As a result, the sleep set consists of the already sleeping processes
whose next steps do not interfere with the $next(p)$, i.e, $Sleep' := \{q \in Sleep \mid E \models p \diamondsuit q \} $ 
After finishing the exploration of $E.p$, $p$ is added to the sleep set because we want to refrain from executing an equivalent trace.

\section{Classic-DPOR}

Since Source-DPOR uses vector clocks to track events the DPOR using Clock Vectors \cite{FlanaganDPOR} variation will be used which is shown in Algorithm \ref{DPORV}.

\begin{algorithm}
    \caption{DPOR using Clock Vectors (Classic-DPOR)}
    \label{DPORV}
    \SetKwInOut{Input}{input}
    \SetKwInput{Initialization}{Explore($\emptyset, \lambda x. \bot$)}
    \SetKwHangingKw{Let}{let}
    \Fn{Explore($E$,$C$)}{
        \Let{$s := last(E)$}
        \For{all process $p$}{
            \If{$\exists i = max(\{ i \in dom(E) \mid E_i$ is dependent and may be co-enabled with $next(s,p)$ and $i \not \leq  C(p)(proc(E_i)) \} $}{
                \uIf{$p \in enabled(pre(E,i)))$}{
                    add $p$ to $backtrack(pre(E,i))$ \;
                }
                \Else{add $enabled(pre(E,i))$ to $backtrack(pre(E,i))$ \;}
            }
        }
        \If{$\exists p \in enabled(s)$}{
            $backtrack(s) := {p}$ \;
            \Let{$done = \emptyset$}
            \While{$\exists p \in (backtrack(s) \backslash done)$}{
               add $p$ to $done$ \;
               \Let{$t = next(s,p)$} 
                \Let{$E' = E.t$} 
                \Let{$cu = max \{ C(i) \mid i \in 1.. | S |$ and $E_i$ dependent with $t \}$} 
                \Let{$cu2 = cu [ p := | E' | ] $} 
               \Let{$C' = C [p:= cu2, |E'| := cu2 ]$} 
                $Explore(E',C')$ \;
            }
        }
    }
\end{algorithm}

A clock vector $C(p)$ is maintained through out the algorithm for each process $p$. $C(p_i) = \langle c_1, c_2, ..., c_m \rangle$ represents the clock vector of process $p_i$.
where $c_j$ is the index of the last transition by process $p_j$ such that
$c_j \rightarrow_s p_i$. Intuitively, the clock vector of a process $p_j$ informs the process $p_j$ about the execution steps of the other processes that 
happen-before the execution steps of $p_j$. The clock vectors are based on the Lamport's algorithm \cite{Lamp} and more details of their usage in DPOR can
be found here \cite{FlanaganDPOR}. We represent the initial state of all vector clocks as $\bot = \langle 0, ..., 0 \rangle$. With $C(p)(proc(E_i ))$ we represent the 
value of clock vector of process $p$ for the process in the i-th step of $E$.

The initial state of Classic-DPOR is an empty sequence of events with all vector clocks set to $\bot$.

The Classic-DPOR consists of two phases. The first phase is the race detection. During that phase, the next transition
of all processes $p$ is considered.
For each such transition $next(s, p)$ (which may be enabled or disabled
in s), the last transition dependent transition  $i$ in $E$ is computed.
The computation takes place in line 4.
If there exists such a transition $i$, there might be a race
condition or dependency between $i$ and $next(s, p)$, and hence
we might need to introduce a “backtracking point” in the
state $pre(S, i)$, i.e., in the state just before executing the
transition $i$. If $p$ is enabled then it is added as backtrack point. Otherwise the whole set of the 
enabled transitions is added as backtracks.

During the exploration phase a process $p$ in $enabled(s)$ is added to the backtrack point as in the Source-DPOR. 
Then the vector clocks are updated according to Lamport's algorithm.

\section{Comparison of Classic-DPOR and Source-DPOR}

As one can easily notice, both algorithms consist of the same two phases i.e., the race detection phase and the exploration phase. Moreover both
algorithms are based on vector-clocks, in spite of the fact that their usage is just implied in the Source-DPOR in line 6 and line 8 of Algorithm \ref{Source}.

The major difference lies in the race detection phase. During the race detection in Classic-DPOR all processes $p$ are considered, before even be scheduled, and, thus, many of them
may be backtracked. On the other hand, in Source-DPOR only the last scheduled process is considered. Moreover the two algorithms differ in the way they deal with
the case of non-enabled process in the backtrack addition. In the case of Classic-DPOR the whole enabled set will be added whereas none or just one process will
be added in the Source-DPOR. All in all, Source-DPOR takes advantage of Clock vectors more than Classic-DPOR does.

\section{Combining Classic-DPOR and Source-DPOR}

In order to take advantage of Nidhugg's infrastructure Algorithm \ref{NDPOR} will be used. As a result there is no need to add all
available threads when $p$ is not enabled. If no sufficient candidate is found
a candidate suggested by the Source-DPOR algorithm is added. This way of calculating persistent sets is considered to be more 
complex and thus a bad option \cite{Gode05}. However, source sets algorithm is closer to this approach.

\begin{algorithm}
    \caption{Nidhugg-DPOR}
    \label{NDPOR}
    \SetKwInOut{Input}{input}
    \SetKwHangingKw{Let}{let}
    Explore($\langle \rangle$,$\emptyset$)\;
    \Fn{Explore($E$,$Sleep$)}{
        \If{$\exists p \in (enabled(s_{[E]}) \backslash Sleep$}{
            backtrack(E) $:={p}$ \;
            \While{$\exists p \in (backtrack(E) \backslash Sleep)$}{
                \ForEach{$e \in dom(E)$ such that $e \lesssim_{E.p} next_{[E]}(p)$}{
                    \Let{$E' = pre(E,e)$}
                    \Let{$u = notdep(e,E).p$}
                    \Let{$CI = \{ i \in I_{E'}(u) \mid i \rightarrow p \}$}
                    \If{$CI \cap backtrack(E') = \emptyset$}{
                        \If{$CI \neq \emptyset$}{
                            add some $q' \in CI \text{ to } backtrack(E') $ \;
                        }
                        \Else{add some $q' I_{E'}(u) \text{ to } backtrack(E')$}
                    }
                }
                \Let{ $Sleep' := \{q \in Sleep \mid E \models p \diamondsuit q \} $ } 
                $Explore(E.p, Sleep)$ \;
                add $p$ to $Sleep$ \;
            }
        }
    }
\end{algorithm}

The Algorithm \ref{NDPOR} differs from Source-DPOR (Alg. \ref{Source}) in the calculation of the initials.
It is clear that persistent algorithm implemented differs from Source-DPOR in the calculation of the initials. 
Specifically a subset of the initials ($CI$) that happen-before $p$ is used. 
Intuitively in the case of a writer and two readers, both readers will be added to the branch since the first read does not
happen before the second one.
To generalize this idea: since Nidhugg does not enable us to create branches for $last(E)$, when at the scheduling phase, we add the 
branches later as in Source-DPOR (Alg. \ref{Source}). When a race is considered usually only the thread that causes the race will be added since 
$CI$ contains this thread only.

We will give an intuition why Nidhugg-DPOR calculates a persistent set and explain why when the algorithm finishes a persistent set will have been calculated in each step.
The interesting part is to show for every branch that is a write command all the process that conflict with this write command happen concurrently with this
command will be added to the branch set.

Let us assume two processes that are in race with $last(S)$.
\begin{itemize}

\item Case 1: at least one process contains a write command.
We know that the two processes will be inverted at some point. Since Nidhugg-DPOR ignores weak initials it will branch both processes. 
In Figure \ref{Construction of persistent sets in Nidhugg when there is a write process} we notice that processes $q$ and $r$ are inverted. In Source-DPOR only one of the two processes should be branched since
they share the same initials. However, in Nidhugg-DPOR this is not true since the $CI$ set does not contain steps from the other process.

\trace{nidhuggpersistent1.pdf}{Construction of persistent sets in Nidhugg when there is a write process}

\item Case 2: both processes are read operations.
Since we do not calculate $I$ but $CI$ the first read operation will not be considered as it does not happen before the second read operation and as result
both processes will be added to $backtrack$. In Figure \ref{Construction of persistent sets in Nidhugg when both are read processes} we notice that by calculating the $CI$ set when the race
between $p$ and $r$ is detected $q$ process will be ignored and, thus, $r$ will be added as a branch.

\trace{nidhuggpersistent.pdf}{Construction of persistent sets in Nidhugg when both are read processes}

\end{itemize}

It is clear that no process that does not belong to the $backtrack(S)$ has race with a process that belongs to $backtrack(S)$. The introduction of $CI$ set allows
as to ignore the initials that Source-DPOR takes advantage of in order to optimize the exploration.
